% This file was converted from HTML to LaTeX with
% Tomasz Wegrzanowski's <maniek@beer.com> gnuhtml2latex program
% Version : 0.1
\documentclass{article}
\begin{document}



  
  
    
      
      
      
    
  



\\
\\

\section*{Chapter35}


[\textit{Previous Chapter}]
[\textit{Contents}]
[\textit{Next Chapter}]


\section*{Layers}



In
Chapters 11
and
15
we examined different kinds of intervals,
including both harmonic and melodic intervals.
A number of different types of intervals were distinguished
and we learned how to calculate such intervals.
One type of melodic interval mentioned in
Chapter 11
is the
\textit{distance interval}
-$\,$-  an interval between pitches which are separated by
intervening musical materials.
In this chapter we consider more sophisticated ways of determining
distance intervals.
These types of intervals are the foundation of various notions
of hierarchies or "layers" of pitch analysis.

\par 
This chapter also visits a related issue of implied harmony.
Many melodic passages outline clear harmonic progressions
which are also implicated in layer-related analyses.


\subsection*{Implied Harmony}

\par 
Example 35.1 shows a two-phrase trumpet solo from Aaron Copland's
\textit{El Salon Mexico}.
Harmonic progressions may be evident only when arpeggiated figures
are collapsed.
In this case, an implicit harmony may is evident
where a G major chord is followed by a D dominant seventh chord.
The barlines provide convenient ways of parsing the harmonies.

\par 
\textbf{Example 35.1}  Aaron Copland, \textit{El Salon Mexico}.



A
\texttt{**kern}
encoding of the passage is given below:

\texttt{!!!COM: Copland, A.
\texttt{!!!OTL: El Salon Mexico
\texttt{**kern
\texttt{*Itromp
\texttt{*clefG2
\texttt{*k[]
\texttt{*M4/4
\texttt{=29
\texttt{2r
\texttt{8r
\texttt{\{8d
\texttt{8g
\texttt{8b
\texttt{=30
\texttt{28dd
\texttt{28b
\texttt{28dd
\texttt{28b
\texttt{28dd
\texttt{28b
\texttt{28dd
\texttt{8b
\texttt{8dd
\texttt{8gg
\texttt{8dd
\texttt{8b
\texttt{8g
\texttt{=31
\texttt{8cc
\texttt{[4.a
\texttt{8a]\}
\texttt{\{8d
\texttt{8f\#
\texttt{8a
\texttt{=32
\texttt{4cc
\texttt{8a
\texttt{8f\#
\texttt{8d
\texttt{4dd
\texttt{8dd
\texttt{=33
\texttt{8ff\#\}~
\texttt{8r
\texttt{4r
\texttt{2r;
\texttt{=
\texttt{*-}


\par 
We can collapse the arpeggiated chords using the
\textbf{context}
command:

\par 

\texttt{context -b = -o = copland}


\par 
Identify the chords is facilitated by using the pitch-class
(\texttt{**pc})
representation described in
Chapter 34.

\par 

\texttt{context -b = -o = copland | pc -a | rid -d}


\par 
The corresponding output is:

\par 

\texttt{!!!COM: Copland, A.
\texttt{!!!OTL: El Salon Mexico
\texttt{**pc
\texttt{*Itromp
\texttt{*clefG2
\texttt{*k[]
\texttt{*M4/4
\texttt{2 B 2 B 2 B 2 B 2 7 2 B 7
\texttt{0 9 9 2 6 9
\texttt{0 9 6 2 2 2
\texttt{6 r r r
\texttt{*-}

In order to identify these as G major and D dominant chords
it would be convenient to reduce the sets to (2,7,B)
and (0,2,6,9) respectively.
For this task, we can use a
The following awk script eliminates repeated tokens within a record:
(huniq:
We might call this script
\textbf{huniq}
since it acts like a horizontal version of the
\textbf{uniq}
command:

\par 

\texttt{awk '\{

\# A script to eliminate repeated tokens within a record.
\\
if (\$0 ~ /\^{}[!*]/) \{print \$0; next\}
\\
else

\{
array[\$1] = line = \$1
\\
for (i=2; i<=NF; i++)

\{
\\
if (array[\$i] == "") \{array[\$i]=\$i; line = line " " \$i\}
\\
\}

print line
\\
for (i in array) delete array[i]
\\
\}

\}' \$1}


Applying this script to our output

!!!COM: Copland, A.
!!!OTL: El Salon Mexico
**pc
*Itromp
*clefG2
*k[]
*M4/4
r 2 7 B
2 B 7
0 9 2 6
0 9 6 2
6 r
*-

Identifying implicit harmonic intervals can be a little
more taxing.
Let's begin by considering a monophonic passage that
exhibits a pseudo-polyphonic or compound melodic tendency.
A passage from Bach's "Gigue" from the solo 'cello
\textit{Suite No. 3}
is shown in Example 35.1.
\\\\
\textbf{Example 35.1}  J.S. Bach, "Gigue" from \textit{Suite No. 3} for solo 'cello (excerpt).




\texttt{**kern
\texttt{*M3/8
\texttt{=88
\texttt{(16F\#
\texttt{16c)
\texttt{(16E
\texttt{16c)
\texttt{(16D
\texttt{16c)
\texttt{=89
\texttt{(16B
\texttt{16D)
\texttt{(16A
\texttt{16D)
\texttt{(16B
\texttt{16D)
\texttt{=90
\texttt{(16c
\texttt{16D)
\texttt{(16B
\texttt{16D)
\texttt{(16A
\texttt{16D)
\texttt{=91
\texttt{(16B
\texttt{16D)}








\\
\begin{itemize}
\item 

\textbf{Next Chapter}
\item 

\textbf{Previous Chapter}
\item 

\textbf{Table of Contents}
\item 

\textbf{Detailed Contents}
\\\\

� Copyright 1999 David Huron
\end{document}
