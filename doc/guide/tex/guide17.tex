% This file was converted from HTML to LaTeX with
% Tomasz Wegrzanowski's <maniek@beer.com> gnuhtml2latex program
% Version : 0.1
\documentclass{article}
\begin{document}



  
  
    
      
      
      
    
  



\\
\\

\section*{Chapter17}


[\textit{Previous Chapter}]
[\textit{Contents}]
[\textit{Next Chapter}]


\section*{Creating Inventories}



Many research problems can be addressed by building an
\textit{inventory}
-$\,$-  that is, identifying the number of occurrences of various
types of data.
Questions such as the following all pertain to the generation of
inventories:
\begin{itemize}
\item 
Does Liszt use a greater variety of harmonies than Chopin?
\item 
What is the most frequently used dynamic marking in Beethoven,
and how does Beethoven's practice compare with that of Brahms?
\item 
Are flats more common than sharps in Monteverdi?
\item 
Did Bart�k's preferred articulation marks change over his
lifetime?
\item 
Is there a tendency to use the subdominant pitch less often in
pop melodies than in (say) French chanson?
\item 
How frequent are light-related words such as "lumen" or "lumine"
in the different monastic offices for Thomas of Canterbury?
\item 
Is it true that 90 percent of the notes in a given work by Bach
use just two durations (such as eighths and sixteenths, or
eighths and quarters)?
\item 
What is the most common instrumental combination for orchestral works
by Mussorgsky?
\end{itemize}

\par 
At the end of this chapter we will show the Humdrum
commands needed to answer each of the above questions.

\par 
The above questions are all variations of one of the following forms:

\par 

How many different types of \_\_\_\_\_ are there?
\\
What is the most/least common \_\_\_\_\_?
\\
What is the frequency of occurrence for various \_\_\_\_\_s?


\par 
In some cases, we're asked to compare two or more repertories when
answering one of these basic questions.

\par 
For illustration purposes, consider the case of a Humdrum file
named \texttt{alpha} containing the following simple input:
\\\\

\texttt{**alpha
\texttt{A
\texttt{B
\texttt{A
\texttt{A
\texttt{C
\texttt{B
\texttt{*-}

It doesn't matter what the data represent.
The "\texttt{A}", "\texttt{B}", and "\texttt{C}" might
signify different articulation marks, chords, harmonic intervals, or
instrumental configurations.
Whatever is represented, the process of generating an inventory
is the same.
Ultimately, we'd like to produce a simple distribution that indicates:

\par 

\texttt{3 occurrences of "A"
\\
2 occurrences of "B"
\\
1 occurrence  of "C"}


\par 

\subsection*{Filter, Sort, Count}

\par 
Building an inventory is a three-step process.
First we need to
\textit{filter}
the input so only the data of interest is present.
Second we need to
\textit{sort}
like-with-like.
And third we need to
\textit{count}
the number of occurrences of each type of data token.

\par 
Let's begin by discussing the second process.
In
Chapter 3
we saw how the UNIX
\textbf{sort}
command will rearrange lines of data so that they are in
alphabetical/numerical order.
The command:

\par 

\texttt{sort alpha > sorted.alpha}


\par 
will sort the file \texttt{alpha} and place the results in a file named \texttt{sorted.alpha}.
The file \texttt{sorted.alpha} will contain the following:
\\\\

\texttt{**alpha
\texttt{*-
\texttt{A
\texttt{A
\texttt{A
\texttt{B
\texttt{B
\texttt{C}

Notice that the asterisk is treated as alphabetically prior to the
the letter `A', so all the Humdrum interpretation records have been
moved to the beginning of the output.
Notice also that all of the lines beginning with the letter `A'
are now collected on successive lines in
the output.
Similarly, the `B's have been rearranged on successive lines.

\par 
The third step in generating an inventory is to count the number of
occurrences of each unique data token.  The
\textbf{uniq}
command described in
Chapter 3
will eliminate successive duplicate
lines.
For example, if we type:

\par 

\texttt{uniq sorted.alpha}


\par 
The output will be as follows:
\\\\

\texttt{**alpha
\texttt{*-
\texttt{A
\texttt{B
\texttt{C}

Notice that repetitions of the data "A" and "B" have disappeared.
The simple
\textbf{uniq}
command is useful for telling us
\textit{how many different things}
there are in an input.
For example, the above output identifies just five different
records -$\,$- and three different types of data records.

\par 
Recall that the
\textbf{-c}
option for
\textbf{uniq}
will cause a `count' to be prepended to each output line.
The command:

\par 

\texttt{uniq -c sorted.alpha > unique.alpha}


\par 
will produce the following output:
\\\\

\texttt{1**alpha
\texttt{1*-
\texttt{3A
\texttt{2B
\texttt{1C}

The prepended counts tell us that `A' occurs three times, `B'
occurs twice, and all other records occur just once.

\par 
In the above output, \texttt{**alpha}, and \texttt{*-} are
Humdrum interpretations rather
than data, so we probably don't want them to appear in our inventory.
If our file had contained comments, or null data records,
then these would have also appeared in our output,
although we are not likely to be interested in them.
This leads us to what is normally the first step in generating
an inventory -$\,$-
\textit{filtering}
the input in order to eliminate records
that we'd prefer to omit from our final output.

\par 

\subsection*{Filtering Data with the \textit{rid} Command}

\par 
As we saw in
Chapter 13,
the
\textbf{rid}
command can be used to eliminate various classes
of Humdrum records.
For example, \textbf{rid -G} eliminates all global comments;
\textbf{rid -D} eliminates all data records, etc.
The option combination \textbf{-GLId} is very common with
\textbf{rid}
since only data records are retained in the output.
That is, eliminating all global and local comments,
omitting all interpretations, and deleting all null data records
will result in an output consisting only of non-null data records.

\par 
Returning to our \texttt{**alpha} data, we can eliminate everything
but data records as follows:

\par 

\texttt{rid -GLId alpha > filtered.alpha}


\par 
By way of summary, generating an inventory is a three-step process.
First we
\textit{filter}
the input so only the data of interest is present.
Typically, this means using the
\textbf{rid}
command with one or more options to eliminate comments,
interpretations, and perhaps null data records.
Second we
\textit{sort}
the data using the
\textbf{sort}
command so that identical records are amalgamated as
neighbors.
Finally, we use the
\textbf{uniq -c}
to
\textit{count}
the number of occurrences of each type of data token.
All three steps can be amalgamated into a single pipeline:

\par 

\texttt{rid -GLId alpha | sort | uniq -c > inventory.alpha}


\par 
Notice that the inventory will pertain to whatever data was provided
in the original input.  We've been using the abstract data "A", "B",
and "C".
However, this data might represent any type of discrete
data, such as Latin text, piano fingerings, or dance steps.


\subsection*{Inventories for Multi-spine Inputs}

\par 
In the above example, we assumed that the input consists of a single
Humdrum spine (i.e. a single column of data).
However, Humdrum files can have any number of spines,
and each spine might represent radically different types of data.
For example, the following file (named \texttt{alphabet})
contains two spines, one with "alpha" data, and the second
with "bet" data.
These data types might represent melodic intervals
and fingering information, or dynamic markings and stem-directions,
or whatever.
\\\\

\texttt{**alpha**bet
\texttt{A\$50
\texttt{B\$50
\texttt{A\$50
\texttt{A\$200
\texttt{C\$50
\texttt{B\$50
\texttt{*-*-}

If we apply our above inventory-generating commands for the
file "alphabet," the result will be as follows:
\\\\

\texttt{1A\$200
\texttt{2A\$50
\texttt{2B\$50
\texttt{1C\$50}

Notice that the inventory is based on
\textit{entire records}
containing both "alpha" and "bet" data.
This is the reason why the alpha-bet
data-pair "\texttt{A   \$50}"  is
considered different from alpha-bet data "\texttt{A   \$200}".
Depending on the user's goal, this may or may not be the most
appropriate output.

\par 
A situation where this approach might be desired arises when we are
counting the number of different spellings of chords (e.g., how
many different sonorous arrangements are there?).
If **alpha and **bet represent pitches in two concurrent voices,
then it may be important to have both concurrent data tokens
participating in the inventory.

\par 
In other circumstances, we may not want this.
For example, if we are interested only in alpha-related data,
we need to eliminate the irrelevant **bet data so it won't interfere.
This can be done using the Humdrum
\textbf{extract}
command.

\par 
For example, we can create an inventory of just the \texttt{**bet} data:

\par 

\texttt{extract -i '**bet' alphabet | rid -GLId | sort | uniq -c $\backslash$

> inventory.bet}



\par 
The resulting \texttt{inventory.bet} file will contain:
\\\\

\texttt{1\$200
\texttt{5\$50}

-$\,$-  meaning 5 occurrences of the data "\$50" and 1 occurrence
of "\$200".

\par 
Sometimes it is useful to create an aggregate inventory of the
data in each separate spine.
In such cases, we will need to use
\textbf{extract}
several times so that each spine is placed in a separate file:

\par 

\texttt{extract -i '**alpha' alphabet > justalpha}
\\
\texttt{extract -i '**bet' alphabet > justbet}


\par 
The
\textbf{cat}
command can then be used to concatenate the files end-to-end
so they form a single column of data.
With each data token of interest is on its own line, we can
generate the appropriate inventory:

\par 

\texttt{cat justalpha justbet | rid -GLId | sort | uniq -c}



\subsection*{Sorting By Frequency of Occurrence}

\par 
When the output inventory list is short,
it is easy to identify which records are
the most common and which records are the least common.
Frequently inventory lists will contain dozens or hundreds of items
so it may be more difficult to scan through the output to find the
most frequent or least frequent occurrences.
For such long outputs, it might be more convenient to produce an output
sorted according to frequency of occurrence.
Notice that each output record from
\textbf{uniq -c}
begins with a number, and so the output is ideally
suited for numerical sorting.
We've already learned that the
\textbf{sort}
command rearranges input records in alphabetic/numeric order.

\par 
If we type

\par 

\texttt{sort inventory.alpha}


\par 
The output will be as follows:
\\\\

\texttt{1C
\texttt{2B
\texttt{3A}

Now the output is sorted so that the least frequent occurrences are
at the beginning, and the most frequent occurrences are at the end
of the output.
Incidentally,
\textbf{sort}
has a
\textbf{-r}
option that causes the output to be sorted in reverse order.
If we use \textbf{sort -r}, then the most common occurrences will
be placed at the beginning of the output:

\par 

\texttt{sort -r inventory.alpha}


\par 
produces the following output:
\\\\

\texttt{3A
\texttt{2B
\texttt{1C}

Once again, we can amalgamate all of the required commands into a
single pipeline.
The following pipeline produces an inventory
for any type of Humdrum input, sorted from the most common to the
least common data:

\par 

\texttt{rid -GLId alpha | sort | uniq -c | sort -r > inventory.alpha}



\subsection*{Counting with the \textit{wc} Command}

\par 
In other circumstances, it may be helpful to determine the proportion
or percentage values rather than the actual numerical count.
This can be calculated by dividing each of the inventory count
numbers by the total number of data records processed.
A convenient way to count records is via the UNIX
\textbf{wc}
(word count) command.
The
\textbf{wc}
command provides three options.
With the
\textbf{-c}
option,
\textbf{wc}
counts the number of characters in an input.
With the
\textbf{-w}
option,
\textbf{wc}
counts the number of words in an input.
A "word" is defined as any sequence of
characters delineated by white space, such as
spaces, tabs or new lines.
With the
\textbf{-l}
option,
\textbf{wc}
counts the nmber of lines or records in the input.

\par 
We can count the total number of non-null data records in a
Humdrum input using the following pipeline:

\par 

\texttt{rid -GLId alpha | wc -l}


\par 
This will give us the total number of items in our inventory.
Simple division will generate the percentages for each type
of data record.

\par 
Suppose, for example, that the total number of data records
was determined to be 874.
Using the UNIX
\textbf{awk}
command will allow us to easily generate the percentages for each
data type via the command:

\par 

\texttt{awk '\{print \$1/847*100 "$\backslash$t" \$2\}' inventory.alpha}


\par 
This will create a two-column output.
The first column will indicate the percentage of occurrence,
and the second column will identify the corresponding type of data.


\subsection*{Excluding or Seeking Rare Events}

\par 
Recall from
Chapter 3
that the
\textbf{uniq}
command provides other options (besides the
\textbf{-c}
option).
The
\textbf{-d}
option causes
\textbf{uniq}
to output
\textit{only}
those records that are duplicated.
In other words, records that occur only once are eliminated
from the input.
This option can be useful when there are a lot of
single-occurrence data tokens and you are only interested in
those data records that occur more frequently.

\par 
By contrast, the
\textbf{uniq -u}
option causes
\textit{only}
those records that are unique (occur only once) to be output.
This option can be useful when looking for rare
circumstances in our data.

\par 

\texttt{rid -GLId alpha | sort | uniq -u}        (output only the rare events)
\\
\texttt{rid -GLId alpha | sort | uniq -d}        (eliminate all the rare events)



\subsection*{Transforming and Editing Inventory Data}

\par 
Notice that two data records must be identical in order for them
to be considered "the same" by
\textbf{sort}
and \textbf{uniq}.
This means that records such as the following are considered
entirely different:
\\\\

\texttt{ABC
\texttt{abc
\texttt{Abc
\texttt{"ABC"
\texttt{ABC.
\texttt{CBA}

Remember that step \#1 in generating inventories requires that we
filter the data so only the data of interest is passed to
\textbf{sort}
and \textbf{uniq}.
This means we must be careful about the state of the input.
Depending on your goal, we will either want to
\textit{translate}
the input to some other more appropriate representation, or
\textit{edit}
the existing representation in order to discard or transform
otherwise confounding data.

\par 
\textit{Translating}
data involves changing from one type of information
to another -$\,$- that is, changing the exclusive interpretations.
For example, if we want to produce an inventory of melodic intervals,
then we might use the
\textbf{mint}
or
\textbf{xdelta}
commands to generate a suitable representation.
Alternatively, we might want to generate an inventory
of scale degrees using the
\textbf{deg}
or
\textbf{solfa}
commands.

\par 
Instead of translating our data, we might wish to edit
the data using the
\textbf{sed}
or
\textbf{humsed}
stream editors.
Suppose we had a file (named "notes") consisting of
pitch information, and we wanted to create an inventory of
the diatonic pitch-letter names.
Our input might look like this:
\\\\

\texttt{**notes
\texttt{A
\texttt{B
\texttt{B
\texttt{D
\texttt{F\#
\texttt{D\#
\texttt{E
\texttt{*-}

Without modification, our inventory would appear as follows:
\\\\

\texttt{1A
\texttt{2B
\texttt{1D
\texttt{1D\#
\texttt{1E
\texttt{1F\#}

But this inventory distinguishes D-sharp from D-natural -$\,$- which
is not what we want.  The answer is to filter our input so that
the sharps are removed.

\par 
Adding the appropriate
\textbf{humsed}
command to our pipe:

\par 

\texttt{humsed 's/\#//' notes | rid -GLId | sort | uniq -c}


\par 
-$\,$-  will produce the following output:
\\\\

\texttt{1A
\texttt{2B
\texttt{2D
\texttt{1E
\texttt{1F}


\par 

\subsection*{Further Examples}

\par 
Given your current background, you should now be able to
generate inventories to answer a wide variety of questions.
You should now understand how the commands given below can be used
to solve the question posed:


\par 
\textit{Does Liszt use a greater variety of harmonies than Chopin?}

\par 

\texttt{extract -i '**harm' liszt* | rid -GLId | sort | uniq | wc -l}
\\
\texttt{extract -i '**harm' chopin* | rid -GLId | sort | uniq | wc -l}



\par 
\textit{What is the most frequently used dynamic marking in Beethoven,}
\textit{and how does Beethoven's practice compare with that of Brahms?}

\par 

\texttt{extract -i '**dynam' beeth* | rid -GLId | sort | uniq -c $\backslash$

| sort -r | head -1}

\texttt{extract -i '**dynam' brahm* | rid -GLId | sort | uniq -c $\backslash$

| sort -r | head -1}




\par 
\textit{Are flats more common than sharps in Monteverdi?}
Let's presume that the input is monophonic \texttt{**kern} data.

\par 

\texttt{humsed 's/[\^{}\#-]//g' montev* | rid -GLId | sort | uniq -c}



\par 
\textit{Did Bart�k's preferred articulation marks change}
\textit{over his lifetime?}
Assume that copies of early and late works have been
concatenated to the files \texttt{early} and \texttt{late}.
The
\textbf{humsed}
command here eliminates all data with the exception of
\texttt{**kern}
articulation marks.
(See
Chapter 6
for details on **kern articulation marks.)

\par 

\texttt{extract -i '**kern' early | humsed 's/[\^{}"`~\^{}:I]//g' $\backslash$

| rid -GLId | sort | uniq -c}

\texttt{extract -i '**kern' late | humsed 's/[\^{}"`~\^{}:I]//g' $\backslash$

| rid -GLId | sort | uniq -c}




\par 
\textit{Is there a tendency to use the subdominant pitch less often}
\textit{in pop melodies than in (say) French chanson?}
Once again assume that the inputs are monophonic.

\par 

\texttt{deg -t pop* | grep -c '4'}
\\
\texttt{deg -t chanson* | grep -c '4'}



\par 
\textit{How frequent are light-related words such as "lumen"}
\textit{or "lumine" in the different monastic offices for}
Thomas of Canterbury?
Familiarity with regular expressions helps:

\par 

\texttt{extract -i '**text' office* | egrep -ic 'lum.+n[e]*\$'}



\par 
\textit{Is it true that 90 percent of the notes in a given work by Bach}
\textit{use just two durations (such as eighths and sixteenths, or}
\textit{eighths and quarters)?}

\par 

\texttt{humsed 's/[\^{}0-9.]//g' bach | rid -GLId | sort | uniq -c}

(Repeat the above command for each work and inspect the results.)



\par 
\textit{What is the most common instrumental combination for sonorities}
\textit{by Mussorgsky?}

\par 

This problem is addressed in
Chapter 36.




\subsection*{Reprise}

\par 
In this chapter we have discussed how to answer questions
that involve the creation of inventories.
Creating an inventory typically entails
\textit{filtering}
some data so only the information of interest is preserved,
\textit{sorting}
the data so that like data are amalgamated,
and then
\textit{counting}
each instance of each data type.

\par 
In later chapters we will see how classifying data,
identifying musical contexts, and marking occurrences
of patterns can be used to significantly enhance
the inventory-building tools described in this chapter.

\\
\begin{itemize}
\item 

\textbf{Next Chapter}
\item 

\textbf{Previous Chapter}
\item 

\textbf{Table of Contents}
\item 

\textbf{Detailed Contents}
\\\\

� Copyright 1999 David Huron
\end{document}
