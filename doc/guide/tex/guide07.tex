% This file was converted from HTML to LaTeX with
% Tomasz Wegrzanowski's <maniek@beer.com> gnuhtml2latex program
% Version : 0.1
\documentclass{article}
\begin{document}



  
  
    
      
      
      
    
  



\\
\\

\section*{Chapter7}


[\textit{Previous Chapter}]
[\textit{Contents}]
[\textit{Next Chapter}]


\section*{MIDI Output Tools}



MIDI is a standard method for exchanging information
between sound synthesizers and controlling computers.
Humdrum provides a number of MIDI-related tools that allow
MIDI data to be input and output.
In this chapter we will discuss the output-related tools:
\textbf{midi},
\textbf{perform},
\textbf{smf}
and
\textbf{tacet}.
MIDI input tools are discussed in
Chapter 30.


\subsection*{The \textit{**MIDI} Representation}

\par 
Humdrum provides a
\texttt{**MIDI}
representation that closely
parallels the commercial MIDI specification but conforms
to the Humdrum data format.
This representation provides an intermediate format;
using Humdrum commands such as
\textbf{perform}
and
\textbf{smf},
true MIDI data can be generated.
However, since the \texttt{**MIDI} representation conforms to the
Humdrum syntax, the data can be manipulated, modified and searched
in the same way as other Humdrum data.
For example, we can use
\textbf{grep}
to search MIDI data, etc.

\par 
MIDI is a type of "tablature" notation.
It describes a set of performance actions rather than specifying
either the sounded result or the analytic notation.
MIDI represents note-related events for
various "channels."
MIDI events include note-on, note-off, key-velocity,
after-touch, control codes, and system-exclusive codes.
The original commercial MIDI standard is unable to represent
many other musically pertinent signifiers
such as pitch spelling (e.g., F-sharp versus G-flat),
stem directions, ties, slurs, phrasing, etc.
In addition, MIDI does not represent rests.

\par 
A simple
\texttt{**MIDI}
example is given below.
It consists just a single note (middle C):

\texttt{!! A single MIDI note.
\texttt{**MIDI
\texttt{*Ch1
\texttt{54/60/64
\texttt{80/-60/64
\texttt{*-}


\par 
Notice that there are two \texttt{**MIDI} data tokens:
one to specify note-on and one to specify note-off events.
Each \texttt{**MIDI} data token consists of three elements or components,
delimited by a slash character (/).
The first element in the data token represents the number of clock
ticks before the current event is to occur.
The absolute duration of a single clock tick is determined by the
\texttt{**MIDI} clock speed, which is variable.
In our example, 54 clock ticks elapse before the note is turned on,
and another 80 clock ticks elapse before the note is turned off.
The initial pause before the first note begins is not necessary;
however, many MIDI cards introduce a brief delay before beginning to send
the first data in a stream.
The leading pause prevents a "rushed" burst of initial notes.

\par 
The second element in a data token represents the
\texttt{**MIDI} key number.
Successive numbers refer to semitone pitches where middle C
is designated key \texttt{60}.
Key-off events are represented by negative integers, so \texttt{-60}
means turn off key \texttt{60}.
Key numbers range between 1 and 127.

\par 
The third element in a data token represents the
\texttt{**MIDI} key velocity.
MIDI instruments normally interpret key velocity as
dynamic or accent information.
Higher key-velocity values are associated with accented notes.
In the case of key-off events, the key-velocity component represents
the key-up velocity.

\par 
Unlike most other Humdrum representations, the
\texttt{**MIDI}
format requires two data tokens for each note.
The following example shows a three note C-major triad.
The
\texttt{**kern}
data is shown in the left spine
with the corresponding \texttt{**MIDI} data in the right spine.
Notice that each data token consists of three subtokens,
one for each note:

\texttt{!! A C-majortriad.
\texttt{**kern**MIDI
\texttt{**Ch1
\texttt{2c 2e 2g72/60/64 72/64/64 72/67/64
\texttt{.144/-60/64 144/-64/64 144/-67/64
\texttt{*-*-}


\par 
Middle C corresponds to MIDI key 60;
The pitches E4 and G4 correspond to MIDI keys 64 and 67
respectively.
The half-note durations result in a delay of 144 clock ticks
before the key-off commands are executed.
Notice that all key-off events occur simultaneously.

\par 
Suppose we add a second chord consisting of the dyad D4 and F4.
The dyad begins at the same time that the C major triad ends.
This results in five subtokens in the corresponding \texttt{**MIDI}
data record.
Three key-off events are synchronous with two key-on events:

\texttt{!! Two chords.
\texttt{**kern**MIDI
\texttt{**Ch1
\texttt{2c 2e 2g72/60/64 72/64/64 72/67/64
\texttt{4d 4f144/-60/64 144/-64/64 144/-67/64 144/62/64 144/65/64
\texttt{.72/-62/64 72/-65/64
\texttt{*-*-}


\par 
Notice that the difference in duration between the half-notes
and quarter-notes is reflected when the notes are
turned \textit{off} rather than when the notes are turned \textit{on}.

\par 
Example 7.1 illustrates a slightly more complex excerpt from the beginning
of Darius Milhaud's \textit{Touches Blanches}.
\\\\
\textbf{Example 7.1}  Excerpt from Darius Milhaud's \textit{Touches Blanches}





\texttt{!!!: Milhaud, D.
\texttt{!!!OTL: TouchesBlanches
\texttt{**kern**kern**MIDI**MIDI
\texttt{*staff2*staff1*Ch1*Ch1
\texttt{*clefF4*clefG2*clefF4*clefG2
\texttt{*k[]*k[]*k[]*k[]
\texttt{*M3/4*M3/4*M3/4*M3/4
\texttt{=1-=1-=1-=1-
\texttt{4e(4g72/64/6472/67/64
\texttt{4c2a)72/-64/64 72/60/6472/-67/64 72/69/64
\texttt{4F.72/-60/64 72/53/64.
\texttt{=2=2=2=2
\texttt{4f(8a72/-53/64 72/65/6472/-69/64 72/69/64
\texttt{.8b.36/-69/64 36/71/64
\texttt{4d2g)36/-65/64 36/62/6436/-71/64 36/67/64
\texttt{4G.72/-62/64 72/55/64.
\texttt{=3=3=3=3
\texttt{4e(4g72/-55/64 72/64/6472/-67/64 72/67/64
\texttt{4c2a)72/-64/64 72/60/6472/-67/64 72/69/64
\texttt{4F.72/-60/64 72/53/64.
\texttt{=4=4=4=4
\texttt{4f(8a72/-53/64 72/65/6472/-69/64 72/69/64
\texttt{.8b.36/-69/64 36/71/64
\texttt{4d2g)36/-65/64 36/62/6436/-71/64 36/67/64
\texttt{4G.72/-62/64 72/55/64.
\texttt{..72/-55/6472/-67/64
\texttt{*-*-*-*-}

The
\texttt{**MIDI}
representation always expects a tandem interpretation
indicating the MIDI channel assignment.
In Example 7.1 both parts have been assigned to channel 1.
Once again, simultaneous key-on and key-off events often appear as double-stops.
Also notice that an additional data record is required at the end
of the passage in order to turn off the final notes.


\subsection*{The \textit{midi} Command}

\par 
The
\textbf{midi}
command converts Humdrum
\texttt{**kern}
data into Humdrum \texttt{**MIDI} data.
By way of example, the above \texttt{**MIDI} data can be generated
as follows:

\par 

\texttt{midi inven05.krn > inven05.hmd}


\par 
The \texttt{.hmd} filename extension is a common way of designating
Humdrum \texttt{MIDI} data.

\par 
Since the \texttt{**kern} representation does not encode key-velocity
information, the
\textbf{midi}
command assumes a default key velocity of 64
(from a range of 1 to 127).
If the input is monophonic,
\textbf{midi}
will also allow the user to set a fixed note duration using the
\textbf{-d}
option.
This is useful for auditing notes that do not have duration values.
For example, a Gregorian chant might be represented without durations.
The following command takes a file containing a 12-tone row (pitch information only)
and produces a \texttt{**MIDI} output where all notes assigned to a
quarter duration:

\par 

\texttt{midi -d 4 tonerow > tonerow.hmd}


\par 
The most common use of \texttt{**MIDI} data is to create a
standard MIDI file using the
\textbf{smf}
command, or to listen to the output using the
\textbf{perform}
command.
In some cases, it is useful to carry out
processing of \texttt{**MIDI} data itself.


\subsection*{The \textit{perform} Command}

\par 
The
\textbf{perform}
command allows the user to listen to synthesized performances of
\texttt{**MIDI}-format input.
When invoked,
\textbf{perform}
provides a simple interactive environment suitable for
proof-listening and other audition tasks.

\par 
The
\textbf{perform}
command accepts any Humdrum input; however,
only
\texttt{**MIDI}
spines present in the input stream are performed.
Non-MIDI spines are simply ignored and do not affect the sound output.
The
\textbf{perform}
command generates serial MIDI data which are sent directly to a
MIDI controller card or on-board sound-card.

\par 
The
\textbf{perform}
command is typically the last command in a pipe preceded by the
\textbf{midi}
command.
For example, a \texttt{**kern}-format score can be heard
using the following command:

\par 

\texttt{midi clara.krn | perform}


\par 
When invoked, the
\textbf{perform}
command reads in the entire input into memory.
This allows the user to move freely both forward and backward
through the MIDI score.

\par 
The
\textbf{perform}
command provides a set of interactive commands that allow the user
to pause and resume playback, to change tempo,
to move to any measure by absolute or relative reference,
and to search forward or backward for commented markers.
The
\textbf{perform}
command remains active until either the end of the score is reached
or the user terminates performance by typing the letter `q'
or the escape key (ESC).

\par 
Playback can be paused by typing the space-bar
and resumed by typing any key.
Typing the carriage return by itself will return to the beginning of the score
and re-initiate playback.
If a number is typed before pressing the carriage return then
\textbf{perform}
will search for a corresponding measure number and initiate playback
from that measure.
Other commands are provided that allow moving forward or
backward a specified number of measures.

\par 
In the default operation,
\textbf{perform}
echoes all global comments on the screen as the comments are
encountered in the input.
For inputs containing appropriate annotations, the echoing of comments
can provide useful visual markers or reminders of particular
moments in the sound output.
Whether or not global comments are echoed on the standard output,
users can use the
\textbf{perform}
forward-search (\texttt{/}) or backward-search (\texttt{?}) commands
to move directly
to a particular commented point in the score.
For example, if an input contains a global comment containing
the character string "\texttt{Second theme},"
then the user can move immediately to this position in the input by
entering the following command:

\par 

\texttt{/Second theme}


\par 
Similarly, backward searches can be carried out by typing the
question mark (\texttt{?}) rather than the slash.
If the search is successful, playback continues immediately from
the new score position.


\subsection*{Data Scrolling During Playback}

\par 
The
\textbf{midi}
command provides a useful
\textbf{-c}
option that causes each data record to be repeated as a comment.
For example, when the
\textbf{-c}
option is used a sequence of data records such as the following:

\texttt{4C4E4G4c
\texttt{4D4F4G4B
\texttt{4AA4E4A4c}


\par 
is transformed to:

\texttt{4C4E4G4c
\texttt{!!4C4E4G4c
\texttt{4D4F4G4B
\texttt{!!4D4F4G4B
\texttt{4AA4E4A4c
\texttt{!!4AA4E4A4c}



\par 
Since, by default, the
\textbf{perform}
command echoes all global comments on the screen during playback,
this means that the Humdrum data will also appear on the screen
as it is being played.
In addition, the commented data records are accessible to the
forward- and backward-search commands.
For example, in the
\texttt{**kern}
representation, pauses
are indicated by a semicolon;
hence the user might search for the next pause symbol by typing:

\par 

\texttt{/;}



\par 
Similarly, the user could search for a particular pitch, e.g.

\par 

\texttt{/gg\#}



\par 
Since the
\textbf{perform}
command accepts any Humdrum input, other Humdrum data may be
used for searching.
For example, the input data might contain melodic interval data
(see
Chapter 11),
allowing the user to search for a
particular interval such as a diminished octave:

\par 

\texttt{/d8}


\par 
If the string pattern is found in the input,
\textbf{perform}
will move immediately forward (or backward) to the next occurrence
and begin playing from that point.


\subsection*{Changing Tempo}


\par 
During playback, the tempo can be modified by typing the
greater-than (>) and less-than (<) signs
to increase or decrease the tempo respectively.
In addition to modifying the tempo interactively,
the performance tempo may be specified either in the command line
or in the input Humdrum representation.
The tempo may be specified on the command line by using the
\textbf{-t}
option.
For example, the following command causes the file \texttt{Andean}
to be performed at half tempo:

\par 

\texttt{midi Andean | perform -t 0.5}


\par 
Performing at fast speeds can often be useful when scanning
for a particular passage.

\par 
Tempo specifications may be present in the input data
via  the tandem interpretation for metronome marking (e.g. \texttt{*MM96}).
If no tempo information is available,
\textbf{perform}
uses a default tempo of 66 quarter-notes per minute.


\subsection*{The \textit{tacet} Command}

\par 
In rare circumstances, ciphers (stuck notes) can occur during MIDI
performances;
for instance, an intermittently functioning MIDI cable may
fail to convey a "note-off" instruction to an active synthesizer.
The
\textbf{p}
command ("panic") turns off all active notes.
Should a cipher remain after terminating the
\textbf{perform}
command, the Humdrum
\textbf{tacet}
command can be used to send
"all-notoff" commands on all MIDI channels.

\par 
In
Chapter 12,
we will see how
\textbf{perform}
can be used in conjunction with other commands (such as
\textbf{extract}
and
\textbf{yank})
to listen selectively to specific parts or passages.
In
Chapter 21
we will use
\textbf{perform}
in conjunction with the
\textbf{patt}
command to listen to patterns (such as harmonic, rhythmic and melodic
patterns) found in some repertory.


\subsection*{The \textit{smf} Command}

\par 
Another MIDI-related tool is the
\textbf{smf}
command.
This command allows the user to create "standard MIDI
files" from Humdrum \texttt{**MIDI}-format files.
Standard MIDI files are industry-standard binary files that
can be imported by a variety of MIDI applications software packages
on many different platforms,
including sequencer programs and most music notation packages.

\par 
The
\textbf{smf}
command translates only \texttt{**MIDI} input spines;
all non-\texttt{**MIDI} spines are simply ignored.
Suppose we begin with a \texttt{**kern}-format file named \texttt{joplin}.
We can create a standard MIDI file as follows:

\par 

\texttt{midi joplin | smf > joplin.smf}


\par 
The
\textbf{smf}
command provides two options.
The
\textbf{-t}
option allows the user to set the tempo,
whereas the
\textbf{-v}
option allows the user to specify a default MIDI key velocity.
See the
\textit{Humdrum Reference Manual}
for details.



\subsection*{Reprise}

\par 
In this chapter we have learned how Humdrum data can be output
as MIDI data.
Humdrum provides a \texttt{**MIDI} representation that closely
parallels MIDI but remains in conformity with the Humdrum syntax.
This means that the data can still be processed with other
Humdrum tools (as we will see in later chapters).

\par 
The
\textbf{midi}
command can translate
\texttt{**kern}
data to
\texttt{**MIDI}
and the
\textbf{perform}
and
\textbf{smf}
commands can be used to generate true MIDI data
for listening.
The
\textbf{perform}
command provides a simple interactive command-line sequencer
for playing whatever input is provided.
The
\textbf{smf}
command generates standard MIDI files that can be
used to transport MIDI data to a vast array of
commercial and non-commerical applications software.
In
Chapter 30
we will explore some of the Humdrum tools for
inputting MIDI data into Humdrum.

\\
\begin{itemize}
\item 

\textbf{Next Chapter}
\item 

\textbf{Previous Chapter}
\item 

\textbf{Table of Contents}
\item 

\textbf{Detailed Contents}
\\\\

� Copyright 1999 David Huron
\end{document}
