% This file was converted from HTML to LaTeX with
% Tomasz Wegrzanowski's <maniek@beer.com> gnuhtml2latex program
% Version : 0.1
\documentclass{article}
\begin{document}



  
  
    
      
      
      
    
  



\section*{Chapter5}


[\textit{Previous Chapter}]
[\textit{Contents}]
[\textit{Next Chapter}]


\section*{The Humdrum Syntax}



In the previous chapters we have seen several examples of
pre-defined Humdrum representations, such as
\texttt{**kern},
\texttt{**solfa}
and
\texttt{**MIDI}.
These representations exhibit a number of common properties,
including the manner in which the data are organized in spines.
In this chapter, we provide a complete description of
the Humdrum representation syntax.
This chapter will help you better understand how Humdrum
representations are organized, and will provide essential
foundations for designing your own Humdrum representations.

\par 
The Humdrum syntax provides a framework
within which representation schemes can be defined.
Each scheme consists of a mapping between the concepts we wish to
represent (called \textit{signifieds}) and how we wish to represent
them (called \textit{signifiers}).
The signifieds can be any music-related concept determined by the user.
The signifiers consist of the text characters commonly available on computers.

\par 
Humdrum regards each file as a two-dimensional plane,
much like a sheet of paper.
\textit{Successions}
of events proceed vertically down the page, whereas
\textit{concurrent}
events extend horizontally across the page.
Two signifiers that occupy the same horizontal line represent concurrent
(or overlapping) events.
The basic organization of Humdrum files may be schematically illustrated
as follows:

successiveevents

concurrent events ->AJVetc.
BKWetc.
CLXetc.
DMYetc.
etc.etc.etc.


\par 

\subsection*{Types of Records}

\par 
Humdrum encodings consist of a set of one or more lines or
\textit{records.}
There are three types of Humdrum records:
\begin{itemize}
\item 
comment records,
\item 
interpretation records,  and
\item 
data records.
\end{itemize}

\par 
These three record types are mutually exclusive, so it is not possible
to mix comments, interpretations, or data records on the same line.


\subsection*{Comment Records}

\par 
As we noted in
Chapter 2,
there are two kinds of comments:
global comments and local comments.
\textit{Global comments}
pertain to all concurrent parts,
whereas
\textit{local comments}
pertain to some specific column of data
(such as a particular staff, instrument, note, finger, etc.).
Comments are lines that contain an exclamation mark (!) at the beginning
of the record (in the left-most position);
subsequent characters up to and including the occurrence
of a carriage return or newline character constitute the comment record.
Recall that global comments are denoted by two exclamation marks
(!!) at the beginning of the record.
Global comments may contain any sequence of printable characters,
including `blank space' such as tabs and spaces.
Local comments may contain any sequence of printable ASCII characters,
with the important exception of the tab character which is used to separate spines.
Comments may be used to insert free-format commentaries in Humdrum encodings.


\subsection*{Interpretation Records}

\par 
\textit{Interpretations}
are lines that begin with the asterisk character (\texttt{*}).
Interpretations are used to identify more precisely the state of the
representation -$\,$- for example, to indicate that an encoded part
is for a transposing instrument in E-flat, or to indicate that
the representation is for a given Balinese tuning, or
that the representation encodes a conductor's physical gestures.
Humdrum requires that at least one interpretation must be specified before
any data records are encountered.
The difference between a comment and an interpretation is that
interpretations are formal, potentially
\textit{executable}
statements;
interpretations pass information to programs that process
the Humdrum encoding.

\par 
As in the case of comments, there are two types of interpretations:
\textit{exclusive}
interpretations begin with a double asterisk (\texttt{**})
whereas
\textit{tandem}
interpretations begin with a single asterisk (\texttt{*}).
Exclusive interpretations are mutually exclusive:
only one such interpretation can be active at a given time for a given string
of data.
No set of data is complete without the presence of an exclusive interpretation.
Tandem interpretations,
by contrast, provide supplementary information
about how a set of data is to be interpreted.
Several tandem interpretations may pertain to a given set of data;
unlike exclusive interpretations,
tandem interpretations are not necessarily mutually exclusive.


\subsection*{Data Records}

\par 
Lines that do not contain either an exclamation mark or
an asterisk in the first column are
\textit{data records.}
Blank lines (i.e., lines which are either empty, or contain
only blank space, such as tabs and spaces) are forbidden in Humdrum.
Thus data records may be formally defined as non-empty lines that do not begin
with either an exclamation mark or an asterisk.

\par 
In Humdrum, each data record encodes information pertaining either to a
particular moment in time or to a particular time window or duration.
(Whether a record represents a precise moment or 
an expanse of time depends on the accompanying interpretation.)
Each data record may contain one or more data
\textit{tokens.}
When more than one token is present, tokens are separated from
each other by tabs.
When several data records are present, multiple tokens are aligned
in columns through the file.
As we noted earlier, these columns are referred to as
\textit{spines.}
By itself, a spine has no particular meaning;
it is simply a way of linking together related tokens through time.
Spines become meaningful only when they are labelled by adding
an interpretation.

\par 
By itself, Humdrum recognizes only six ASCII characters.
Two of these characters -$\,$- the exclamation mark (!) and the asterisk (*) -$\,$-
have a special meaning
\textit{only}
when they appear in the first column of
a record (or are preceded by a tab; see below).
The remaining special Humdrum characters are the period (.),
the space, the tab character, and the carriage return
(= newline character).
As we have seen, the exclamation mark and the asterisk are used
to identify comments and interpretations, respectively.
The tab and carriage return characters are used to format the
data into
\textit{spines}
and
\textit{records,}
respectively.


\subsection*{Data Tokens and Null Tokens}

\par 
As we noted above, the data in the data records are conceptually
divided into tokens.
In Humdrum, there are two possible types of tokens:
\begin{itemize}
\item 
\textit{data}
tokens, and
\item 
\textit{null}
tokens (.).
\end{itemize}
Consider, for example, the following file:
\texttt{

X.X
XXX
.XX
X.X

}
This file consists of three vertical spines and four horizontal records.
The first and third spines begin with data tokens,
while the second spine begins with a null token.
Without the presence of interpretations, the meaning of this file
is indeterminate.
The file below contains two spines that have been labelled
using Humdrum interpretations:
\texttt{

**left**right
X.
.X
X.
.X
X.
*-*-

}
The user has defined two interpretations: "left" and "right."
The intention is to represent the footfalls of a person's left and right feet.
The representation simply encodes that the left and right feet
have alternating events, such as might be produced by walking or running.
Notice that null tokens (.) indicate nothing at all and merely act as
place-holders to maintain the format of the two spines.
Notice also that interpretations must be defined for each spine,
and that each interpretation consists of some keyword appended to
the double asterisks (e.g. \texttt{**left}).
No intervening spaces are permitted between the interpretation
\textit{keyword}
(\texttt{left}) and the asterisks; however, spaces may appear as part of
the keyword itself.
In addition, when more than one spine is present, both the data tokens
and the associated interpretations must be separated by a tab character;
spaces cannot be used to separate spines.
Finally, note that each spine is formally terminated by a
\textit{spine-path terminator}
-$\,$-  an asterisk followed by a minus sign.

\par 
Interpretations can be cascaded so that a single spine has more than
one interpretation associated with it.
This is done through the addition of tandem interpretations.
Consider the following example:
\texttt{

**foot**foot**arm**arm
*left*right*left*right
X..X
.XX.
X..X
.XX.
X..X
*-*-*-*-

}
In this case the categories "left" and "right" have been
transformed to tandem interpretations.
The first spine is interpreted both as "left" and as "foot."
The exclusive interpretation (double asterisks) takes conceptual
precedence over the tandem interpretation (single asterisk).
That is, tandem interpretations merely modify or supplement the
exclusive interpretation.
Hence, given the above representation, we could say that "left"
is an attribute of "foot" or "arm," but we could not
say that "foot" is an attribute of "left."

\par 
Users are free to define as many different exclusive and tandem
interpretations as they wish.
For example, a user might define the interpretation \texttt{**bowing}
that would be suitable for encoding detailed bowing information
in works for strings.
For each exclusive interpretation, the Humdrum user can re-define the
meaning of all of the text characters, with the
exception of the tab and the carriage return, which always retain their
functions as `token/spine separator' and `record separator' respectively.
The characters   !   .   *   can also be re-defined,
although there are some restrictions as to how they can be used.
Specifically, the exclamation mark cannot occur in the first column of
the record unless it is used to indicate a comment.
Similarly, the asterisk cannot occur in the first column of a record
unless it is used to indicate a Humdrum interpretation.
The period cannot appear in the first column unless it is used to
indicate a null data token.
In addition, the exclamation mark, asterisk, and period cannot appear
following a tab unless they are used to indicate a comment, interpretation,
or null token, respectively.


\subsection*{Data Sub-Tokens}

\par 
Data tokens can be split into sub-tokens via the space character.
In the first data record of the following example,
the first spine contains two sub-tokens whereas the third
spine contains three sub-tokens.
Sub-tokens do not have their own spine organization and
can appear and disappear as necessary:

\texttt{**spine1**spine2**spine3
A BJX Y Z
ABJXYZ
A B C.X Z
*-*-*-}


\par 
Data sub-tokens are useful in a variety of circumstances.
An appropriate use of sub-tokens might be to encode double-
and triple-stops in string parts.

\par 
In the Humdrum data records, the space character is
reserved solely for use as a sub-token delimiter.
Note that consecutive spaces are illegal, and that
data tokens cannot begin or end with a space character.
Of course spaces can be used freely in comments and
in interpretations.


\subsection*{Spine Paths}

\par 
Humdrum representations often consist of a fixed number of spines
that continue throughout the course of an encoded file.
As we have seen in the preceding chapters,
a typical use of spines is to encode
different voices or parts in a musical work.
However, there is no reason to equate spines with voices;
spines are used for many other purposes as well.

\par 
In encoding Humdrum representations it is
occasionally useful to be able to vary the number of spines.
However, files with varying numbers of spines can pose significant questions
of interpretation.
Consider, for example, the following sequence of Humdrum-like data records:
\texttt{

123
123
123
AB
AB
AB

}
At the point where three spines are reduced to two spines
the continuity is ambiguous:
Has spine `3' been discontinued?
Or is spine `B' a continuation of spine `3' with spine `A' a continuation
of spine `1' or `2'?
For some representations such questions will be of little concern;
however, in other circumstances, the manner in which the spines continue will
be of critical importance.
For example, if all of the above spines encoded pitch information
for various musical parts,
a study of melodic intervals would need to resolve the specific
melodic paths as the representation moves from three to two spines.
Failure to clarify the pitch paths would make it difficult
to determine or search for specific successions of melodic intervals.

\par 
The Humdrum syntax provides special
\textit{spine path indicators}
that make it possible to resolve such ambiguities
and to ensure that the continuity (or lack of continuity) is made clear.
Humdrum provides five special path indicators,
one of which we have already encountered:
\begin{itemize}
\item 
a new spine may be introduced
\item 
an existing spine may terminate (without continuing further)
\item 
a previous spine may be split into two spines
\item 
two or more spines may be amalgamated into a single spine
\item 
the positions of two spines may be exchanged
\end{itemize}
Spine path indicators use the following signifiers:
the plus sign (add a spine), the minus sign (terminate a spine),
the caret (split a spine), the lower-case letter `v' (join spines),
and the lower-case letter `x' (exchange spines).
In addition to these, a
\textit{null interpretation}
exists whose purpose is merely to act as a place-holder
in interpretation records:

\par 

\texttt{*+}add a new spine (to the right of the current spine)
\texttt{*-}terminate a current spine
\texttt{*\^{}}split a spine (into two)
\texttt{*v}join (two or more) spines into one
\texttt{*x}exchange the position of two spines

\texttt{*}null interpretation (place holder)

\textit{Spine Path Interpretations}
Spine paths are types of interpretations, so the
spine path indicators are encoded as Humdrum interpretations,
using the asterisk signifier (*).
The following examples illustrate a few possible path changes:

\par 
\texttt{

123
**-*}(elimination of spine \#2)\texttt{
13

123
**x*x}(exchange spines \#2 and \#3)\texttt{
132

123
**\^{}*}(splitting of spine \#2)\texttt{
12a2b3

123
**v*v}(amalgamation of spines \#2 and \#3)\texttt{
12\&3

}
Notice that in cases where two or more spines are amalgamated,
the spines must be adjacent neighbors.
For example, the arrangement below is forbidden by the Humdrum syntax
since it is not clear whether spines \#1 and \#3 amalgamate into
spine `A' or spine `B'.
\texttt{

123
*v**v}(syntactically illegal)\texttt{
AB

}
In such cases, amalgamating the two outer spines can be accomplished by first
using the exchange path signifier.
Here we exchange spines \#2 and \#3 before amalgamating the original
first and third spines:
\texttt{

123
**x*x
*v*v*
1\&32

}
In cases where the user wishes to amalgamate several spines, a number
of interpretation records may be necessary.
In the following example, spines \#1 and \#2 are first joined together
(momentarily defining three spines: 1\&2, 3, 4).
In the subsequent interpretation record, spine \#2 (previous spine \#3) and
spine \#3 (previous spine \#4) are then joined:
\texttt{

1234
*v*v**
**v*v
1\&23\&4

}
In addition, it is possible to join more than two spines
at the same time:
\texttt{

1234
*v*v*v*v
1\&2\&3\&4

}
In cases where a new spine is introduced, it is essential to indicate
the exclusive interpretation that applies to the new data.
Thus an `add spine' indication must be followed by a second
interpretation record:

\texttt{123
**+*}(add a new spine.)\texttt{
****inter*}(define exclusive interpretation for the new spine)\texttt{
12new3

Failing to follow the introduction of a new spine by a subsequent
exclusive interpretation is illegal.

\par 
The following examples illustrate a variety of more complex
path redefinitions:
\texttt{

1234
*v*v*\^{}*\^{}
1\&23a3b4a4b

}

\par 
\texttt{

12345
**-**-*
*v*v*v
1\&3\&5

}

\par 
\texttt{

12345
**-**\^{}*+
*******new
*v*v****
1\&34a4b5new

}

\par 
\texttt{

1234
*x*x**
**x*x*
***x*x
2341

}
Note that with judicious planning, the user can completely reconfigure
all spines within a Humdrum file.

\par 
Syntactically, some path constructions are illegal;
here are some examples of illegal constructions:

\texttt{123}
\texttt{*v**v}(The join-spine indication in spine \#1 does not adjoin
 spine \#3.)
\texttt{123}
\texttt{*x*x*x}(No more than two exchange interpretations at a time.)

\texttt{123}
\texttt{*x**}(Must have two exchange interpretations together.)

\texttt{123}
\texttt{*v**}(Must have two or more join interpretations at a time.)

\texttt{123}
\texttt{**}(Spine eliminated without using a termination interpretation.)
\texttt{12}

\texttt{123}
\texttt{***+}(Adding a new spine should result in 4 interpretations.)
\texttt{123}

\texttt{12}
\texttt{***-}(Cannot eliminate non-existent spine.)

\texttt{12}
\texttt{*+*}
\texttt{1new2}(New spine started without specifying new interpretation.)

\texttt{12}
\texttt{**+}
\texttt{***inter*}(Interpretation labels the wrong spine.)
\texttt{ABC}


\par 

\subsection*{The Humdrum Syntax:  A Formal Definition}


\par 
With the preceding background it is now possible to define formally
a Humdrum representation.
First we can define a Humdrum file.
A Humdrum file must conform to one of the following:
\begin{itemize}
\item 
A file containing
\textit{comments, data records}
and
\textit{interpretations}
with the restriction that no data record or local comment appears before
the first
\textit{exclusive interpretation.}
\item 
A file containing
\textit{data records}
preceded by at least one
\textit{exclusive interpretation.}
\item 
A file containing only
\textit{comments}
and
\textit{interpretations}
with the restriction that no local comments appear before the first
interpretation.
\item 
A file containing only
\textit{interpretations}
beginning with an exclusive interpretation.
\item 
A file containing only global
\textit{comments.}
\item 
A totally empty file (i.e. a file containing no records).
\end{itemize}
In addition, each spine in a Humdrum file must ultimately end with
a path terminator (*-).
Only global comments (or new exclusive interpretations) may occur
following the termination of all spines.
A property of Humdrum files is that the concatenation of two or
more Humdrum files will always result in a Humdrum file.

\par 
Additional interpretations may be added throughout the file.
Global comments may appear anywhere in the file.
However, local comments are much more restricted:
(1) Local comments may not appear until after the first
interpretation record, and
(2) The number of sub-comments in a local comment record must
be equivalent to the number of currently active spines.
}

CommentEither a global or local comment.  Any record beginning
with an exclamation mark.
Global commentAny record beginning with two exclamation marks (!!).
Local commentAny record beginning with one and only one exclamation mark (!).
Every spine in that record must also begin with an exclamation
mark.
Null commentA comment record containing no commentary; only the
appropriate exclamation mark(s) are present.
InterpretationEither an exclusive or tandem interpretation.  Any record
beginning with an asterisk (*).
Exclusive interpretationAny record beginning with one or more asterisks (*), where at
least one spine begins with two asterisks.
Tandem interpretationAny record beginning with a single asterisk (*) where none of the
spines begin with two asterisks.
Path indicatorOne of five special tandem interpretations *+  *-  *v  *\^{}  *x found only
in tandem interpretation records.
Null interpretationAn interpretation for a given spine or spines consisting of just the
interpretation signifier (i.e., a single asterisk).
Data recordAny record that is not a comment or interpretation.  Must contain
the same number of tokens as the number of current spines.
Null tokenThe period (.) either alone on a single record or separated from
other characters by a tab.  Appears only in data records.
Null data recordA data record consisting only of null tokens.
SpineA column-like "path" of information -$\,$- including data records,
local comments, and interpretations.

\textit{Humdrum Terminology}
As a supplement to the above "positive" definition of the
Humdrum syntax, we can also describe various inputs that do
\textit{not}
conform to the Humdrum syntax:

An empty record.
A record containing only tabs.
A record beginning with a tab.
A record ending with a tab.
Any record containing two successive tab characters.
Any data record having fewer or more spines than the immediately
   preceding data record.
A record having only one join-spine indication.
A record having only one exchange-spine indication.
A record having more than two exchange-spine indications.

\textit{Some Illegal Humdrum Constructions}


\subsection*{The \textit{humdrum} Command}


\par 
One of the most important commands in the Humdrum Toolkit is the
\textbf{humdrum}
command itself.
This command is used to identify whether a file or other input stream
conforms to the above Humdrum syntax.
Where appropriate, the
\textbf{humdrum}
command issues error messages identifying the type and location
of any syntactic transgressions.
If no infractions are found,
\textbf{humdrum}
produces no output (i.e., in UNIX parlance "silence is golden").
All of the commands in the Humdrum toolkit
assume that the inputs given to them conform to the Humdrum syntax.
Whenever you encounter a problem, you should always test the input to
assure that it is in the proper Humdrum format.

\par 
The examples given below provide further illustrations
of Humdrum representations:

\texttt{**form
Introduction
Exposition
Development
Recapitulation
Coda
*-}


\texttt{**American**British
quartercrotchet
eighthquaver
dotted halfdotted minim
*-*-}


\texttt{**Opus/No**Year
23/11821
23/21821
23/31822?
241822
*-*-}


\texttt{**recip**diaton**accidental**stem-dir**kern
4c\#/4c\#/
8d./8d/
8e./8e/
2f\#/8f\#/
*-*-*-*-*-}


\texttt{**heart-rate
74
73
74
77
78}
*-


\texttt{**foreground
flute
*\^{}
fluteviolin1
*-*
violin1
*\^{}
violin1bassoon
**\^{}
violin1bassoon'cello
***\^{}
violin1bassoon'cellotrombone
*-*-*-*
trombone
*\^{}
trombonetrumpet
*-*-}

}


\subsection*{Reprise}

\par 
This chapter has identified the formal
structural and organizational features of the Humdrum syntax.
The syntax provides a framework within which sequential
symbolic data can be represented.
Individual representation schemes map the ASCII character set (signifiers)
to various music-related concepts (signifieds).

\par 
Each representation is designated by an exclusive interpretation.
The corresponding data are organized in spines that 
may meander throughout the file.
New spines may be added, spines joined together, exchanged,
split, or terminated.
Data are organized as tokens, although tokens
can consist of multiple subtokens separated by single spaces.
Null tokens can appear as place-holders where no specific data exists.

\par 
Free-form comments may be interspersed throughout the file.
Global comments pertain to all spines whereas
local comments pertain to individual spines.
Additional interpretive information may be encoded using tandem
interpretations.
Both local comments and tandem interpretations may occur anywhere,
but must be preceded in the spine by some exclusive interpretation.

\\
\begin{itemize}
\item 

\textbf{Next Chapter}
\item 

\textbf{Previous Chapter}
\item 

\textbf{Table of Contents}
\item 

\textbf{Detailed Contents}
\\\\

� Copyright 1999 David Huron
\end{document}
