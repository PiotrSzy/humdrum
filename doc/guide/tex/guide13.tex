% This file was converted from HTML to LaTeX with
% Tomasz Wegrzanowski's <maniek@beer.com> gnuhtml2latex program
% Version : 0.1
\documentclass{article}
\begin{document}



  
  
    
      
      
      
    
  



\\
\\

\section*{Chapter13}


[\textit{Previous Chapter}]
[\textit{Contents}]
[\textit{Next Chapter}]


\section*{Assembling Scores}



In the previous chapter we learned how to extract parts
and passages from Humdrum files.
In this chapter we discuss the reverse procedures:
how to assemble and coordinate larger
documents from individual segments and parts.
We will discuss four tools: the UNIX
\textbf{cat}
command, and the Humdrum
\textbf{assemble},
\textbf{timebase}
and
\textbf{rid}
commands.


\subsection*{The \textit{cat} Command}


\par 
The UNIX
\textbf{cat}
command allows two or more inputs to be concatenated together.
If we concatenate two files, the output will consist of the
contents of the first file, followed immediately by the contents
of the second file.
For example, the following command will concatenate together the
files \texttt{mov.1}, \texttt{mov.2} and \texttt{mov.3}
and place the result in the file \texttt{complete}.
The order of concatenation is the same as the order of file names
given on the command line:

\par 

\texttt{cat mov.1 mov.2 mov.3 > complete}


\par 
If each of the concatenated files conforms to the Humdrum syntax,
then the resulting combined file is guaranteed to conform
to the Humdrum syntax.
However, in many cases, there may be redundant information
present in the concatenated output.
Suppose we had three files, each of which encoded a single measure of music.
Concatenating them together might result in an output such as the following:

\texttt{!! File 1
\texttt{**kern**kern**kern**kern
\texttt{*M4/4*M4/4*M4/4*M4/4
\texttt{*G:*G:*G:*G:
\texttt{*k[f\#]*k[f\#]*k[f\#]*k[f\#]
\texttt{4GG4B4d4g
\texttt{=1=1=1=1
\texttt{4GG4d4g4b
\texttt{4FF\#4d[4a4dd
\texttt{8GG4d8a]4b
\texttt{8BB.[8g.
\texttt{4D4d8g]4a
\texttt{..8f\#.
\texttt{*-*-*-*-
\texttt{!! File 2
\texttt{**kern**kern**kern**kern
\texttt{*M4/4*M4/4*M4/4*M4/4
\texttt{*k[f\#]*k[f\#]*k[f\#]*k[f\#]
\texttt{=2=2=2=2
\texttt{8G4d4g4b
\texttt{8F\#...
\texttt{4E4e4g4cc\#
\texttt{4D;4A;4f\#;4dd;
\texttt{8A4cn4a4ee
\texttt{8G...
\texttt{*-*-*-*-
\texttt{!! File 3
\texttt{**kern**kern**kern**kern
\texttt{*M4/4*M4/4*M4/4*M4/4
\texttt{*k[f\#]*k[f\#]*k[f\#]*k[f\#]
\texttt{*k[f\#]*k[f\#]*k[f\#]*k[f\#]
\texttt{=3=3=3=3
\texttt{4F\#4d[4a8dd
\texttt{...8cc
\texttt{4G4d8a]4b
\texttt{..[8g.
\texttt{8C\#8e8g]4a
\texttt{8D\#4B8f\#.
\texttt{8E.8e8g
\texttt{8F\#8A8d8a
\texttt{*-*-*-*-}


\par 
Notice that each complete measure ends with spine-path terminators
and that the
\texttt{**kern}
exclusive interpretations are repeated.
This organization has a number of repercussions for various
Humdrum tools.
For example, the
\textbf{mint}
command calculates melodic intervals between successive notes
within a spine.
However,
\textbf{mint}
will not calculate intervals between pitches that are separated by
a spine-path terminator.
In other words, in the above output,
\textbf{mint}
will fail to calculate the melodic intervals between notes
in successive measures.


\subsection*{The \textit{rid} Command}

\par 
This problem can be resolved by using the Humdrum
\textbf{rid}
command.
The
\textbf{rid}
command can be used to eliminate various kinds of records.
Each option for
\textbf{rid}
eliminates a different class of records.
Here are the record classes with their associated options:

-Geliminate all global comments
-geliminate only global comments that are empty
-Leliminate all local comments
-leliminate only local comments that are empty
-Ieliminate all interpretations
-ieliminate only null interpretations
-Deliminate all data records
-deliminate only null data records
-Teliminate all tandem interpretations
-teliminate duplicate (repeated) tandem interpretations
-Ueliminate unnecessary exclusive interpretations (see below)
-usame as -U


\par 
Null records are devoid of content.
For example,
null interpretations consist of a single asterisk in each spine;
null data record consists of just null data tokens (.) in each spine;
null local comments consist of a single exclamation mark in each spine.
Null global comments contain just two exclamation marks at
the beginning of a record.

\par 
Upper- and lower-case options are used to distinguish
\textit{all}
records
of a certain class (upper-case) from \textit{empty}, \textit{null} or
\textit{repeated}
records of a certain class (lower-case).

\par 
By way of example, the following command will eliminate all global
comments (including reference records) from the input:

\par 

\texttt{rid -G Saint-Saens}


\par 
Similarly, the following command will eliminate all tandem interpretations
from an input:

\par 

\texttt{rid -T Vaughan-Williams}


\par 
Options can be combined.
The following command eliminates all global and local comments,
interpretations, and null data records:

\par 

\texttt{rid -GLId}


\par 
The option combination \textbf{-GLId} is very common with
\textbf{rid}
since only non-null data records are retained in the output.

\par 
With the
\textbf{-u}
option,
\textbf{rid}
will remove "unnecessary" exclusive interpretations.
Exclusive interpretations are deemed unnecessary if they
don't change the current status of the data.
In the following example, the second \texttt{**psaltery}
interpretation is redundant.
The \textbf{rid -u} command would remove the first spine-path terminator
and the second exclusive interpretation -$\,$- leaving a continuous
data spine.

\texttt{**psaltery
\texttt{.}
\texttt{.}
\texttt{.}
\texttt{*-}
\texttt{**psaltery
\texttt{.}
\texttt{*-}


\par 
In addition,
\textbf{rid}
provides a
\textbf{-t}
option which removes "duplicate" or repeated
tandem interpretations.
In the above example there is no need to repeat the meter signature
and key signature in each measure.

The following command will concatenate each of the three measures
together, and then eliminate the unwanted interpretations:

\par 

\texttt{cat bar1 bar2 bar3 | rid -ut}


\par 
The resulting output is given below:

\texttt{!! File 1
\texttt{**kern**kern**kern**kern
\texttt{*M4/4*M4/4*M4/4*M4/4
\texttt{*k[f\#]*k[f\#]*k[f\#]*k[f\#]
\texttt{4GG4B4d4g
\texttt{=1=1=1=1
\texttt{4GG4d4g4b
\texttt{4FF\#4d[4a4dd
\texttt{8GG4d8a]4b
\texttt{8BB.[8g.
\texttt{4D4d8g]4a
\texttt{..8f\#.
\texttt{!! File 2
\texttt{=2=2=2=2
\texttt{8G4d4g4b
\texttt{8F\#...
\texttt{4E4e4g4cc\#
\texttt{4D;4A;4f\#;4dd;
\texttt{8A4cn4a4ee
\texttt{8G...
\texttt{!! File 3
\texttt{=3=3=3=3
\texttt{4F\#4d[4a8dd
\texttt{...8cc
\texttt{4G4d8a]4b
\texttt{..[8g.
\texttt{8C\#8e8g]4a
\texttt{8D\#4B8f\#.
\texttt{8E.8e8g
\texttt{8F\#8A8d8a
\texttt{*-*-*-*-}


\par 
Of course care should be exercised when
concatenating inputs together.
Although an output may conform to the Humdrum syntax,
the result can nevertheless violate conventions for a specific
representation such as \texttt{**kern}.
For example, if we were to concatenate measure 85 to measure 87,
it is possible that tied-notes won't match up, or that phrases will
begin without ending, etc.
These anomalies may cause problems with subsequent processing.


\subsection*{Assembling Parts Using the \textit{assemble} Command}

\par 
Assembling parts into a full score is slightly more challenging
than concatenating together musical segments.
The principle tool for joining spines together is the
\textbf{assemble}
command.
Consider the following two files:

\texttt{!! Assemble example
\texttt{!! File 1
\texttt{**Letters
\texttt{! A to E
\texttt{A
\texttt{B
\texttt{C
\texttt{D
\texttt{E
\texttt{*-}


\texttt{!! Assemble example
\texttt{!! File 2
\texttt{**Numbers
\texttt{! 1 to 5
\texttt{1
\texttt{2
\texttt{3
\texttt{4
\texttt{5
\texttt{*-}


\par 
These two files can be aligned side by side using
\textbf{assemble}:

\par 

\texttt{assemble letters numbers}


\par 
The resulting output is:

\texttt{!! Assemble example
\texttt{!! File 1
\texttt{!! File 2
\texttt{**Letters**Numbers
\texttt{! A to E! 1 to 5
\texttt{A1
\texttt{B2
\texttt{C3
\texttt{D4
\texttt{E5
\texttt{*-*-}


\par 
Note the following:
(1) The spines are joined side by side from left to right in the
same order as specified on the command line.
(2) Local comments are preserved in their appropriate spines.
(3) When identical global comments occur at the same
location in both files, only a single instance of the comment is output.
(4) Dissimilar global comments are output successively.

\par 
The files joined by
\textbf{assemble}
are not confined to a single spine.
For example, one input file may contain 2 spines and a second file
may contain 20 spines.
The resulting output will contain 22 spines.
There is no limit to the number of files that can be
assembled at one time.

\par 
In many cases, the input files will have dissimilar lengths.
The
\textbf{assemble}
command will correctly terminate the appropriate spines.
For example, in the above case, if the \texttt{numbers} file
contained only the numbers 1 to 3, the assembled output would
appear as follows:

\texttt{!! Assemble example
\texttt{!! File 1
\texttt{!! File 2
\texttt{**Letters**Numbers
\texttt{! A to E! 1 to 3
\texttt{A1
\texttt{B2
\texttt{C3
\texttt{**-
\texttt{D
\texttt{E
\texttt{*-}


\par 
If the order of the input files was reversed,
\textbf{assemble}
would produce an output with the appropriate spine-path changes:

\texttt{!! Assemble example
\texttt{!! File 2
\texttt{!! File 1
\texttt{**Numbers**Letters
\texttt{! 1 to 3! A to E
\texttt{1A
\texttt{2B
\texttt{3C
\texttt{*-*
\texttt{D
\texttt{E
\texttt{*-}


\par 
Note that if all of the input files conform to the Humdrum syntax,
then
\textbf{assemble}
guarantees that the assembled output will also conform to the Humdrum syntax.


\subsection*{Aligning Durations Using the \textit{timebase} Command}

\par 
Suppose now that we wanted to join two hypothetical files containing
\texttt{**kern}
data.
The first file contains two quarter notes,
whereas the second file contains four eighth notes:

\texttt{!! File 1
\texttt{**kern
\texttt{4c
\texttt{4d
\texttt{*-}


\texttt{!! File 2
\texttt{**kern
\texttt{8e
\texttt{8g
\texttt{8f
\texttt{8g
\texttt{*-}


\par 
Using
\textbf{assemble}
to paste them together will clearly lead to an uncoordinated result.
The two quarter notes in file 1 will be incorrectly matched with the
first two eighth notes in file 2.

\par 
The Humdrum
\textbf{timebase}
command can be used to reformat either
\texttt{**kern}
or
\texttt{**recip}
inputs so that each output data record represents an
equivalent slice (elapsed duration) of time.
(Barlines are ignored by \textbf{timebase}.)
The
\textbf{timebase}
command achieves this by padding an input with null data records.
In the above case, we would preprocess file 1 as follows:

\par 

\texttt{timebase -t 8 file1 > file1.tb}


\par 
The new file would look like this:

\texttt{!! File 1
\texttt{**kern
\texttt{4c
\texttt{.
\texttt{4d
\texttt{.
\texttt{*-}


\par 
The
\textbf{-t}
option is used to indicate the "time base" -$\,$- in this
case, an eighth duration.
Since all non-barline data records in both files represent elapsed
durations of an eighth-note, we can continue by using the
\textbf{assemble}
command as before.
The command:

\par 

\texttt{assemble file1.tb file2}


\par 
will result in the following two-part score:

\texttt{!! File 1
\texttt{!! File 2
\texttt{**kern**kern
\texttt{4c8e
\texttt{.8g
\texttt{4d8f
\texttt{.8g
\texttt{*-*-}

Suppose that \texttt{file2} also contained a quarter-note.
For example, consider a revised \texttt{file2}:

\texttt{!! File 2
\texttt{**kern
\texttt{8e
\texttt{8g
\texttt{4f
\texttt{*-}


\par 
Before assembling the two parts, we would need to apply the
\textbf{timebase}
command to this file (using the same 8th-note time-base value).
Assembling the two "time-based" files would produce the
following result:

\texttt{!! File 1
\texttt{!! File 2
\texttt{**kern**kern
\texttt{4c8e
\texttt{.8g
\texttt{4d4f
\texttt{..
\texttt{*-*-}


\par 
Notice that we have a spurious null data record in the last line;
both parts encode null tokens.
For most processing, the presence of null data records is inconsequential.
However, if we wish, these null data records can be eliminated using the
\textbf{rid}
command with the
\textbf{-d}
option.
In fact it is common to follow an
\textbf{assemble}
command with
\textbf{rid -d}
to strip away unnecessary null data records.
The command:

\par 

\texttt{assemble file1.tb file2.tb | rid -d}


\par 
would result in the following output:

\texttt{!! File 1
\texttt{!! File 2
\texttt{**kern**kern
\texttt{4c8e
\texttt{.8g
\texttt{4d4f
\texttt{*-*-}


\par 
The
\textbf{timebase}
command can be applied to multi-spine inputs as well as
single-spine inputs.
Consider, the following input:
\\\\

\texttt{**kern**kern**kern**kern**commentary
\texttt{4g8.r8.cc16ee2nd inversion
\texttt{...8ff.
\texttt{.32b16cc16ggclash
\texttt{.32a...
\texttt{8f8cc8dd8ffsuspension
\texttt{*-*-*-*-*-}

The following command will cause the addition of null data records
so that each data record represents an elapsed time of a 32nd duration.
Incidentally, notice that any spine contain non-rhythmic data
-$\,$-  such as the \texttt{**commentary} spine in the above example -$\,$-
is also transformed so that synchronous data is maintained.

\par 

\texttt{timebase -t 32 Corelli}


\par 
The corresponding output is as follows.
\\\\

\texttt{**kern**kern**kern**kern**commentary
\texttt{*tb32*tb32*tb32*tb32*tb32
\texttt{4g8.r8.cc16ee2nd inversion
\texttt{.....
\texttt{...8ff.
\texttt{.....
\texttt{.....
\texttt{.....
\texttt{.32b16cc16ggclash
\texttt{.32a...
\texttt{8f8cc8dd8ffsuspension
\texttt{.....
\texttt{.....
\texttt{.....
\texttt{*-*-*-*-*-}

Notice that
\textbf{timebase}
has added a tandem interpretation (\texttt{*tb32}).
This indicates that the output has been processed
so that each non-barline data record represents
an elapsed duration equivalent to a thirty-second note.


\subsection*{Assembling N-tuplets}

\par 
Typically, one can simply use the shortest duration present
as a guide for a suitable time-base value.
The shortest duration can be determined using the
\textbf{census} -k
command described in
Chapter 4.
However, tuplets require a little more sophistication.
Suppose we wanted to assemble two parts, one containing
just eighth-notes and the other containing just quarter-note triplets.
(The quarter-note triplets will be encoded as three notes in the time
of a half-note, or "6th" notes.)
We need to create an output whose rhythmic structure will appear
as follows:
\\\\

\texttt{**kern**kern
\texttt{*M2/4*M2/4
\texttt{=1=1
\texttt{68
\texttt{.8
\texttt{6.
\texttt{.8
\texttt{6.
\texttt{.8
\texttt{=2=2
\texttt{68
\texttt{.8
\texttt{6.
\texttt{.8
\texttt{6.
\texttt{.8
\texttt{=3=3
\texttt{*-*-}

In this case, choosing a time-base according to the shortest
duration (8th) will not work since a 6th note is not an
integral multiple of the eighth duration.
We need to find a
\textit{common duration factor}
for both values.
The shortest common duration is a 24th note (there are three 24th notes
in an 8th note, and four 24th notes in a 6th note).
Applying the time-base value `24' to both files will allow us to
coordinate them properly.
Remember that
\textbf{rid -d}
can be used to eliminate unnecessary null data records once we
have finished assembling the spines.
In the worst case, you can determine a common duration factor by
simply multiplying together the shortest notes in the files
to be assembled.
For example, 6 x 8 = 48;
so a time-base of 48 will be guaranteed to work for both files.


\subsection*{Checking an Assembled Score Using \textit{proof}}

\par 
In assembling any score from a set of parts, there is always
the danger of using the wrong time-base value.
When parts are miscoordinated, it is typically the consequence
of one or more notes being discarded by \textbf{timebase}.
Fortunately, such miscoordinations are easily detected by applying the
\textbf{proof}
command to any assembled \texttt{**kern} output.
The
\textbf{proof}
utility checks \texttt{**kern} representations for a wide
variety of possible encoding errors or ambiguities:

\par 

\texttt{proof fullscore}


\par 
By way of summary, creating a full score from a set of
\texttt{**kern}
parts involves the following five tasks:
(1) Identify a common duration factor for all the parts.
Use
\textbf{census}
to determine the shortest duration;
if any of the parts contains an N-tuplet, then the common duration
factor may be smaller than the shortest note.
(2) Use the
\textbf{timebase}
command to expand each input file separately using the common duration
factor.
(3) Assemble the parts using
\textbf{assemble}.
(4) If desired, eliminate unnecessary null data records usingi
\textbf{rid -d}.
(5) Check the assembled score for rhythmic coherence using the
\textbf{proof}
command.


\subsection*{Other Uses for the \textit{timebase} Command}

\par 
The most common use of
\textbf{timebase}
is as a way of expanding a file by padding it with null data records.
However,
\textbf{timebase}
can also be used to contract a file, giving us only those
sonorities that lie a fixed duration apart.
For example, specifying a time-base of
\textbf{-t 2}
will cause only those sonorities that are separated by a half-note
duration to be output.
This sort of rhythmic reduction can be useful in certain circumstances.
For example, suppose you suspect there may be a hemiola
tendency in a given work by Brahms, where the duration separating
hemiola notes is a dotted-quarter.
The command:

\par 

\texttt{timebase -t 4. brahms}


\par 
can be used to extract only those sonorities that are separated by
a dotted-quarter duration.

\par 
Similarly, suppose we want to extract all sonorities falling
on the third beat of a waltz written in 3/2 meter.
First we would edit the input file so it begins
on the third beat of some measure.
Then we could use the following command:

\par 

\texttt{grep -v \^{}= waltz | timebase -t 1. > 3rd\_beat}


\par 
Note that the use of
\textbf{grep}
here is essential in order to eliminate barlines.
The
\textbf{timebase}
command resets itself with each barline, so time-base durations
are calculated from the beginning of the bar.
When barlines are eliminated,
\textbf{timebase}
cannot synchronize to the beginning of each bar and so simply floats
along at the fixed time-base.


\subsection*{Additional Uses of \textit{assemble} and \textit{timebase}}

\par 
Although we normally assemble parts together,
sometimes it is useful to assemble entire scores together.
Suppose we wanted to listen to a theme at the same time as one
of its variations.
We might first use
\textbf{yank}
to extract the appropriate sections.
At the same time we might determine a common duration factor
and expand them using \textbf{timebase}.

\par 

\texttt{yank -s Theme -r 1 blacksmith | timebase -t 32 > temp1}
\\
\texttt{yank -s 'Variation 1' -r 1 blacksmith | timebase -t 32 > temp2}


\par 
Then we assemble the two sections together, translate to
the
\texttt{**MIDI}
representation and use
\textbf{perform}
to listen to both sections at the same time:

\par 

\texttt{assemble temp1 temp2 | midi | perform}


\par 
Similarly, suppose we would like to compare the bass lines
for each variation in some set.
We might extract each of the bass lines, assemble them into a single
score, and then use the
\textbf{ms}
and
\textbf{ghostview}
commands to allow us to see all of the bass lines for all of the
variations concurrently.

\par 

\texttt{yank -s 'Variation 1' -r 1 goldberg | timebase -t 16 > temp1}
\\
\texttt{yank -s 'Variation 2' -r 1 goldberg | timebase -t 16 > temp2}
\\
     etc.   ...
\\
\texttt{assemble temp1 temp2 temp3 ... | rid -d | ms > basslines.ps}
\\
\texttt{ghostview basslines.ps}


\par 
The most common use of
\textbf{assemble}
is not to assemble parts, but to assemble different types of
concurrent information.
Suppose we would like to determine whether descending minor seconds
are more likely to be \textit{fa-mi} rather than \textit{do-ti}.
We can use the
\textbf{mint}
command to characterize melodic intervals, and the
\textbf{solfa}
command to characterize scale degrees.
Assume that our input is monophonic:

\par 

\texttt{mint melodies > temp1}
\\
\texttt{solfa melodies > temp2}


\par 
The files \texttt{temp1} and \texttt{temp2} will have the same
length, so we can assemble them together.
This will generate an output consisting of two spines,
\texttt{**mint}
and
\texttt{**solfa}.
In effect, the \texttt{**mint} spine data will tell us the interval
used to approach the scale degree encoded in the \texttt{**solfa} spine.
We can use
\textbf{grep}
to search for the appropriate combinations of interval and scale degree
and count the number of occurrences:

\par 

\texttt{assemble temp1 temp2 | grep -c '-m2.*mi'}
\\
\texttt{assemble temp1 temp2 | grep -c '-m2.*ti'}



\par 
This same approach can be used to address (innumerable) questions
pertaining to concurrent patterns.
For example, suppose we have a
\texttt{**harm}
spine that identifies
the `Roman numeral' functional harmony for some choral work.
We can identify complex situations such as the following:
for the soprano voice, count how many subdominant pitches are approached
by an interval of a rising third or a rising sixth and coincide with a
dominant seventh chord.
First, let's extract the soprano line and create a corresponding
scale degree representation using \textbf{deg}.
We can use the
\textbf{-a}
option to avoid outputting the melodic direction
signifiers (\texttt{\^{}} and \texttt{v}):

\par 

\texttt{extract -i '*Isopran' howells | deg -a > temp1}


\par 
Next, let's again extract the soprano voice and create a
corresponding melodic interval representation using \textbf{mint}.
Since we are not interested in interval qualities we can invoke the
\textbf{-d}
option to output only diatonic interval sizes.

\par 

\texttt{extract -i '*Isopran' howells | mint -d > temp2}
\\
\texttt{extract -i '**harm' howells > temp3}


\par 
We have also extracted the \texttt{**harm} spine and placed
it in the file \texttt{temp3}.
If we assemble together our three temporary files, the result will
have three spines: \texttt{**deg}, \texttt{**mint} and \texttt{**harm}.
We can now use
\textbf{grep}
to search and count all instances of subdominant pitches that are
approached by ascending thirds/sixths and that coincide with dominant
seventh chords (in the \texttt{**kern} representation: `\texttt{V7}'):

\par 

\texttt{assemble temp1 temp2 temp3 | grep -c '\^{}4}\textit{}\texttt{+[36]}\textit{}\texttt{V7}



\par 
The
\textbf{timebase}
command can also be used for tasks other than assembling parts
together.
Suppose we would like to determine whether secondary dominant
chords are more likely to appear on the third beat than other
beats in a triple meter work.
The
\textbf{timebase}
command can be used to reformat a score so that each measure
occupies the same number of data records.
For example, in a 3/4 meter, an eighth-note time-base will mean
that each measure will contain six data records, and the fifth
data record will correspond to the onset of the third beat.
Recall from
Chapter 12
that the
\textbf{yank -m}
command allows us to extract particular data records following
a specified marker.
In the following command, we have defined the marker as a barline
(\texttt{-m \^{}=}) and instructed
\textbf{yank}
to fetch the fifth line following each occurrence of the marker (\texttt{-r 5}).
In our example, the
\textbf{grep}
command is being used to count V/V chords occurring on third beats:

\par 

\texttt{timebase -t 8 strauss | solfa | yank -m = -r 5 | grep re $\backslash$

| grep fe | grep -c la}



\par 
We can repeat this command for beats one and two by changing the
\textbf{-r}
parameter to 1 and 3 respectively.



\subsection*{Reprise}

\par 
In this chapter we have learned how to concatenate musical passages
together using the
\textbf{cat}
command.
We also learned how to eliminate redundant exclusive and
tandem interpretations from concatenated outputs using the
\textbf{-u}
and
\textbf{-t}
options for
\textbf{rid}.
In addition, we learned how to assemble two or more spines into
a single output file using
\textbf{assemble}.
In the case of
\texttt{**kern}
and
\texttt{**recip}
representations, we learned how to use the
\textbf{timebase}
command to preprocess each constituent file so that all data records
represent equivalent elapsed durations.
Having assembled a full score from parts,
\textbf{rid}
\textbf{-d}
can be used to eliminate any residual or unnecessary null data records.
The
\textbf{proof}
command can be used to ensure that any assembled \texttt{**kern}
data is correctly aligned.

\par 
Finally, we learned that the
\textbf{timebase}
command can be used for other analytic purposes.
Specifically, it can be used to reduce a score rhythmically
so only particular onset points or beats are retained.
In
Chapter 23
we will see additional uses of
\textbf{timebase}
for a variety of types of rhythmic tasks.

\\
\begin{itemize}
\item 

\textbf{Next Chapter}
\item 

\textbf{Previous Chapter}
\item 

\textbf{Table of Contents}
\item 

\textbf{Detailed Contents}
\\\\

� Copyright 1999 David Huron
\end{document}
