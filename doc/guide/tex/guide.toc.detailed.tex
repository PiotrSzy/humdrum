% This file was converted from HTML to LaTeX with
% Tomasz Wegrzanowski's <maniek@beer.com> gnuhtml2latex program
% Version : 0.1
\documentclass{article}
\begin{document}


  
  
    
      
      
      
    
  


\subsection*{Detailed Table of Contents}

\\


\textbf{Regular Table of Contents}



\begin{itemize}
\item 
\textbf{Preface}
\item 
\textbf{Acknowledgments}
\item 
\textbf{How to Use This Book}
\end{itemize}

\textbf{Chapter\textbf{


\begin{enumerate}
\item 

\textbf{Humdrum:  A Brief Tour}
	
	What Can Humdrum Do?
	\\The Humdrum Syntax and the Humdrum Toolkit
	\\Humdrum Syntax
	\\Humdrum Tools
	\\Some Sample Commands
	\\Reprise
	
\item 

\textbf{Representing Music Using **kern (Part I)}
	
	Comment Records
	\\Reference Records
	\\Reprise
	
\item 

\textbf{Some Initial Processing}
	
	The census Command
	\\Simple Searches Using the \textit{grep} Command
	\\Pattern Locations Using \textit{grep -n}
	\\Counting Pattern Occurrences Using \textit{grep -c}
	\\Searching for Reference Information
	\\The \textit{sort} Command
	\\The \textit{uniq} Command
	\\Options for the \textit{uniq} Command
	\\Reprise
	
\item 

\textbf{Basic Pitch Translations}
	
	ISO Pitch Representation
	\\German Ton�he
	\\French Solf�ge
	\\Frequency
	\\Cents
	\\Semitones
	\\MIDI
	\\Scale Degree
	\\Pitch Translations
	\\Transposition Using the \textit{trans} Command
	\\Key Interpretations
	\\Pitch Processing
	\\Uses for Pitch Translations
	\\Reprise
	
\item 

\textbf{The Humdrum Syntax}
	
	Types of Records
	\\Comment Records
	\\Interpretation Records
	\\Data Records
	\\Data Tokens and Null Tokens
	\\Data Sub-Tokens
	\\Spine Paths
	\\
	The Humdrum Syntax:  A Formal Definition
	\\The \textit{humdrum} Command
	\\Reprise
	
\item 

\textbf{Representing Music Using **kern (Part II)}
	
	Grace Notes, Gruppetos and Appoggiaturas
	\\Multiple Stops
	\\Further Examples
	\\Reprise
	
\item 

\textbf{MIDI Output Tools}
	
	The \textit{**MIDI} Representation
	\\The \textit{midi} Command
	\\The \textit{perform} Command
	\\Data Scrolling During Playback
	\\Changing Tempo
	\\The \textit{tacet} Command
	\\The \textit{smf} Command
	\\Reprise
	

\item 

\textbf{The Shell (I)}
	
	Shell Special Characters
	\\File Redirection
	\\Pipe (|)
	\\Shell Wildcard (*)
	\\Comment (\#)
	\\Command Delimiter (;)
	\\Background Command (\&)
	\\Shell Command Syntax
	\\Output Redirection
	\\Tee
	\\Reprise
	
\item 

\textbf{Searching with Regular Expressions}
	
	Literals
	\\Wild-Card
	\\Escape Character
	\\Repetition Operators
	\\Context Anchors
	\\OR Logical Operator
	\\Character Classes
	\\Examples of Regular Expressions
	\\Examples of Regular Expressions in Humdrum
	\\Basic, Extended, and Humdrum-Extended Regular Expressions
	\\Reprise
	
\item 

\textbf{Musical Uses of Regular Expressions}
	
	The \textit{grep} Command (Again)
	\\German, French, Italian, and Neapolitan Sixths
	\\AND-Searches Using the \textit{xargs} Command
	\\OR-Searches Using the \textit{grep -f} Command
	\\Reprise
	
\item 

\textbf{Melodic Intervals}
	
	Types of Melodic Intervals
	\\Melodic Intervals Using the \textit{mint} Command
	\\Unvoiced Inner Intervals
	\\Calculating Distance Intervals Using the \textit{mint -s} Command
	\\Simple and Compound Melodic Intervals
	\\Diatonic Intervals, Absolute Intervals and Contour
	\\Using the \textit{mint} Command
	\\Calculating Melodic Intervals Using the \textit{xdelta} Command
	\\Reprise
	
\item 

\textbf{Selecting Musical Parts and Passages}
	
	
	Extracting Spines:  The \textit{extract} Command
	\\Extraction by Interpretation
	\\Using \textit{extract} in Pipelines
	\\Extracting Spines that Meander
	\\Field-Trace Extracting
	\\
	Extracting\_Passages: The \textit{yank} Command
	\\Yanking by Marker
	\\Yanking by Delimiters
	\\Yanking by Section
	\\Examples Using \textit{yank}
	\\Using \textit{yank} in Pipelines
	\\Reprise
	
\item 

\textbf{Assembling Scores}
	
	The \textit{cat} Command
	\\The \textit{rid} Command
	\\Assembling Parts Using the \textit{assemble} Command
	\\Aligning Durations Using the \textit{timebase} Command
	\\Assembling N-tuplets
	\\Checking an Assembled Score Using \textit{proof}
	\\Other Uses for the \textit{timebase} Command
	\\Additional Uses of \textit{assemble} and \textit{timebase}
	\\Reprise
	
\item 

\textbf{Stream Editing}
	
	The \textit{sed} and \textit{humsed} Commands
	\\Simple Substitutions
	\\Selective Elimination of Data
	\\The \textit{stats} Command
	\\Eliminate Everything But...
	\\Deleting Data Records
	\\Adding Information
	\\Multiple Substitutions
	\\Switching Signifiers
	\\Executing from a File
	\\Writing to a File
	\\Reading a File as Input
	\\Reprise
	
\item 

\textbf{Harmonic Intervals}
	
	Types of Harmonic Intervals
	\\Harmonic Intervals Using the \textit{hint} Command
	\\Propagating Data Using the \textit{ditto} Command
	\\Using the \textit{ditto} and \textit{hint} Commands
	\\Determining Implicit Harmonic Intervals
	\\The \textit{ydelta} Command
	\\More Examples Using the \textit{ydelta} Command
	\\Reprise
	

\item 

\textbf{The Shell (II)}
	
	Shell Special Characters
	\\Shell Variables
	\\The Shell Greve
	\\Single Quotes, Double Quotes
	\\Using Shell Variables
	\\Aliases
	\\Reprise
	
\item 

\textbf{Creating Inventories}
	
	Filter, Sort, Count
	\\Filtering Data with the \textit{rid} Command
	\\Inventories for Multi-spine Inputs
	\\Sorting By Frequency of Occurrence
	\\Counting with the \textit{wc} Command
	\\Excluding or Seeking Rare Events
	\\Transforming and Editing Inventory Data
	\\Further Examples
	\\Reprise
	
\item 

\textbf{Fingers, Footsteps and Frets}
	
	Heart Beats and Other Esoterica
	\\The \textit{**fret} Representation
	\\Additional Features of \textit{**fret}
	\\Reprise
	
\item 

\textbf{Musical Contexts}
	
	The \textit{context} Command
	\\Harmonic Progressions
	\\Using \textit{context} with the \textit{-b} and \textit{-e} Options
	\\Using context with \textit{sed} and \textit{humsed}
	\\Linking context Outputs with Inputs
	\\Using \textit{context} with the \textit{-p} Option
	\\Reprise
	
\item 

\textbf{Strophes, Verses and Repeats}
	
	Section Labels
	\\Expansion Lists
	\\Using \textit{yank} to Extract Sections
	\\Using the \textit{thru} Command to Expand Encodings
	\\Alternative Versions
	\\Section Types
	\\Hierarchical Sections
	\\Using the \textit{yank} and \textit{thru} Commands
	\\Strophic Representations
	\\The \textit{strophe} Command
	\\Using the \textit{strophe} and \textit{thru} Commands
	\\Reprise
	
\item 

\textbf{Searching for Patterns}
	
	The \textit{patt} Command
	\\Using \textit{patt}'s Tag Option
	\\Matching Multiple Records Using the \textit{patt} Command
	\\The \textit{pattern} Command
	\\Patterns of Patterns
	\\Reprise
	
\item 

\textbf{Classifying}
	
	The \textit{recode} Command
	\\Classifying Intervals
	\\Clarinet Registers
	\\Open and Close Position Chords
	\\Flute Fingering Transitions
	\\Classifying with \textit{humsed}
	\\Classifying Cadences
	\\Orchestration
	\\Reprise
	
\item 

\textbf{Rhythm}
	
	The \textit{**recip} Representation
	\\The \textit{dur} Command
	\\Classifying Durations
	\\Using \textit{yank} with the \textit{timebase} Command
	\\The \textit{metpos} Command
	\\Changes of Stress
	\\Reprise
	

\item 

\textbf{The Shell (III)}
	
	Shell Programs
	\\
	Flow of Control: The \textit{if} Statement
	\\
	Flow of Control: The \textit{for} Statement
	\\A Script for Identifying Transgressions of Voice-Leading
		
		
		(1) Parts Out Of Range
		\\
		(2) Augmented/Diminished Melodic Intervals
		\\
		(3) Consecutive Fifths or Octaves
		\\
		(4) Doubled Leading Tone
		\\
		(5) Avoid Unisons
		\\
		(6) Crossed Parts
		\\
		(7) Parts Separated by Greater than an Octave
		\\
		(8) Overlapped Parts
		\\
		(9) Exposed Octaves
		
	Reprise
	
\item 

\textbf{Similarity}
	
	The \textit{correl} Command
	\\Using a Template with \textit{correl}
	\\The \textit{simil} Command
	\\Defining Edit Penalties
	\\The \textit{accent} Command
	\\Reprise
	
\item 

\textbf{Moving Signifiers Between Spines}
	
	The \textit{rend} Command
	\\The \textit{cleave} Command
	\\Creating Mixed Representations
	\\Reprise
	
\item 

\textbf{Text and Lyrics}
	
	The \textit{**text} and \textit{**silbe} Representations
	\\The \textit{text} Command
	\\The \textit{fmt} Command
	\\Rhythmic Feet in Text
	\\Concordance
	\\Simile
	\\Word Painting
	\\Emotionality
	\\Other Types of Language Use
	\\Reprise
	
\item 

\textbf{Dynamics}
	
	The \textit{**dynam} and \textit{**dyn} Representations
	\\The \textit{**dyn} Representation
	\\The \textit{**dB} Representation
	\\The \textit{db} Command
	\\Processing Dynamic Information
	\\Terraced Dynamics
	\\Dynamic Swells
	\\MIDI Dynamics
	\\Reprise
	
\item 

\textbf{Differences and Similarities}
	
	Comparing Files Using \textit{cmp}
	\\Comparing Files Using \textit{diff}
	\\Comparing Inventories -$\,$- The \textit{comm} Command
	\\Reprise
	
\item 

\textbf{MIDI Input Tools}
	
	The \textit{record} Command
	\\The \textit{encode} Command
	\\Reprise
	
\item 

\textbf{Repertoires and Links}
	
	The \textit{find} Command
	\\Content Searching
	\\Using \textit{find} with the \textit{xargs} Command
	\\Repertoires As File Links
	\\Reprise
	

\item 

\textbf{The Shell (IV)}
	
	The \textit{awk} Programming Language
	\\Automatic Parsing of Input Data
	\\Arithemtic Operations
	\\Conditional Statements
	\\Assigning Variables
	\\Manipulating Character Strings
	\\The \textit{for} Loop
	\\Reprise
	
\item 

\textbf{Word Sounds}
	
	The \textit{**IPA} Representation
	\\Alliteration
	\\Classifying Phonemes
	\\Properties of Vowels
	\\Vowel Coloration
	\\Rhyme and Rhyme Schemes
	\\Reprise
	
\item 

\textbf{Serial Processing}
	
	Pitch-Class Representation
	\\The \textit{pcset} Command
	\\Prime Form and Normal Form
	\\Interval Vectors Using the \textit{iv} Command
	\\Segmentation Using the \textit{context} Command
	\\The \textit{reihe} Command
	\\Generating a Set Matrix
	\\Locating and Identifying Tone-Rows
	\\Reprise
	
\item 

\textbf{Layers}
	
	Implied Harmony
	
\item 

\textbf{Sound and Spectra}
	
	The \textit{**spect} Representation
	\\The SHARC Database and \textit{spect} Command
	\\The \textit{mask} Command
	\\The \textit{sdiss} Command
	\\Connecting Humdrum with Csound -$\,$- the \textit{kern2cs} Command
	\\Sound Analysis
	\\Reprise
	
\item 

\textbf{Electronic Music Editing}
	
	The Process of Electronic Editing
	\\Establishing the Goal
	\\Documenting Encoded Data
	\\Sources
	\\Selecting a Sample from Some Repertory
	\\Encoding
	\\Transposing Instruments
	\\Instrument Identification
	\\Leading Barlines
	\\Ornamentation
	\\Editing Sections
	\\Editorialisms in the \textit{**kern} Representation
	\\Adding Reference Information
	\\Proof-reading Materials
	\\Data Integrity Using the VTS Checksum Record
	\\Preparing a Distribution
	\\Electronic Citation
	\\Reprise
	
\item 

\textbf{Systematic Musicology}
	
	Comparison Repertory
	\\Randomizing
	\\Using the \textit{scramble} Command
	\\Retrograde Controls Using the \textit{tac} Command
	\\Autophase Procedure
	\\Reprise
	
\item 

\textbf{Troubleshooting}
	
	Encoding Errors
	\\Searching Tips
	\\Pipeline Tips
	\\Reprise
	
\item 

\textbf{Conclusion}
	
	Pursuing a Project with Humdrum
	
\end{enumerate}

\begin{itemize}
\item 

\textbf{Bibliography}

\item 

\textbf{Appendix I:  Humdrum Reference Records}
	
	
	Authorship Information
	\\
	Performance Information
	\\
	Work Identification Information
	\\
	Imprint Information
	\\
	Copyright Information
	\\
	Analytic Information
	\\
	Representation Information
	\\
	Electronic Citation
	\\
	Further Reference Record Codes
	


\item 

\textbf{Appendix II:  Instrumentation Codes}
	
	
	Introduction
	\\
	Voice Range
	\\
	Voice Quality
	\\
	String Instruments
	\\
	Wind Instruments
	\\
	Percussion Instruments
	\\
	Keyboard Instruments
	

\item 

\textbf{Index to Problems}

\item 

\textbf{Index to Names, Works and Genres}

\item 

\textbf{General Index}
\end{itemize}


\par 
\begin{itemize}
\item 

\textbf{Regular Table of Contents}
\item 
\textbf{Index of Humdrum Commands}
\item 
\textbf{On-line Humdrum Course Description}
\item 
\textbf{Index to Humdrum On-line Resources}
\item 
\textbf{Humdrum Exercises}}
\item 
\textbf{Summary of **kern Music Representation}
\item 
\textbf{
Humdrum Toolkit Home Page}
\item 
\textbf{
Humdrum Toolkit Home Page (University of Virginia) 
}
\item 
\textbf{Ohio State University Music Cognition Home Page}
\end{itemize}
\\
\\
\\
\\

� Copyright 1999 David Huron
\end{document}
