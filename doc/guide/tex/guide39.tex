% This file was converted from HTML to LaTeX with
% Tomasz Wegrzanowski's <maniek@beer.com> gnuhtml2latex program
% Version : 0.1
\documentclass{article}
\begin{document}



  
  
    
      
      
      
    
  



\\
\\

\section*{Chapter39}


[\textit{Previous Chapter}]
[\textit{Contents}]
[\textit{Next Chapter}]


\section*{Trouble-Shooting}



Computers have an unbounded capacity to generate nonsense.
Even when commands appear to execute correctly, there is
no guarantee that the results are accurate or meaningful.
In this chapter we will identify some of the many things
that can go wrong when using the Humdrum Toolkit.
We will also present a number of suggestions and tips that
will help you avoid potential problems.

\par 
Errors can arise from a number of sources including corrupt or
error-prone input data, failure to anticipate special
circumstances or exceptions, improper processing,
software bugs, and incautious interpretations of results.


\subsection*{Encoding Errors}

\par 
In the first instance, the accuracy of your results will depend
on the accuracy of the input data.
Humdrum data may originate from a variety of sources.
Users may encode their own materials, or use existing
data encoded by other individuals or available from
institutional sources.
Data quality can be highly variable and there may be no
easy way to determine the accuracy of a given data set.

\par 
It is important to spend time with a data set.
Historical musicologists may spend a considerable amount of
time becoming familiar with a manuscript,
and the same practice is recommended for computational musicologists.
Users should look at the data, listen to the data,
compare the data to published sources, and generally
browse and peruse it.
Most data errors are discovered while processing the data -$\,$-
such as finding a suspicious major ninth melodic interval
in a simple song.
Over time, more and more errors are eventually discovered
and corrected.
Unfortunately, there is no magic flag that pops up to notify
us when all errors have been eliminated from an encoded
musical work.
Only over time will the user gain confidence (or lose confidence)
in a given data set.
In working with a file, we are far more apt to discover
something is wrong with the data than to learn that
the data is a pristine encoding.

\par 
Errors can be magnified by the type of processing that is applied.
For example, consider the case of an encoded repertory
that is known to have a pitch-related error rate of 1 percent.
That is, roughly 1 out of every 100 notes has an incorrect
pitch representation.
If we were to do an inventory of pitches in this repertory,
then our results would also exhibit a 1 percent error rate.
For many applications, such errors are not a problem.

\par 
However, consider what happens when we create an inventory
of melodic intervals.
One incorrect pitch will falsify
\textit{two}
melodic intervals, hence the error rates for intervals is now
2 percent.
Similarly, if we are looking at four-note chords,
a single wrong pitch will falsify an entire chord.
So we  will have roughly a 4 percent error rate for
chord identification.
If we are investigating simple chord progressions,
a single wrong note will now disrupt the identification
of two successive chords.
Thus we have an error rate of 8 percent for chord progressions.

\par 
There are two general lessons that can be drawn from these
observations.
The first lesson is obvious.
Always try to use the best quality data that is available.
When encoding your own data, aim for total accuracy.
The second lesson is more subtle.
The more data that participates in identifying
some pattern, the greater the likelihood that a single
data error will cause a problem.
Whenever possible try to restrict pattern searches to
small or concise patterns.


\subsection*{Searching Tips}

\par 
Many of the problems in computer-based musicology are evident
when searching for some pattern.
In general, there are two types of searching errors:
\textit{false hits}
and \textit{misses}.
A
\textit{false hit}
occurs when the search returns something that is not intended.
A
\textit{miss}
occurs when the search fails to catch an instance that was
intended to be a match.
Unfortunately, efforts to reduce the number of false hits
often tend to increase the number of misses.
Similarly, efforts to avoid misses often tend to increase
the number of false hits.
Precision and caution are necessary.

\par 
Search failures can arise from five sources:
(1) corrupt or inaccurate data,
(2) failure to search all of the intended data,
(3) inaccurate or inappropriate definition of the search template,
(4) failure to understand how a given search tool or option operates,
(5) failure on the part of the user to form a clear idea
of what is being sought.
Let's deal with each of these problems in turn.

\par 
\textbf{(1) No search can produce accurate results if the data}
\textbf{to be searched is inaccurate.}
You can increase the accuracy of your search by choosing
high quality data and preparing the files in
an appropriate manner.

\par 
\textbf{Tips:}
\begin{itemize}
\item 
Use the
\textbf{humdrum}
command to ensure that the input data conforms to the Humdrum
syntax.
\item 
Use the
\textbf{humdrum} \textbf{-v}
command to determine whether the kind of data (signifiers)
you are interested in are truly present in all of the files
to be searched.
\item 
Use the
\textbf{proof}
command to ensure that any
\texttt{**kern}
data is properly encoded.
\item 
Visually inspect sample passages from the input data.
Do not rely solely on the
\textbf{ms}
command;
instead, look at the actual ASCII text data using the
\textbf{more}
command.
\item 
Use the
\textbf{midi}
and
\textbf{perform}
commands to listen to sample passages;
ensure that the data makes sense.
\item 
If you are uncertain of the quality of the data,
try encoding a randomly selected subset and then
use the UNIX
\textbf{diff}
command to identify any differences between the original
data and the re-encoded data.
\item 
Always read any release notes or README files
that accompany the data.
\item 
Use
\textbf{grep}
to search for
\textit{warning}
and
\textit{Nota Bene}
reference records (\texttt{!!!RWG:} and \texttt{!!!ONB:}).
These records may contain important editorial notes or warnings.
\item 
Where appropriate, expand files to through-composed versions
(using
\textbf{thru}) before searching.
If more than one editorial version is present in a document,
select the most appropriate edition before processing.
\item 
Create an inventory of all the types of data tokens
present in an input.
Inspect the inventory list to determine whether any
unexpected data are present.
\item 
If necessary, eliminate certain types of data that might
confound or interfere with your search in some way.
Use
\textbf{rid},
\textbf{grep -v},
\textbf{extract} \textbf{-i},
\textbf{sed}
and/or
\textbf{humsed}
to restrict the data.
\end{itemize}

\par 
\textbf{(2) Ensure that you are searching all of the intended data:}
\begin{itemize}
\item 
Use
\textbf{grep}
to search for titles, or composers, or opus numbers, etc.,
in order to ensure that the file or files you are
searching are the ones you want.
Also check to ensure that you are not searching
materials that don't belong in the input.
\item 
Be wary of searching duplicated materials.
Create inventories of titles, opus numbers, etc.,
and use
\textbf{uniq -d}
to identify unwanted duplicate copies of works or files.
\item 
Use the
\textbf{ls -l}
or
\textbf{wc}
commands to determine the size of the search data.
Does the amount of input data seem unduly small
or unduly large?
\item 
Use the
\textbf{find}
command to search the system for other files that
ought to be included in the search task.
\end{itemize}

\par 
\textbf{(3) One of the most common problems in searching arises}
\textbf{from inaccurate or inappropriate search templates.}

\par 
\textbf{Tips:}
\begin{itemize}
\item 
Be careful when formulating regular expressions.
Read aloud the meaning of the regular expression.
\item 
Do not use
\textit{extended}
regular expressions with the
\textbf{grep}
command.
Use
\textbf{egrep}
instead.
\item 
Ensure that you know which characters in your search
template are meta-characters.
\item 
Execute your command from a shell script file so that
you don't inadvertently make a typing error when
entering the command.
\item 
Maintain a command history file so that you have a permanent
record of what you did.
Depending on the system settings, the UNIX
\textbf{history}
command will display the past 100 (or more) commands
you have executed.
Place this information in a permanent \texttt{record} file
as follows:

\par 

\texttt{history > record}


\par 
In addition, keep records of the precise regular expressions
used for a given project.
These records will help you determine later whether you
made a mistake.
For added security, print-out these files and glue them
into a lab book.
\item 
Create a test file containing different patterns, and test
the ability of your regular expressions to catch all cases.
Included "lures" in your test -$\,$- i.e., patterns
that are close to what you want, but should be rejected.
\item 
Use extra caution when using "not" logic.
For example, the
\textbf{grep}
expression "not-A" (i.e. \texttt{[\^{}A]})
will still match records containing the letter \texttt{A}
as long as one non-A letter is present.
The commands

\texttt{grep [\^{}A]}
and\texttt{grep -v A}


\par 
are
\textit{not}
the same.
\item 
Compare outputs from a search that you know ought to increase
the number of false hits.
Compare outputs from a search that you know ought to miss some
sought patterns.
\item 
Translate the data to another representation and repeat the
search using a different pattern tailored to the new representation.
The results should be identical.
\item 
Maintain a file containing regular expressions you have tested
so you can re-use them in later projects.
\item 
Visually inspect the ASCII output to ensure that the results
are correct.
Remember that visual inspection will only help you identify
\textit{false hits.}
Visual inspection of the output will not help you identify
\textit{misses.}
\item 
Use the
\textbf{midi}
and
\textbf{perform}
commands to proof-listen to your output.
Again remember that aural inspection will only help
you identify
\textit{false hits.}
\item 
Ask whether the output makes sense.
Given the amount of music searched, does it make
sense to find the number of occurrences found?
\item 
Try making a slight modification to your pattern template
-$\,$-  a modification that you know should produce
a different result.
\item 
Look for converging evidence.
Try two or three contrasting approaches to ensure
that the same answer arises for each approach.
For example, try searching each part individually using the
\textbf{extract}
command.
\end{itemize}

\par 
\textbf{(4) Ensure that you understand how a given search tool or option}
\textbf{operates.}

\par 
\textbf{Tips:}
\begin{itemize}
\item 
Remember that
\textit{extended}
regular expressions require the use of
\textbf{egrep}
rather than \textbf{grep}.
\item 
Re-read the documentation to ensure that each software
tool does what you think it does.
\item 
Refer to the examples given in the
\textit{Humdrum Reference Manual}
in order to ensure that you understand what a given option does.
\item 
Compare outputs using different options.
Ensure that your selected option(s) is matching
the correct pattern.
\item 
Use the
\textbf{humver}
command to determine which version of the Humdrum Toolkit
you are using.
Ensure that the documentation pertains to the correct version.
\item 
Read the "Release Notes" for the software you use.
Known software bugs are often reported in such notes or in the
documentation.
\item 
Report discovered bugs to the software's author.
Even if the software is not revised, other users should
be informed of the problem.
\end{itemize}

\par 
\textbf{(5) Perhaps the most onerous problems in pattern searching}
\textbf{arises when the user fails to have a clear understanding}
\textbf{of what is being sought:}

\par 
\textbf{Tips:}
\begin{itemize}
\item 
Think carefully about the search problem.
What precisely are you looking for?
\item 
Inspect the input to familiarize yourself with various
contexts and possible variants.
\item 
Check your search by carrying out a manual search of
a random subset of the data.
\end{itemize}

\par 
Compared with manual research, computer searches are impressively fast.
However, don't let yourself be caught-up by the speed of interaction.
Take your time and reflect on the problem being addressed.
Formulate a search strategy away from the computer so that
you have time to consider possible confounds.


\subsection*{Pipeline Tips}

\par 
Apart from searching tasks, most Humdrum processing involves two or
more software tools linked in a pipeline.
Pipelines can obscure all sorts of processing errors.

\par 
\textbf{Tips:}
\begin{itemize}
\item 
Slowly assemble your pipeline by adding one software tool
at a time.
Visually inspect the output following the addition of each process.
\item 
Start with a small volume of input data.
Once you have some confidence in your pipeline use a
\textit{different}
sample of input data.
Again add one software tool at a time while inspecting
the results at each stage.
\item 
Use the UNIX
\textbf{tee}
command to generate files at intermediate points in the processing.
Use the
\textbf{assemble}
command to align inputs and outputs at various stages in
the processing.
\item 
Execute your finalized pipeline from a shell script in order to
avoid undetected typing errors.
\end{itemize}



\subsection*{Reprise}

\par 
In research-oriented activities, it is essential to exercise care when
relying on computer-based methods.
Computers have an unbounded capacity to generate false results.
Unfortunately, computer outputs often seem deceptively authoritative.
Take your time and develop a coherent strategy for solving
a particular problem.
Test your materials and processes, and maintain good records
of what you have done.
For critical tasks, always use two or more independent methods
to ensure that the results agree.
In general, cultivate a skeptical attitude;
wise users are wary users.

\\
\begin{itemize}
\item 

\textbf{Next Chapter}
\item 

\textbf{Previous Chapter}
\item 

\textbf{Table of Contents}
\item 

\textbf{Detailed Contents}
\\\\

� Copyright 1999 David Huron
\end{document}
