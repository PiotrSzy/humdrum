% This file was converted from HTML to LaTeX with
% Tomasz Wegrzanowski's <maniek@beer.com> gnuhtml2latex program
% Version : 0.1
\documentclass{article}
\begin{document}



  
  
    
      
      
      
    
  



\\
\\

\section*{Chapter11}


[\textit{Previous Chapter}]
[\textit{Contents}]
[\textit{Next Chapter}]


\section*{Melodic Intervals}



A musical interval is the distance between two pitches.
When the pitches are consecutive the distance is referred to as a
\textit{melodic interval;}
when the pitches are concurrent the distance is referred to as a
\textit{harmonic interval.}

\par 
The simplicity of these definitions is deceptive.
In real music, the determination of pitch intervals can be
surprisingly complicated.
In this chapter we will discuss Humdrum tools related to
melodic pitch intervals -$\,$- specifically the
\textbf{mint}
(melodic interval) and
\textbf{xdelta}
commands.
Discussion of harmonic intervals will be delayed until
Chapter 15.


\subsection*{Types of Melodic Intervals}

\par 
Example 11.1 provides a contrived illustration of seven different types
of melodic intervals.
(The corresponding \texttt{**kern} representation is given on the
following page.)
The simplest melodic interval is calculated between successive
pitches within the same voice or part.
We might call this a
\textit{voiced melodic interval.}
Examples of voiced melodic intervals are the descending perfect
fourth between the first two notes of the upper part,
and the rising semitone at the beginning of the lower part.

\\\\
\textbf{Example 11.1}  Types of melodic intervals.
\\
\\

\\
\par 
Even within a monophonic score, successive pitches may have one
or more rests interposed between them.
Depending on the research task, the interval spanning across the
rest(s) may or may not be considered important.
We might call such intervals
\textit{interrupted melodic intervals.}
An example of an interrupted melodic interval is the ascending perfect
fourth in the upper voice between the last note of the first measure
(A4) and the first note of the last measure (D5).
\\\\
\textbf{Example 11.1 continued.}

\par 

\texttt{**kern**kern
\texttt{*staff1*staff1
\texttt{*M4/4*M4/4
\texttt{=1-=1-
\texttt{4.e2cc
\texttt{8f.
\texttt{4.e4g
\texttt{.8a
\texttt{[8f8r
\texttt{=2=2
\texttt{2f]2dd
\texttt{4d 4g4b 4ff
\texttt{4c 4e 4g4cc 4ee
\texttt{*-*-}

In some cases, implied melodic intervals may arise by the interaction
of two or more parts.
For example, Example 11.2 shows a plausible reinterpretation of
the voicings for the first measure.
Here the quarter-note G is presumed to continue
from the preceding eighth-note F rather than from the half-note C.
In the context of the original \texttt{**kern} encoding above,
we might call such intervals
\textit{cross-voice melodic intervals.}

\textbf{Example 11.2}  Possible re-interpretation of opening measure for Example 11.1.



\par 
In the second measure, both parts are encoded using multiple stops.
In the upper part, two successive double stops are encoded.
In the case of multiple-stops, it is common to perceive the outer-most
notes as connected.
Hence, the B would resolve to the C and the F would resolve to the E.
We might call such intervals
\textit{unvoiced outer intervals.}
We may call them
\textit{unvoiced}
because they aren't encoded using separate spines.
The voicings are implied, principally because the notes form
an `upper' or `lower' part.

\par 
The multiple-stops in the lower voice illustrate even more possibilities.
Apart from the unvoiced outer intervals (D->C and G->G),
there are other possible melodic intervals.
These include D->E and G->E.
We might refer to such intervals as
\textit{unvoiced inner intervals.}

\par 
In many research tasks (such as identifying melodic variations),
important interval relationships may stretch across several intervening notes.
In Example 11.1, for example, the two half-notes in the upper voice
might be viewed as forming an ascending major second interval (i.e. C->D).
We might call such intervals
\textit{distance intervals.}

\par 
Finally, although in most situations tied notes should be treated
as a single note, in some circumstances there is merit to
considering each notehead as independent.

\par 
By way of summary, we have distinguished no less than seven
different types of melodic intervals:
voiced melodic intervals, interrupted melodic intervals,
cross-voice melodic intervals,
unvoiced outer intervals,
unvoiced inner intervals,
distance intervals,
and tied note intervals.

\par 
Humdrum commands related to melodic intervals provide users
with several alternative ways of interpreting melodic intervals.
Users are typically interested in only certain types of intervals
and so it is useful to restrict the outputs to specified
interval classes.

\par 
Apart from the question of types of melodic intervals,
melodic intervals can be calculated according to a variety of
\textit{units}.
Depending on the circumstance, the user may wish to calculate
diatonic intervals, semitones, cents, frequency differences,
or even differences in cochlear coordinates.


\subsection*{Melodic Intervals Using the \textit{mint} Command}

\par 
The Humdrum
\textbf{mint}
command calculates melodic intervals for pitch-related representations
such as
\texttt{**kern},
\texttt{**pitch},
\texttt{**solfg}
and
\texttt{**Tonh}.
Output intervals are expressed using the traditional diatonic terms
where both interval quality and interval size are specified.
Interval qualities include perfect (\texttt{P}),
major (\texttt{M}),
minor (\texttt{m}),
augmented (\texttt{A})
and diminished (\texttt{d}).
Interval qualities may also be doubly augmented (\texttt{AA}),
triply diminished (\texttt{ddd}) and so on.
Diatonic interval sizes are indicated by numbers
(1=unison, 2=second, ... 8=octave, 9=ninth, etc.).
Ascending and descending intervals are distinguished by a
leading plus sign (+) or minus sign (-) respectively.

\par 
In the default operation,
\textbf{mint}
outputs three of the seven types of melodic intervals.
These are
\textit{voiced melodic intervals,}
\textit{unvoiced outer intervals}
and
\textit{interrupted melodic intervals;}
By way of illustration, Example 11.3 shows the output from the
\textbf{mint}
command for the input shown in Example 11.1.

\textbf{Example 11.3}  Default interval outputs from the \textbf{mint} command corresponding to Example 11.1.
\par 

\texttt{**mint**mint
\texttt{*M4/4*M4/4
\texttt{=1-=1-
\texttt{..
\texttt{+m2.
\texttt{-m2-P4
\texttt{.+M2
\texttt{+m2r
\texttt{=2=2
\texttt{.+P4
\texttt{-m3 +M2-m3 +m3
\texttt{-M2 P1+m2 -m2
\texttt{*-*-}

Notice that the
\textit{interrupted interval}
(spanning the rest)
has been calculated, and that no unison has appeared for the tied note
in the lower voice.

\par 
If desired, the unison intervals between successive tied notes
can be output via the
\textbf{-t}
option for
\textbf{mint.}

\par 
Sometimes it is useful to maintain the initial starting pitches
in the output.
The presence of these "offset" pitch values can prove useful
in later reconstructing the original pitches from the
\texttt{**mint}
interval data.
When the
\textbf{-o}
option is invoked,
\textbf{mint}
outputs the initial starting pitches (placed in square brackets) from
which the subsequent melodic intervals have been calculated.

\par 
In order to avoid outputting interrupted intervals, the
\textbf{-b}
(break) option can be used.
This option requires a subsequent regular expression that
defines the contexts where the interval calculation should be
suspended and restarted.
A common invocation would identify \texttt{**kern} rests (\texttt{r})
as a suitable place to break melodic interval calculations.
For example,

\par 

\texttt{ mint -b r inputfile}


\par 
would produce the following output when applied to Example 11.1:
\\\\

\texttt{**mint**mint
\texttt{*M4/4*M4/4
\texttt{=1-=1-
\texttt{..
\texttt{+m2.
\texttt{-m2-P4
\texttt{.+M2
\texttt{+m2r
\texttt{=2=2
\texttt{P1.
\texttt{-m3 +M2-m3 +m3
\texttt{-M2 P1+m2 -m2
\texttt{*-*-}

Notice that the perfect fourth (+P4) has been replaced by a
null token at the beginning of measure 2.
In addition, the rest token `\texttt{r}' has been
echoed just prior to the barline.

\par 
Depending on the regular expression given, the
\textbf{-b}
option can used for a variety of specialized intervals.
For example, suppose that we wanted to avoid calculating
intervals between the last note of a phrase and the
first note of the next phrase.
In the
\texttt{**kern}
representation, the open and closed
curly braces are used to indicate the beginnings and ends
of phrases.
We need to tell
\textbf{mint}
to break interval calculations each time an end-of-phrase
signifier is encountered:

\par 

\texttt{mint -b '\}' inputfile}


\par 
Similarly, the \texttt{**kern} representation uses the semicolon (\texttt{;})
to represent pauses.
We might instruct
\textbf{mint}
to avoid calculating intervals between notes having pauses and
the subsequent note:

\par 

\texttt{mint -b ';' inputfile}


\par 
Since the
\textbf{-b}
option accepts regular expressions, we can combine patterns.
For example, the following command instructs
\textbf{mint}
to calculate melodic intervals, not including intervals spanning
phrase boundaries, and not following notes with pauses:

\par 

\texttt{mint -b '[;\}]' inputfile}



\subsection*{Unvoiced Inner Intervals}

\par 
Unvoiced inner intervals can be included in the output
by using the
\textbf{-i}
or
\textbf{-I}
options.
With the
\textbf{-I}
option, unvoiced inner intervals appear in the output
in parentheses.
For example, the following output is generated for Example 1.1
with the
\textbf{-I}
option.
Notice the addition of \texttt{(+M2)} and \texttt{(-2)}.
The rising major second arises from the pitches D4 and E4;
the falling minor third arises from the pitches G4 and E4.
\\\\

\texttt{**mint**mint
\texttt{*M4/4*M4/4
\texttt{=1-=1-
\texttt{..
\texttt{+m2.
\texttt{-m2-P4
\texttt{.+M2
\texttt{+m2r
\texttt{=2=2
\texttt{P1.
\texttt{-m3 +M2-m3 +m3
\texttt{-M2 (+M2) (-m3) P1+m2 -m2
\texttt{*-*-}

With the
\textbf{-i}
option, the parentheses surrounding the unvoiced inner intervals
would be omitted.


\subsection*{Calculating Distance Intervals Using the \textit{mint -s} Command}

\par 
Another option provided by
\textbf{mint}
is the
\textbf{-s}
or skip option.
Like the
\textbf{-b}
option, this option requires a subsequent regular expression.
Any token matching this expression is transformed to a null
data token and is ignored when processing.
One possible use for this option is to help calculate
\textit{distance intervals.}
Consider Example 11.4 where all of the durations are either
sixteenth notes or eighth notes.
Suppose we wanted to calculate the intervals only between
the eighth notes.
\\\\

\par 
\textbf{Example 11.4}




\par 

\texttt{**kern
\texttt{*M4/4
\texttt{=1-
\texttt{8cc
\texttt{16b
\texttt{16cc
\texttt{8g
\texttt{16f\#
\texttt{16g
\texttt{=2
\texttt{8e
\texttt{16d\#
\texttt{16e
\texttt{8c
\texttt{8r
\texttt{*-}

We can use the skip option to instruct
\textbf{mint}
to ignore any note token matching the string `\texttt{16}':

\par 

\texttt{ mint -s 16 inputfile}


\par 
This command would produce an output that highlights the
descending arpeggiated major chord -$\,$- from C5 to G4 (down a P4),
to E4 (down a m3) to C4 (down a M3).
\\\\




\texttt{**mint
\texttt{*M4/4
\texttt{=1-
\texttt{.
\texttt{.
\texttt{.
\texttt{-P4
\texttt{.
\texttt{.
\texttt{=2
\texttt{-m3
\texttt{.
\texttt{.
\texttt{-M3
\texttt{*-}

Using duration information is a somewhat limited technique for
calculating distance intervals.
Typically, users will want to define much more refined ways of
identifying structural tones.
More sophisticated methods for calculating distance intervals
are discussed in
Chapter 35
on "Layers."

\par 
Cross-voice melodic intervals can be calculated by
amalgamating several spines into a single spine.
In
Chapter 26
we will learn more about the
\textbf{cleave}
command.
But here is a typical use:

\par 

\texttt{cleave -d ' ' -i '**kern' -o '**kern' example11a}


\par 
With the \texttt{**kern} encoding for Example 11.1 as input,
the corresponding output would be:
\\\\

\texttt{**kern
\texttt{*M4/4
\texttt{=1- =1-
\texttt{4.e 2cc
\texttt{8f
\texttt{4.e 4g
\texttt{8a
\texttt{[8f 8r
\texttt{=2 =2
\texttt{2f] 2dd
\texttt{4d 4g 4b 4ff
\texttt{4c 4e 4g 4cc 4ee
\texttt{*-}

Note that this output doesn't quite conform to the
\texttt{**kern} syntax:
the barlines have been duplicated as double-stops, and the
durations aren't right for multiple-stops.
We can clean up the output using
\textbf{humsed,}
but the incoherent durations won't cause problems if our intention
is to calculate pitch intervals.

\par 
If we pipe the above output through the
\textbf{mint}
command, the appropriate command pipeline becomes:

\par 

\texttt{cleave -d ' ' -i '**kern' -o '**kern' example11a $\backslash$

| humsed 's/ =.*//' | mint -I}



\par 
The corresponding melodic interval output is:
\\\\

\texttt{**mint
\texttt{*M4/4
\texttt{=1-
\texttt{.
\texttt{+m2 -P5
\texttt{-m2 +M2
\texttt{+P4 +M2
\texttt{-M3 r
\texttt{=2
\texttt{+P4
\texttt{-m3 (+M2) (-P5) (+A4) (-m3) +m3
\texttt{-M2 (-m3) (-P5) (P1) (-M3) (+P4) (+m2) -m2
\texttt{*-}


\par 

\subsection*{Simple and Compound Melodic Intervals}

\par 
Of course, some melodic variants alter the octave placement
of pitches.
The
\textbf{mint -c}
option outputs compound intervals (i.e. intervals of an
octave or greater) as non-compound equivalents.
For example, the interval of a major tenth (\texttt{M10}) will be
output as a major third (\texttt{M3}).

\par 

\subsection*{Diatonic Intervals, Absolute Intervals and Contour}

\par 
The
\textbf{mint}
command provides three further options of interest.
The
\textbf{-d}
option causes
\textbf{mint}
to output only the diatonic interval size without the interval quality
information.
The
\textbf{-a}
option causes
\textbf{mint}
to output absolute pitch intervals without distinguishing
ascending intervals from descending intervals.
That is, the leading plus (+) and minus (-) signs are discarded.

\par 
Finally, the
\textbf{-A}
option causes
\textbf{mint}
to output just one of three states:
a plus sign (+) indicating a rising interval,
the minus sign (-) indicating a falling interval,
and the number zero (0) indicating no pitch movement (i.e., unison).
In short, the
\textbf{-A}
option outputs only gross contour.
The
\textbf{-a}
and
\textbf{-A}
options are complementary.


\subsection*{Using the \textit{mint} Command}

\par 
Consider some of the following uses of the \textbf{mint} command.



\par 
Are there any major or minor ninth melodic intervals in the
file \texttt{Sinatra}?

\par 

\texttt{mint Sinatra | grep '[Mm][9]'}




\par 
Are there any compound melodic intervals in the file \texttt{Piaf}?

\par 

\texttt{mint Piaf | egrep '([Mm][9])|([MmPAd][1-9][0-9]''}



\par 
Are descending seconds more common than ascending seconds
in melodies by Maurice Chevalier?

\par 

\texttt{mint Chevalier* | grep -c '+[Mm]2'}
\\
\texttt{mint Chevalier* | grep -c '-[Mm]2'}


\par 
An alternative way of achieving the same goal might simplify
the regular expression to \textbf{grep} and use the \textbf{-d} (diatonic)
option for \textbf{mint}:

\par 

\texttt{mint -d Chevalier* | grep -c '+2'}
\\
\texttt{mint -d Chevalier* | grep -c '-2'}


\par 
Identify whether there are any tritone melodic intervals in
any of the vocal parts of a score:


\par 

\texttt{extract -i '*Ivox' Platters | mint -c | egrep '(A4)|(d5)'}


\par 
Here we have used the extended regular expression capabilities
of \textbf{egrep} to specify an either/or pattern.

\par 
Suppose we had a directory containing only files
encoding melodies using the
\texttt{**mint}
representation.
Does any melody in the current directory contain both an
ascending major sixth and a descending major sixth?

\par 

\texttt{grep -l '+M6' * | xargs grep '-M6'}


\par 
Do the vocal lines of Louis Jourdain contain successive ascending
major thirds (such as forming an augmented triad)?

\par 

\texttt{mint Jourdain | grep -v = | uniq -d | grep '+M3'}


\par 
What is the longest run of rising intervals in the vocal lines of Marlene Dietrich?

\par 

\texttt{mint -A Dietrich | grep -v = | uniq -cd | grep '+' | sort -n}



\subsection*{Calculating Melodic Intervals Using the \textit{xdelta} Command}

\par 
Often it is useful to calculate melodic intervals in purely
numerical values, such as the number semitones or the number of cents.
The
\textbf{xdelta}
command provides a general tool for calculating numerical differences
between successive values within individual spines.
In order to use
\textbf{xdelta}
to calculate semitone differences, we first need to transform
our representation to
\texttt{**semits}
(discussed in
Chapter 4).
Recall that in the \texttt{**semits} representation, middle C
is designated by the value zero, and all other pitches are
represented by their (positive or negative) semitone distance.
A C-major scale would appear as follows:

\texttt{**semits
\texttt{0
\texttt{2
\texttt{4
\texttt{5
\texttt{7
\texttt{9
\texttt{11
\texttt{12
\texttt{*-}


\par 
We can transform this representation to semitone intervals as follows:

\par 

\texttt{xdelta} \textit{inputfile}


\par 
For the above scale, the output would be:

\texttt{**Xsemits
\texttt{.
\texttt{2
\texttt{2
\texttt{1
\texttt{2
\texttt{2
\texttt{2
\texttt{1
\texttt{*-}


\par 
Notice that the initial numerical value has been replaced by a null token,
and all subsequent values represent the numerical
\textit{difference}
between successive values.
If the scale had been descending in pitch, then the difference values
would be negative.

\par 
Notice also that the input interpretation (\texttt{**semits}) has been
modified to \texttt{**Xsemits}).
The input representation for
\textbf{xdelta}
does not matter.
The output is always modified so the letter X is prepended to
the representation.
This means that
\textbf{xdelta}
could as easily be used to calculate differences in cents (\texttt{**Xcents}),
frequency (\texttt{**Xfreq}) or any other representation that contains
numbers.

\par 
When
\textbf{xdelta}
encounters multiple-stops, it behaves in a manner similar to the
\textbf{mint}
command by calculating the numerical equivalent of unvoiced inner intervals.
Consider the following example:

\texttt{**semits
\texttt{3
\texttt{2 5
\texttt{-1 7 14
\texttt{12
\texttt{*-}


\par 
The
\textbf{xdelta}
command produces the following output:

\texttt{**Xsemits
\texttt{.
\texttt{-1 2
\texttt{-3 (5) (2) 9
\texttt{13 5 -2
\texttt{*-}


\par 
Once again, the interpretation has been modified to \texttt{**Xsemits}
and the leading value has been changed to a null token.
In going from the value 3 to the values 2 and 5,
the output differences are -1 and +2 respectively.
In going from the values 2 and 5 to the values -1, 7 and 14,
we see the outer differences (-1 - 2 = -3) and (14 - 5 = 9).
The inner differences are output in parentheses:
(7 - 2 = 5) and (7 - 5 = 2).

\par 
Like the
\textbf{mint}
command,
\textbf{xdelta}
provides a
\textbf{-b}
option to break calculations of numerical differences
and a
\textbf{-s}
option to skip or completely ignore certain data tokens when processing.
An important use of the
\textbf{-s}
option is to ignore barlines.
Consider the following example:

\texttt{**cents
\texttt{100
\texttt{400
\texttt{800
\texttt{=2
\texttt{600
\texttt{*-}


\par 
The proper way to calculate differences in cents is to ensure that
measure numbers are ignored:

\par 

\texttt{xdelta -s \^{}= inputfile}


\par 
Failure to skip the barline will cause a difference to be calculated
between 800 and =2 (i.e., -798) and between =2 and 600 (i.e., 598).

\par 
Outputs from
\textbf{xdelta}
can be processed again using
\textbf{xdelta}
in order to calculate the differences of the differences.
For example, we can calculate the second derivative of
successive cents by using a pipeline containing two
\textbf{xdelta}
commands:

\par 

\texttt{xdelta -s \^{}= \textit{inputfile}}
\texttt{| xdelta -s \^{}= }




\subsection*{Reprise}
\par 
Intervals come in a mind-boggling range of types and sizes.
Interval sizes can be measured in a variety of ways.
They can be characterized as diatonic qualities such
as minor sevenths or augmented sixths.
They can be measured in terms of semitone distance -$\,$- or
even in cents or hertz (frequency difference).
Only the diatonic size may be of interest (e.g., "a fifth"),
and compound intervals (e.g., major tenth) can be expressed
by their non-compound equivalents (major third).

\par 
Melodic intervals can be described as ascending or descending,
or as absolute distances without regard for direction.
Types of melodic intervals can be distinguished according to
how successive notes are voiced, and according to spans across rests
or across less important pitches.
We defined seven different types of melodic intervals including
\textit{voiced melodic intervals},
\textit{interrupted melodic intervals},
\textit{cross-voice melodic intervals},
\textit{unvoiced outer intervals},
\textit{unvoiced inner intervals},
\textit{distance intervals},
and
\textit{tied note intervals}.

\par 
In this chapter we have seen how to use the
\textbf{mint}
command to calculate these various kinds of intervals.
Specifically, we have illustrated how to calculate
voiced intervals, interrupted intervals, unvoiced outer intervals
and unvoiced inner intervals.
In
Chapter 26
we will show how to calculate cross-voice intervals,
and in
Chapter 35
("Layers") we will consider how to
calculate distance intervals.

\par 
We have also seen how
\textbf{xdelta}
can be used to measure purely numerical distances between successive values.
As we will see,
\textbf{xdelta}
will prove useful in many other applications apart from calculating
pitch distances.

\\
\begin{itemize}
\item 

\textbf{Next Chapter}
\item 

\textbf{Previous Chapter}
\item 

\textbf{Table of Contents}
\item 

\textbf{Detailed Contents}
\\\\

� Copyright 1999 David Huron
\end{document}
