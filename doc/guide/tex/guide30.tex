% This file was converted from HTML to LaTeX with
% Tomasz Wegrzanowski's <maniek@beer.com> gnuhtml2latex program
% Version : 0.1
\documentclass{article}
\begin{document}



  
  
    
      
      
      
    
  



\\
\\

\section*{Chapter30}


[\textit{Previous Chapter}]
[\textit{Contents}]
[\textit{Next Chapter}]


\section*{MIDI Input Tools}



The Humdrum Toolkit provides two tools for inputting
MIDI data.
In this chapter we briefly introduce the
\textbf{record}
and
\textbf{encode}
commands.
These tools provide ways for capturing MIDI input
and translating them to representations that conform to
the Humdrum syntax.
The
\textbf{record}
command translates a live or computer-generated MIDI performance
to the
\texttt{**MIDI}
representation.
The
\textbf{encode}
command provides an interactive editor that translates
MIDI events to any pre-defined or user-defined Humdrum representation.

\par 
Note that the Humdrum
\textbf{record}
and
\textbf{encode}
commands are currently available only for the DOS operating system.
These commands can be used only with computers that have MIDI
capable hardware.


\subsection*{The \textit{record} Command}

\par 
The
\textbf{record}
command captures a stream of input MIDI data and
translates this data into the Humdrum \texttt{**MIDI}
representation described in
Chapter 7.
The input data is obtained from a MIDI instrument
such as a keyboard synthesizer.

\par 
Recording commences as soon as the command is invoked
and recording ceases when any key is pressed -$\,$- with
the exception of the space bar.
Pressing the space bar causes a \texttt{**MIDI} barline
token to be output.
Measure numbers are incremented automatically beginning
with measure 1.

\par 
Only MIDI key-press activity (including after-touch)
information is recorded.
MIDI "system-exclusive" instructions and other
non-key-press data are not recorded.

\par 
Each MIDI channel is represented using a separate Humdrum spine.
New spines are added automatically during the recording
in response to MIDI activity appearing in any MIDI channel.
Once a MIDI channel becomes active, the corresponding Humdrum spine
continues to be output until the recording is terminated.

\par 
The recorded output is normally directed to a file as
in the following:

\par 

\texttt{record > filippa}



\subsection*{The \textit{encode} Command}

\par 
The
\textbf{encode}
command provides an interactive editor for capturing Humdrum
data from a MIDI input, such as a keyboard synthesizer.
MIDI events are mapped to user-defined signifiers so
\textbf{encode}
can be used to enter data directly into a particular representation
such as
\texttt{**kern},
\texttt{**fret},
\texttt{**solfg},
etc.
Since the mapping of MIDI events to Humdrum data tokens
is arbitrary, users can enter data using a representation
design by the user.

\par 
The
\textbf{encode}
command is limited to encoding information one spine at a time.
A typical use of
\textbf{encode}
is to encode individual musical parts or voices using a
representation like **kern.
A full score is generated from the individual parts using
the
\textbf{timebase}
and
\textbf{assemble}
commands.

\par 
The
\textbf{encode}
command implements a full-screen interactive editor similar to the
\textbf{vi}
text editor.
When invoked, the screen is divided into three display regions,
including a \textit{status window}, a \textit{command window},
and a larger \textit{text window}.
The
\textit{status window}
displays various items of basic information about the file
being edited.
The
\textbf{command window}
allows the user to execute general-purpose commands to
manipulate the data.
The
\textbf{text window}
is used to display, encode and edit the encoded Humdrum text.

\par 
When the
\textbf{encode}
command is invoked, its operation is determined by
a configuration file (that may be written or edited by the user).
This configuration file contains a series of definitions
that map MIDI events to output strings.
For example, the instruction

\par 

\texttt{KEY 60 middle-C}


\par 
assigns the key-on event for MIDI key \#60 to the string \texttt{middle-C}.
Each time key \#60 is depressed, the string \texttt{middle-C} will appear
in the text window.

\par 
Such mappings can be made for each individual MIDI key.
In addition, the user may define mappings for \textit{key velocity}.
For example, the following instruction in the configuration file
will map any key-velocities between 90 and 127 MIDI units to the
apostrophe character (the
\texttt{**kern}
signifier for a staccato note):

\par 

\texttt{VEL 90 127 '}


\par 
A third class of mapping instructions relates to the elapsed time
between MIDI key onsets -$\,$- "delta time" or DEL.
Consider, for example, the following configuration instructions:

\par 

\texttt{DEL 48 80 8}
\\
\texttt{DEL 81 112 8.}
\\
\texttt{DEL 113 160 4}
\\
\texttt{DEL 161 224 4.}
\\
\texttt{DEL 225 320 2}


\par 
These instructions divide the elapsed time between key onsets into
five ranges.
When the elapsed time lies between 48 and 80 clock ticks,
the string "\texttt{8}" is output.
When the elapsed time lies between 81 and 112 clock ticks,
the string "\texttt{8.}" is output.
And so on.
This allows the durations to be classified as either an eighth note,
a dotted eighth note, a quarter note, a dotted quarter note, or
a half note.
Once again, the user is free to map events to any arbitrary output
string and to arrange the ranges and number of classes as needed.

\par 
The Humdrum Toolkit provides a large selection of predefined configuration
files for use with \textbf{encode}.
Depending on the configuration, the input may be mapped to
a particular representation such as **kern.
For example, Humdrum provides a configuration file that is
optimized for encoding lute tablatures using the **fret representation.
Other configuration files are optimized for particular keys.
For example, one may select a configuration file that
interprets the MIDI events in the key of C\# minor;
in this case, playing the pitch C will result in a default encoding of B\#.

\par 
The
\textbf{encode}
command provides many additional features that facilitate
encoding Humdrum data from a MIDI input device.
These include setting metronome values, assigning the beat,
rearranging the order of signifiers, making global and local
substitutions, replaying an interpreted input, defining
buffers and string constants, and so on.
Using the configuration files, users can tailor the
\textbf{encode}
editor to suit specific needs and skills.



\subsection*{Reprise}

\par 
In this chapter we have briefly identified two tools for
capturing MIDI-related input:
\textbf{encode}
and
\textbf{record}.
These tools allow MIDI data to be translated to a Humdrum format.
Further information regarding these tools is given in the
\textit{Humdrum Toolkit Reference Manual.}

\\
\begin{itemize}
\item 

\textbf{Next Chapter}
\item 

\textbf{Previous Chapter}
\item 

\textbf{Table of Contents}
\item 

\textbf{Detailed Contents}
\\\\

� Copyright 1999 David Huron
\end{document}
