% This file was converted from HTML to LaTeX with
% Tomasz Wegrzanowski's <maniek@beer.com> gnuhtml2latex program
% Version : 0.1
\documentclass{article}
\begin{document}



  
  
    
      
      
      
    
  



\\
\\

\section*{Chapter29}


[\textit{Previous Chapter}]
[\textit{Contents}]
[\textit{Next Chapter}]


\section*{Differences and Commonalities}



In
Chapter 25
we introduced the problem of similarity
via the Humdrum
\textbf{simil}
and
\textbf{correl}
commands.
This chapter revisits the problem of similarity by focussing
on differences and commonalities.
Specifically, this chapter introduces three additional tools, the UNIX
\textbf{cmp},
\textbf{diff}
and
\textbf{comm}
commands.
Although these commands are less sophisticated than the
\textbf{simil}
and
\textbf{correl}
commands, they nevertheless provide convenient tools for
quickly determining the relationship between two or more inputs.


\subsection*{Comparing Files Using \textit{cmp}}

\par 
The
\textbf{cmp}
command does a character-by-character comparison and indicates
whether or not two files are identical.

\par 

\texttt{cmp file1 file2}


\par 
If the two files differ,
\textbf{cmp}
generates a message indicating the first point where the
two files differ.
E.g.,

\par 

\texttt{file1 file2 differ: char 4, line 10}


\par 
If the two files are identical,
\textbf{cmp}
simply outputs nothing ("silence is golden").

\par 
Sometimes files differ in ways that may be uninteresting.
For example, we may suspect that a single work has been
attributed to two different composers.
The encoded files may differ only in that the \texttt{!!!COM:} reference
records are different.
We can pre-process the files using
\textbf{rid}
in order to determine whether the scores are otherwise identical.

\par 

\texttt{rid -G file1 > temp1}
\\
\texttt{rid -G file2 > temp2}
\\
\texttt{cmp temp1 temp2}


\par 
Of course one of the works might be transposed with respect to
the other.
We can circumvent this problem by translating the data to some
key-independent representation such as \texttt{solfa} or \texttt{deg}:

\par 

\texttt{rid -GL file1 | solfa > temp1}
\\
\texttt{rid -GL file2 | solfa > temp2}
\\
\texttt{cmp temp1 temp2}



\par 
Two songs might have different melodies but employ the same lyrics.
We can test whether they are identical by extracting and comparing
any text-related spines.
Since there may be differences due to melismas,
we might also use
\textbf{rid -d}
to eliminate null data records.

\par 

\texttt{extract -i '**silbe' file 1 | rid -GLd > temp1}
\\
\texttt{extract -i '**silbe' file 2 | rid -GLd > temp2}
\\
\texttt{cmp temp1 temp2}



\par 
Similarly, two works might have identical harmonies:

\par 

\texttt{extract -i '**harm' file 1 | rid -GLd > temp1}
\\
\texttt{extract -i '**harm' file 2 | rid -GLd > temp2}
\\
\texttt{cmp temp1 temp2}


\par 
By further reducing the inputs, we can focus on quite
specific elements, such as whether two songs have the same rhythm.
In the following script, we first eliminate bar numbers,
and then eliminate all data except for durations and barlines.

\par 

\texttt{extract -i '**kern' file 1 | humsed '/=/s/[0-9]//; $\backslash$

s/[\^{}0-9.=]//g' | rid -GLd > temp1}

\\
\texttt{extract -i '**kern' file 1 | humsed '/=/s/[0-9]//; $\backslash$

s/[\^{}0-9.=]//g' | rid -GLd > temp2}

\\
\texttt{cmp temp1 temp2}


\par 
For some tasks, we might focus on just a handful of records.
For example, we might ask whether two works have
the same changes of key.

\par 

\texttt{grep '\^{}*[a-gA-G][\#-]*:'  file 1 > temp1}
\\
\texttt{grep '\^{}*[a-gA-G][\#-]*:'  file 2 > temp2}
\\
\texttt{cmp temp1 temp2}



\par 
In the extreme case, we might compare just
a single line of information.
For example, we might identify whether two works have
identical instrumentation:

\par 

\texttt{grep '\^{}!!!AIN:'  file 1 > temp1}
\\
\texttt{grep '\^{}!!!AIN:'  file 2 > temp2}
\\
\texttt{cmp temp1 temp2}



\subsection*{Comparing Files Using \textit{diff}}

\par 
The problem with
\textbf{cmp}
is that it is unable to distinguish whether the difference
between two files is profound or superficial.
A useful alternative to the
\textbf{cmp}
command is the UNIX
\textbf{diff}
command.
The
\textbf{diff}
command attempts to determine the minimum set of changes
needed to convert one file to another file.
The output from
\textbf{diff}
entails editing commands reminiscent of the
\textbf{ed}
text editor.
For example, two latin texts that differ at line 40,
might generate the following output:

\par 

\texttt{40c40
\\
< es quiambulas
\\
-$\,$-$\,$-
\\
> es quisedes}



\par 
Let's consider again the question of whether
two works have essentially the same lyrics.
Many otherwise similar texts might differ in trivial ways.
For example, texts may differ in punctuation or in
the use of upper- and lower-case characters.
The
\textbf{diff}
command provides a
\textbf{-i}
option that ignores
distinctions between upper- and lower-case characters.
Punctuation marks can be eliminated by adding a suitable
\textbf{humsed}
filter.

\par 

\texttt{extract -i '**silbe' file1 | text | humsed 's/[\^{}a-zA-Z ]//g' $\backslash$

| rid -GLId > temp1

extract -i '**silbe' file2 | text | humsed 's/[\^{}a-zA-Z ]//g' $\backslash$

| rid -GLId > temp2

diff -i file1 file2}


\par 
Every time
\textbf{diff}
encounters a difference between the two files, it will output
several lines identify the location of the difference and showing
the conflicting lines in the two files.
The
\textbf{diff}
command is line-oriented.
Two lines need only differ by a single character in order for
\textbf{diff}
to generate an output.

\par 
When there are more than a dozen or so differences, the output becomes
cumbersome to read.
A useful alternative is to avoid looking at the raw output from
\textbf{diff};
instead, we might simply count the number of lines of output (using \textbf{wc -l}).
When compared with the total length of the input,
the number of output lines can provide a rough estimate of the magnitude of
the differences between the two files.
A suitable revision to the last line of the above script would be:

\par 

\texttt{diff -i file1 file2 | wc -l}


\par 
One problem with this approach is that it
assumes that we know which two files we want to compare.
A more common problem is looking for
\textit{any}
work that is somewhat similar to some given work.
We can automate this task by embedding the above script
in a loop so that the comparison (second) file cycles through a series
of possibilities.
A simple
\textbf{while}
loop will enable us to do this.
Since our script may process a large number of scores,
we ought to format our output for ease of reading.
The
\textbf{echo}
command in our script outputs each filename in turn with the
a count of the number of output lines generated by \textbf{diff}.

\par 

\\
\texttt{extract -i '**silbe' \$1 | text | humsed 's/[\^{}a-zA-Z ]//g' $\backslash$
\\

\\
| rid -GLId > temp1
\\

\\
shift
\\
while [ \$\# -ne 0 ]
\\
do
\\

\\
extract -i '**silbe' \$1 | text | humsed 's/[\^{}a-zA-Z ]//g' $\backslash$
\\

\\
| rid -GLId > temp2
\\

\\
CHANGES=`diff -i temp1 temp2 | wc -l`
\\
echo \$1 ":	" \$CHANGES
\\
shift
\\

\\
done
\\
rm temp[12]}


\par 
Of course this same approach may be applied to other musical aspects
apart from musical texts.
For example, with suitable changes in the processing, one could
identify works that have similar rhythms, melodic contours,
harmonies, rhyme schemes, and so on.


\subsection*{Comparing Inventories -$\,$- The \textit{comm} Command}

\par 
The
\textbf{diff}
command is sensitive to the order of data.
Suppose that texts for two songs differ only in that one song
reverses the order of verses 3 and 4.
Comparing the "wrong" verses will tend to exaggerate what
are really minor differences between the two songs.
In addition, the above approach is too sensitive to word or
phrase repetition.
Many works -$\,$- especially polyphonic vocal works -$\,$- 
use extensive repetitions
(e.g., "on the bank, on the bank, on the bank of the river").
Short texts (such as for the \textit{Kyrie} of the Latin mass) are especially
prone to use highly distinctive repetition.
How can we tell whether one work has pretty much the same lyrics
as another?


\par 
Fortunately, most texts tend to have unique word inventories.
Although words may be repeated or re-ordered, phrases interrupted,
and verses re-arranged, the basic vocabulary for similar texts are
often much the same.
A useful technique is to focus on the similarity of the word
inventories.
In the following script, we simply create a list of words
used in both the original and comparison files.

\par 

\texttt{extract -i '**silbe' file1 | text | humsed 's/[.,;:!?]//g' $\backslash$

| rid -GLId | tr A-Z a-z | sort -d > inventory1}

\texttt{extract -i '**silbe' file2 | humsed 's/[.,;:!?]//g' | tr A-Z a-z | text $\backslash$

| rid -GLId | sort | uniq -c | sort -nr > inventory2}



\par 
Suppose that our two vocabulary inventories appear as follows:

\textbf{Inventory 1:}\textbf{Inventory 2:}

dominea
etcoronasti
eumdomine
filioet
gloriaeum
infilio
jerusalemgloria
orieturhonore
patrimanuum
sanctooper
spirituipatri
supersancto
tespiritui
videbitursuper
tuarum


\par 
Notice that a number of words are present in both texts,
such as \textit{domine}, \textit{et}, \textit{eum}, \textit{filio}, and so on.
Identifying the common vocabulary items is easily done by the UNIX
\textbf{comm}
command;
\textbf{comm}
compares two sorted files and identifies which lines are
shared in common and which lines are unique to one file or the other.

\par 
The
\textbf{comm}
command outputs three columns:
the first column identifies only those lines that are present in
the first file,
the second column identifies only those lines that are present in
the second file,
and the third column identifies those lines that are present in
both files.
In the case of our two Latin texts, the command:

\par 

\texttt{comm inventory1 inventory2}


\par 
will produce the following output.
The first and second columns identify words unique to \texttt{inventory1}
and \texttt{inventory1}, respectively.
The third column identifies the common lines:

\texttt{a
coronasti
domine
et
eum
filio
gloria
honore
in
jerusalem
manuum
oper
orietur
patri
sancto
spiritui
super
te
tuarum
videbitur}


\par 
In the above case, five words are unique to \texttt{inventory1},
six words are unique to \texttt{inventory2} and nine words
are common to both.

\par 
The
\textbf{comm}
command provides numbered options that suppress specified columns.
For example, the command
\textbf{comm -13}
will suppress columns one and three (outputing column two).
(Empty lines are also suppressed with these options.)
A convenient measure of similarity is to express the shared vocabulary items
as a percentage of the total combined vocabularies.
We can do this using the word-count command, \textbf{wc}.
The first command counts the total number of words and the second
command counts the total number of shared words:

\par 

\texttt{comm inventory1 inventory2 | wc -l}
\\
\texttt{comm -3 inventory1 inventory2 | wc -l}


\par 
An important point about
\textbf{comm}
is that the order of materials is important in the input files.
If the word \textit{filio} occurs near the beginning of \texttt{inventory1} but
near the end of \texttt{inventory2} then
\textbf{comm}
will not consider the record common to both files.
This is the reason why we used an alphabetical sort (\textbf{sort -d})
in our original processing.

\par 
On the other hand, there are sometimes good reasons to order
the vocabulary lists non-alphabetically.
For example, suppose we created our inventories according to
the frequency of occurrence of the words.
That is, suppose we use
\textbf{uniq -c | sort -nr}
to generate a vocabulary list ordered by how common each word is.
Our inventory files might now appear as follows:

\par 

\textbf{Inventory 1:}

\texttt{3et
2te
2gloria
1videbitur
1super
1spiritui
1sancto
1patri
1orietur
1jerusalem
1in
1filio
1eum
1domine}


\par 
\textbf{Inventory 2:}

\texttt{4et
2gloria
2eum
1tuarum
1super
1spiritui
1sancto
1patri
1oper
1manuum
1honore
1filio
1domine
1coronasti
1a}



\par 
Comparing these two inventories will produce little in common
due to the presence of the numbers.
For example, the records "\texttt{3    et}" and "\texttt{4    et}"
will be deemed entirely different.
However, we can eliminate the numbers using an appropriate
\textbf{sed}
command leaving us with vocabulary lists that are ordered according to the
frequency of occurrence of the words.
If we apply the
\textbf{comm}
command to these lists then the commonality measures
will be sensitive to the relative frequency of words
within the vocabularies.



\subsection*{Reprise}

\par 
In this chapter we have introduced the UNIX
\textbf{cmp}, \textbf{diff} and
\textbf{comm}
commands.
The
\textbf{cmp}
command determines whether two files as are the same or different.
The
\textbf{diff}
command identifies how two files differ.
The
\textbf{comm}
command identifies which (sorted) lines two files share in common;
\textbf{comm}
also allows us to identify which lines are unique
to just one of the files.

\par 
The value of these tools is amplified when the inputs
are pre-processed to eliminate unwanted or distracting data,
and when post-processing is done (using \textbf{wc})
to estimate the magnitude of the differences or commonalities.

\par 
Together with the
\textbf{simil}
and
\textbf{correl}
commands discussed in
Chapter 25,
these five tools provide
a variety of means for characterizing differences,
commonalities, and similarities.

\\
\begin{itemize}
\item 

\textbf{Next Chapter}
\item 

\textbf{Previous Chapter}
\item 

\textbf{Table of Contents}
\item 

\textbf{Detailed Contents}
\\\\

� Copyright 1999 David Huron
\end{document}
