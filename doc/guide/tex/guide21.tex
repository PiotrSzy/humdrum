% This file was converted from HTML to LaTeX with
% Tomasz Wegrzanowski's <maniek@beer.com> gnuhtml2latex program
% Version : 0.1
\documentclass{article}
\begin{document}



  
  
    
      
      
      
    
  



\\
\\

\section*{Chapter21}


[\textit{Previous Chapter}]
[\textit{Contents}]
[\textit{Next Chapter}]


\section*{Searching for Patterns}



The
\textbf{grep}
and
\textbf{egrep}
commands are useful for identifying patterns that occur on
single lines.
As we saw in
Chapter 19,
the
\textbf{context}
command can be used to amalgamate groups of successive data tokens on a
single line -$\,$- and so facilitate searching for sequential patterns using
\textbf{grep}
or \textbf{egrep}.
For many tasks, the combination of
\textbf{context}
and
\textbf{grep}
provides the most convenient way to search for user-specified patterns.
However, not all patterns can be conveniently identified
using this approach.
In this chapter we will introduce two additional tools that are intended
to search directly for sequential patterns without having to use
\textbf{context}
to create pseudo-simultaneous collections.


\subsection*{The \textit{patt} Command}

\par 
The
\textbf{patt}
command may be regarded as a two-dimensional version of \textbf{grep}.
Like \textbf{grep}, \textbf{patt} searches for lines that match user-specified
regular expressions.
However, unlike \textbf{grep}, \textbf{patt} can search for a sequence of records
that match a sequence of user-specified regular expressions.
Specifically, \textbf{patt} will look for an input line that matches the
first (of potential many) user-specified regular expression.
Then \textbf{patt} will determine whether the following input line matches
the second user-specified regular expression ... and so on,
until the entire sequence of the user-specified regular expressions are exhausted.
A pattern match is deemed to occur only if all of the successive
regular expressions match a contiguous sequence of input lines.


\par 
The operation of
\textbf{patt}
is easier to describe through an example.
Consider the following input using the German
\textit{Tonh�he} pitch designations described in
Chapter 4.
Recall that the
\texttt{**Tonh}
system of pitch names allows Bach to spell his name
(B=B-flat; H=B-natural).
Less well-known is the fact that Dmitri Shostakovich also used the
German pitch system to create motives based on his name:
D-S-C-H (S=Es=E-flat).
(The German transliteration of the cyrillic is Schostakowitsch.)

\texttt{**Tonh
\texttt{D4
\texttt{Es4
\texttt{C4
\texttt{H3
\texttt{*-}

Suppose we were looking for possible instances of D-S-C-H.
The
\textbf{patt}
command requires a template file that contains one or more
successive regular expressions.
A suitable template file (named \texttt{dmitri}) would be as follows:

\texttt{D
\texttt{Es
\texttt{C
\texttt{H}


\par 
We would invoke the search as follows:

\par 

\texttt{patt -f dmitri} \textit{inputfile}


\par 
The
\textbf{-f}
option is mandatory:
it conveys to
\textbf{patt}
the name of the template file used in the search.

\par 
In the default operation,
\textbf{patt}
simply outputs a global comment identifying the location of any
matching segments.
One global comment is output for each matching pattern.
In the above case, the output would be as follows:

\par 

\texttt{!! Pattern found at line 2 of file Tonh}


\par 
The
\textbf{patt}
command will also identify any overlapping patterns.
For example, suppose we had an input containing an
ostinato figure in minor thirds:

\par 
\textbf{Example 21.1.}  `DSCH' Ostinato.



\texttt{**Tonh**Tonh
\texttt{*k[b-e-]*K[b-e-]
\texttt{*M9/8*M9/8
\texttt{=1-=1-
\texttt{C4Es4
\texttt{C4Es4
\texttt{H3D4
\texttt{C4Es4
\texttt{C4Es4
\texttt{H3D4
\texttt{C4Es4
\texttt{C4Es4
\texttt{H3D4
\texttt{==}



\par 
If we applied the above
\textbf{patt}
command to this \texttt{ostinato} file, we would get the following output:

\par 

\texttt{!! Pattern found at line 8 of file ostinato}
\\
\texttt{!! Pattern found at line 11 of file ostinato}


\par 
We can instruct
\textbf{patt}
to output specific instances of the pattern using the
\textbf{-e}
(echo) option.
Consider the following command:

\par 

\texttt{patt -f dmitri -e ostinato}


\par 
The resulting output would be:

\texttt{!! Pattern found at line 4 of file ostinato


\texttt{**Tonh\texttt{**Tonh
\texttt{H3\texttt{D4
\texttt{C4\texttt{Es4
\texttt{C4\texttt{Es4
\texttt{H3\texttt{D4
\texttt{*-\texttt{*-


\texttt{!! Pattern found at line 7 of file ostinato


\texttt{**Tonh\texttt{**Tonh
\texttt{H3\texttt{D4
\texttt{C4\texttt{Es4
\texttt{C4\texttt{Es4
\texttt{H3\texttt{D4
\texttt{*-\texttt{*-}


\par 
Notice that each instance of the found pattern is output as a stand-alone
humdrum "mini-encoding," complete with initial exclusive interpretations
and terminating spine-path terminators.
\\\\
\textbf{Example 21.2.}  J.S. Bach, \textit{Well-Tempered Clavier}, Vol. 1, Fugue 2.





\par 
Most Baroque composers were fond of ending works written in minor keys
on the tonic major chord -$\,$- the so-called \textit{tierce de picardie}
or Picardy Third.
Example 21.2 shows a typical example from the final measures of Bach's
second fugue from the \textit{Well-Tempered Clavier}, vol. 1.
Suppose that we wanted to identify all works in some repertory
that end with a \textit{tierce de picardie}.
We need to search for a raised third scale degree in close
proximity to the end of a work for those works in a minor key.
First we might identify those works in minor keys.
The following
\textbf{grep}
command will search all files in the current directory
for a tandem interpretation indicating a minor key.
Recall that minor keys are identified by an asterisk followed by
a lower-case pitch-letter name, followed by an optional
accidental, followed by the colon character.
The
\textbf{-l}
option will list all files that containing a matching record:

\par 

\texttt{grep -l '\^{}$\backslash$*[a-g][-\#]*:' *}


\par 
Recall that the
\textbf{deg}
command is mode sensitive, whereas the
\textbf{solfa}
command is mode insensitive.
That is, in the key of C major,
\textbf{deg}
will represent the pitch E as \texttt{3} and in C minor
\textbf{deg}
will represent the pitch E (natural) as \texttt{3+}.
By contrast, the
\textbf{solfa}
command will represent E as `\texttt{mi}' whether
the mode is major or minor.

\par 
In order to locate picardy thirds, we can look
for raised mediants in the
\texttt{**deg}
representation.
Specifically, we can look for a raised mediant pitch
immediately prior to a double barline.
Our template file (dubbed "\texttt{picardy}") might look as follows:

\par 

\texttt{3[+]
\\
==}


\par 
Notice that the plus sign has been placed in square brackets.
The
\textbf{patt}
command accepts only
\textit{extended}
regular expressions.
The plus sign is a metacharacter that normally indicates
"one or more instances."
So placing it in square brackets causes the special meaning
to be escaped.

\par 
In order to search for such picardy thirds, we should translate each input file
to the \texttt{**deg} representation, and then search for
raised mediants immediately prior to a double bar:

\par 

\texttt{deg \textit{inputfile.krn} | patt -f picardy}


\par 
A problem with this search strategy is that it assumes that the raised
third will occur in the final sonority prior to the double barline.
One possible confound might be the presence of one or more
rests following the final chord.
This situation is evident in Fugue No. 4 from the second volume of Bach's
\textit{Well-Tempered Clavier}:

\par 
\textbf{Example 21.3.}  J.S. Bach, \textit{Well-Tempered Clavier}, Vol. 2, Fugue 4.



\texttt{!!!COM: Bach, Johann Sebastian
\texttt{!!!XEN: The Well-Tempered Clavier, Volume 2, Fugue 4.


\texttt{**kern**kern**kern
\texttt{*clefF4*clefG2*clefG2
\texttt{*M12/16*M12/16*M12/16
\texttt{*k[f\#c\#g\#d\#]*k[f\#c\#g\#d\#]*k[f\#c\#g\#d\#]
\texttt{*c\#:*c\#:*c\#:
\texttt{=70=70=70
\texttt{16E8.f\#]8b\#
\texttt{16D\#..
\texttt{16E.16g\#
\texttt{16FF\#\#8e4.cc\#
\texttt{16GG\#..
\texttt{16AAn16d\#.
\texttt{4.GG\#16e.
\texttt{.16a.
\texttt{.16g\#.
\texttt{.16f\#8.b\#
\texttt{.16e.
\texttt{.16d\#.
\texttt{=71=71=71
\texttt{8.CC\#8.e\#8.cc\#
\texttt{8.r8.r8.r
\texttt{4.r4.r4.r
\texttt{======
\texttt{*-*-*-}



\par 
The
\textbf{patt}
command provides a
\textbf{-s}
option that allows the user to skip or ignore certain records in the input.
Any regular expression can be given as a parameter for the
\textbf{-s}
option.
In the following pipeline, we have instruction
\textbf{patt}
to skip over any records matching the lower-case letter `\texttt{r}'
(the \texttt{**kern} rest signifier):

\par 

\texttt{deg }\textit{inputfile.krn}\texttt{ | patt -s r -f picardy}


\par 
Even ignoring rests may not be sufficient to identify the
raised third near the double barline.
For example, if any other note from the tonic chord follows
after the raised third, then the third will appear several
records prior to the double barline.
We can solve this problem by using the
\textbf{ditto}
command discussed in
Chapter 15;
\textbf{ditto}
can be used to propagate the raised third through the
sustained final chord.
Our revised pipeline is:

\par 

\texttt{deg bach.krn | ditto -s = | patt -s r -f picardy}



\par 
A similar approach can be used to identify consecutive fifths or
octaves between two voices.
A template file (dubbed \texttt{5ths}) might consist of the following pattern:

\par 

\texttt{P5
\\
P5}


\par 
In order to identify consecutive fifths, we might extract two
parts of interest, and then translate to the
\texttt{**hint}
harmonic-interval representation.
The
\textbf{-c}
option for
\textbf{hint}
collapses compound intervals to their non-compound equivalents so
consecutive twelfths, nineteenths, etc. will also be identified.
In the following command pipeline, notice the use of the
\textbf{-s}
option for
\textbf{patt}
in order to skip barlines.
This ensures that crossing a barline does not result in a failure to
identify a consecutive fifth.

\par 

\texttt{extract -i '*Ibass,*Itenor' Fux | hint -c | patt -s = -f 5ths}



\par 
Sometimes patterns will tend to be obscured by the presence of other information.
For example, suppose we want to identify possible Landini cadences
such as the cadence shown in Example 21.4.
Landini cadences are common in much 14th century polyphony including
works by Machaut, Caserta, Dufay, Ciconia, as well as Landini.
One characteristic of the Landini cadence is the distinctive three-note
\textit{ti -> la -> do} in the upper-most part.
The submediant pitch is interposed between the leading-tone and the tonic.
A second characteristic of the Landini cadence is the harmonic relationship
between the highest and lowest voices.
Three intervals are formed: \textit{sixth -> fifth -> octave}.
Either one or both of these characteristics might be used to help
identify this distinctive cadential formula.
\\\\
\textbf{Example 21.4.}  Francesco Landini, Excerpt from \textit{Non avr� ma' piet�}.



Below is a
\texttt{**kern}
encoding of the final two measures along with corresponding
\texttt{**hint}
and
\texttt{**deg}
spines.
The **hint spine was generated using
\textbf{hint}
\textbf{-l}
in order to generate intervals with respect to the lowest pitch.

\texttt{!!!COM: Landini, Francesco


\texttt{**kern\texttt{**kern\texttt{**kern\texttt{**hint\texttt{**deg**deg**deg
\texttt{*clefF4\texttt{*clefG2\texttt{*clefG2\texttt{*\texttt{*\texttt{*\texttt{*
\texttt{*M3/4\texttt{*M3/4\texttt{*M3/4\texttt{*M3/4\texttt{*M3/4\texttt{*M3/4\texttt{*M3/4
\texttt{=\texttt{=\texttt{=\texttt{=\texttt{=\texttt{=\texttt{=
\texttt{4A\texttt{4e\texttt{8e\texttt{P5 P5\texttt{v2\texttt{v6\texttt{\^{}6
\texttt{.\texttt{.\texttt{8f\texttt{-..\texttt{\^{}7-
\texttt{4B-\texttt{4d\texttt{8g\texttt{M3 M6\texttt{\^{}3-\texttt{v5\^{}1
\texttt{.\texttt{.\texttt{4f\#\texttt{-..\texttt{v7
\texttt{4A\texttt{4c\#\texttt{.\texttt{M3\texttt{v2\texttt{v4+.
\texttt{.\texttt{.\texttt{8e\texttt{-..\texttt{v6
\texttt{=\texttt{=\texttt{=\texttt{=\texttt{=\texttt{=\texttt{=
\texttt{2.G\texttt{2.d\texttt{2.g\texttt{P5 P8\texttt{v1\texttt{\^{}5\texttt{\^{}1
\texttt{==\texttt{==\texttt{==\texttt{========
\texttt{*-\texttt{*-\texttt{*-\texttt{*-*-*-*-}


\par 
Notice that
\textbf{hint}
has failed to generate the passing interval forming the perfect fifth between
the E and the A.
This can be remedied by using
\textbf{ditto}
to duplicate all of the pitches.
This will cause
\textbf{hint}
to generate all of the passing harmonic intervals.
The revised  **hint spine is given below.

\texttt{!!!COM: Landini, Francesco


\texttt{**kern**kern**kern**hint**deg**deg**deg
\texttt{*clefF4*clefG2*clefG2****
\texttt{*M3/4*M3/4*M3/4*M3/4*M3/4*M3/4*M3/4
\texttt{=======
\texttt{4A4e8eP5 P5v2v6\^{}6
\texttt{..8fP5 m6..\^{}7-
\texttt{4B-4d8gM3 M6\^{}3-v5\^{}1
\texttt{..4f\#M3 A5..v7
\texttt{4A4c\#.M3 M6v2v4+.
\texttt{..8eM3 P5..v6
\texttt{=======
\texttt{2.G2.d2.gP5 P8v1\^{}5\^{}1
\texttt{==============
\texttt{*-*-*-*-*-*-*-}


\par 
One way to identify Landini cadences is to use the following harmonic-interval template
file (dubbed \texttt{LandCadence}):

\texttt{6
\texttt{5
\texttt{8}


\par 
Using this template, we can identify Landini cadences as follows.
(Notice the use of \textbf{-s \^{}=} to skip barlines.)

\par 

\texttt{ditto -s \^{}= input | hint -l | patt -s \^{}= -f LandCadence}


\par 
It is possible that the 6-5-8 figured bass might arise in non-cadential
situations, so a more circumspect template might also include some
scale-degree movements as well.
The following template file (dubbed \texttt{Landini-Cadence}) combines both the
harmonic-interval and scale-degree data:

\par 

\texttt{[Mm]6
\\
P5.*v6
\\
P8.*$\backslash$\^{}1}


\par 
Using this more sophisticated pattern template,
a suitable sequence of commands would be the following:

\par 

\texttt{ditto -s \^{}= inputfile | hint -l > temp1
\\
deg inputfile > temp2
\\
assemble temp1 temp2 | patt -s \^{}= -f Landini-Cadence}


\par 
In general,
\textbf{patt}
templates can be used to specify both concurrent conditions
as well as dynamic or temporal conditions.
This allows users to define patterns involving a multitude of conditions
involving many different types of data.


\subsection*{Using \textit{patt}'s Tag Option}

\par 
So far, we have seen that
\textbf{patt}
provides two kinds of output.
In the default operation,
\textbf{patt}
outputs a simple global comment each time it finds a matching
segment in the input.
With the
\textbf{-e}
option,
\textbf{patt}
will also echo the specific passage(s) found.
In addition,
\textbf{patt}
provides a third type of output using the
\textbf{-t}
option.

\par 
When the
\textbf{-t}
option is invoked,
\textbf{patt}
will output the original input, plus an addition \texttt{**patt} spine.
The \texttt{**patt} spine typically consists of mostly null tokens.
However, each time the input matches the sought pattern, a user-defined
"tag" will appear in the \texttt{**patt} spine.
Consider the following example.


\par 
Suppose we are interested in identifying deceptive cadences in
Bach's chorale harmonizations.
Imagine that we already have a
\texttt{**harm}
spine containing
a Roman numeral harmonic analysis.
There are different ways of defining a deceptive cadence,
but a frequent definition is that it involves a dominant
chord followed by a submediant chord in a cadential context.
In the case of Bach's chorale harmonizations, cadences
are readily identified by the pause symbol.
Our search template might look as follows:

\par 

\texttt{\^{}V([\^{}I]|\$)
\\
(vi)|(VI);}


\par 
This template means: "look for an upper-case letter \texttt{V}
appearing at the beginning of a line that is followed
by either the end of the line (\texttt{\$}) or by a character
other than the upper-case letter \texttt{I}.
This record will be followed by a record containing
either \texttt{vi} or \texttt{VI} followed by a semicolon."

\par 
When invoking the
\textbf{patt}
command, we can specify our preferred output tag along with the
\textbf{-t}
option as follows:

\par 

\texttt{extract -i '**harm' bwv269.krn | patt -f template -t deceptive}


\texttt{**harm**patt
\texttt{I.
\texttt{I.
\texttt{ii7.
\texttt{Vdeceptive
\texttt{vi;.
\texttt{V.
\texttt{I.
\texttt{IV.
\texttt{IV.
\texttt{I.
\texttt{V;.}
etc.


\par 

In
Chapter 26
we will learn how to collapse several
spines into a single spine.
This will allow us to assemble the results from several "passes" using
\textbf{patt}
-$\,$-  one pass for each type of cadence.
For example, we could collapse several tagged outputs to produce a single
spine that identifies all of the various types of cadences:

\texttt{**harm**cadences
I.
I.
ii7.
Vdeceptive
vi;.
V.
I.
IV.
IV.
Ihalf
V;.}
etc.


\par 
There are no restrictions as to the types of tags that can be generated by
\textbf{patt}.
A user might tag the beginning of motivic or thematic statements,
various harmonic progressions, variation techniques,
fingering patterns, quotations or allusions, stylistic clich�s,
etc.
In
Chapter 35
we will use the
\textbf{-t}
option to label different set forms for statements of a twelve-tone row,
such as primes, inversions, retrogrades, and retrograde inversions.
We will use suitable tags to identify the specific transpositions:
P0, I7, R11, RI8, etc.


\subsection*{Matching Multiple Records Using the \textit{patt} Command}

\par 
Twelve-tone music raises several special issues for sequential pattern matching.
For example, it is common in serial music to collapse segments of a tone-row in order
to create vertical chords.
Consider the following excerpt from Ernst Krenek's suite for solo 'cello.
The tone row consists of the ordered pitches:
D, G-flat, F, D-flat, C, B, E-flat, A, B-flat, A-flat, E, G.
\\\\
\textbf{Example 21.5.}  Ernst Krenek, Opus 84 \textit{Suite for Violoncello}; mov. 1, measures 28-30.



Using a pitch-class representation we would search for the sequence:

\texttt{2
6
5
1
0
11
3
9
10
8
4
7}


\par 
Due to the diads, however, the corresponding pitch-class representation
for the above Krenek passage would be:

\texttt{**pc
2
6
5 1
0
=
3 11
9
10
10 8
=
7 4}
etc.
\texttt{*-}


\par 
The 
\textbf{-m}
option for
\textbf{patt}
invokes a "multi-record matching" mode.
In this mode,
\textbf{patt}
attempts to match as many successive regular expressions in the template
file as possible for a given input record, before continuing with
the next input and template records.
In this way, several records in the template file 
may possibly match a single input record.
In the above case, the tone-row template will be matched and the `P0' tag
issued if the following command is issued:

\par 

\texttt{patt -f tonerow -t P0 -m Krenek}



\subsection*{The \textit{pattern} Command}

\par 
Not all patterns can be identified using \textbf{patt}.
The Humdrum
\textbf{pattern}
command permits an additional regular expression feature that is
especially useful in musical applications.
Specifically,
\textbf{pattern}
permits the defining of patterns spanning more than one line or record.
Record-repetition operators are specified by following the
regular expression with a tab -$\,$- followed by either \texttt{+}, \texttt{*},
or \texttt{?}.
For example, consider the following Humdrum-extension regular expression:

\texttt{X+
Y*
Z?}


\par 
When the metacharacters \texttt{+}, \texttt{*}, or \texttt{?} appear at the end
of a record, preceded by a tab character, they pertain to the number of records,
rather than the number of repetitions of the expression within a record.
The first record of the regular expression (\texttt{X}\textit{}\texttt{+}) will
match one or more successive lines each containing the letter `\texttt{X}'.
The second record of the regular expression (\texttt{Y}\textit{}\texttt{*})
will match zero or more subsequent lines containing the letter `\texttt{Y}'.
The third record of the regular expression (\texttt{Z}\textit{}\texttt{?}) will
match zero or one line containing the letter `\texttt{Z}'.
Hence, the above multi-record regular expression would match an
input such as the following:
three successive lines containing the letter `\texttt{X}', followed by eight
successive lines containing the letter `\texttt{Y}', followed by a single line
containing the letter `\texttt{Z}'.
Similarly, the above regular expression would match an input containing
one line containing the letter `\texttt{X}'.

\par 
Record-repetition operators can be used in conjunction with all
of the other regular expression features.
For example, the following regular expression matches one or more
successive
\texttt{**kern}
data records containing the pitch `\texttt{G}'
(naturals only) followed optionally by a single `\texttt{G\#}' followed by
one or more records containing one or more pitches from an A major
triad -$\,$- the last of which must end a phrase:

\par 
\texttt{

[Gg]+[\^{}\#-]	+
\\
[Gg]+\#[\^{}\#]	?
\\
([Aa]+|([Cc]+\#)|[Ee]+)[\^{}\#-]	*
\\
(\}.*([Aa]+|([Cc]+\#)|[Ee]+)[\^{}\#-]))|(([Aa]+|([Cc]+\#)|[Ee]+)[\^{}\#-].*\})

}


\subsection*{Patterns of Patterns}

\par 
Music often exhibits hierarchical structures where particular types of
patterns may be embedded in other patterns, or where low-level patterns
join together to form higher-level patterns.
As we have seen, the
\textbf{-t}
(tag) option for the
\textbf{patt}
command allows a new output spine to be generated.
This spine contains user-defined labels marking the beginning of each found pattern.
The labels can contain any user-defined text string such as
\texttt{authentic cadence},
\texttt{episode},
\texttt{Motive 3b},
\texttt{augmentation},
\texttt{triplet figuration},
or
\texttt{prolongation}.

\par 
As we will see in
Chapter 26,
the contents of several spines can be amalgamated
to form a single spine.
This means that the results for several independent pattern searches can be
assembled into a single "pattern" spine.
Several pattern spines may be created that related to patterns found at
different hierarchical levels, or patterns found using different search methods.
Of course, these pattern-spines themselves can be used as input to further
pattern searches thus providing unbounded possibilities for searching for
patterns of patterns.

\par 
Consider, for example, the following template for the
\textbf{pattern}
command:

\texttt{Theme 1 (tonic)+
Bridge*
Theme 2 (tonic)+
Coda?}




\par 
The template reads "one or more instances of \texttt{Theme 1 (tonic)},
followed by zero or more instances of \texttt{Bridge}, followed by
one or more instances of \texttt{Theme 2 (tonic)}, followed by zero or one
instance of \texttt{Coda}."
This template might be used by
\textbf{pattern}
to identify a Recapitulation.
Together with outputs from parallel searches for `Exposition' and `Development'
the results of a `Recapitulation' search might similarly be amalgamated and used
as an input for a higher level search for works exhibiting a sonata-allegro structure.



\subsection*{Reprise}

\par 
In this chapter and previous chapters we have identified several search-related
tools, including the UNIX \textbf{grep} and \textbf{egrep} commands as well
as the Humdrum
\textbf{patt}
and
\textbf{pattern}
commands.
Each of these tools has different strengths and weaknesses and it is not always
clear which tool is best for a given task.
When searching, don't forget to consider how
\textbf{context},
\textbf{humsed},
\textbf{rid}
and other tools might facilitate the searching task.
In future chapters will will consider how "similarity" tools such as
\textbf{correl}
and
\textbf{simil}
can contribute to more sophisticated pattern searches.

\\
\begin{itemize}
\item 

\textbf{Next Chapter}
\item 

\textbf{Previous Chapter}
\item 

\textbf{Table of Contents}
\item 

\textbf{Detailed Contents}
\\\\

� Copyright 1999 David Huron
\end{document}
