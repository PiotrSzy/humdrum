% This file was converted from HTML to LaTeX with
% Tomasz Wegrzanowski's <maniek@beer.com> gnuhtml2latex program
% Version : 0.1
\documentclass{article}
\begin{document}



  
  
    
      
      
      
    
  



\\
\\

\section*{Chapter19}


[\textit{Previous Chapter}]
[\textit{Contents}]
[\textit{Next Chapter}]


\section*{Musical Contexts}



Much of what makes an event of interest is the context of the event.
We may be interested in what precedes or follows a note or chord.
We have already seen how the
\textbf{-A}
and
\textbf{-B}
options for
\textbf{grep}
can be used to output `before' and `after' contexts.
In
Chapter 21
we will see how the
\textbf{patt}
and
\textbf{pattern}
commands can provide further flexibility for searching.

\par 
However, in this chapter we introduce the deceptively simple
\textbf{context}
command.


\subsection*{The \textit{context} Command}

\par 
The effect of the
\textbf{context}
command is easier to illustrate than describe.
Consider a file (named \texttt{input}) that contains the numbers
1 through 6 on successive lines.
A null token is interposed between the numbers 2 and 3:
\\\\

\texttt{
**numbers
1
2
.
3
4
5
6
*-}

The command

\par 

\texttt{context -n 3 input}


\par 
will produce the following output:
\\\\

\texttt{
**numbers
1 2 3
2 3 4
.
3 4 5
4 5 6
.
.
*-}

In effect,
\textbf{context}
amalgamates data tokens from successive records and assembles them
as multiple-stops on a single record.
Notice that the number of data records in the output is the same
as in the input:
\textbf{context}
has simply padded the trailing data records with null tokens.
Also notice that individual data tokens can appear more than once.
For example, the number 3 appears at the end of the second line,
in the middle of the third line, and at the beginning of the fourth line.
In the above example only the numbers 1 and 6 appear once.
Finally, notice that null tokens are simply ignored:
the null token in the fourth line of our input also appears in the
fourth line of the output.

\par 
The
\textbf{-n 3}
option tells
\textbf{context}
how many data tokens to amalgamate on each output line.
With the specification \textbf{-n 2},
just two data tokens would be amalgamated on each output line.

\par 
How might
\textbf{context}
be useful?
Suppose we wanted to determine how harmonic octave intervals are approached
in Bach's two-part keyboard \textit{Inventions}.
What harmonic interval tends to precede an octave?
We can use the
\textbf{hint}
command to generate the harmonic intervals for each successive sonority.
To calculate all \textit{passing intervals}, we will preprocess using
\textbf{ditto}:

\par 

\texttt{ditto inventions* | hint}


\par 
Typical outputs might look like this:

\texttt{M3
M6
A4
=12
M6
m7
M3
A4
M6
A4
M6
P4
M6
M7
m9
m10
d12
m10
P11
M9
=13
m10
P4
M9
M10}


\par 
Using
\textbf{context}
with the
\textbf{-n 2}
option will cause pairs of successive intervals to appear in the
data records.
Each data record will consist of a double-stop containing two
harmonic intervals.
We simply need to identify those data records that have \texttt{P8}
as the second token of the double-stop.
In short, we are interested in data records that end with \texttt{P8}.
The dollars-sign can be used in a regular expression to anchor the
pattern to the end of the line.
Hence:

\par 

\texttt{ditto inventions* | hint | context -n 2 -o = | grep ' P8\$'}


\par 
The
\textbf{-o =}
option tells
\textbf{context}
to omit any data tokens matching the equals-sign -$\,$-
that is, to omit barlines from the amalgamated multiple stops.
(The
\textbf{-o}
option accepts any regular expression as a parameter,
so omitted data can be defined in a much more refined manner
than simply specifying an equals-sign.)
The
\textbf{grep}
command grabs all of the lines ending with \texttt{P8}.
We can now create an inventory of harmonic interval pairs and order them
from least common to most common:

\par 

\texttt{ditto inventions* | hint | context -n 2 -o = $\backslash$
\\

| grep ' P8\$' | sort | uniq -c | sort -rn}



\par 
In the case of Bach's fifteen two-part \textit{Inventions} the results
look as follows:

\par 

\\
\texttt{24  m10 P8
\\
24  M10 P8
\\
23  m7 P8
\\
21  M6 P8
\\
19  M9 P8
\\
12  P5 P8
\\
11  m6 P8
\\
 9  P12 P8
\\
 8  m13 P8
\\
 8  - P8}


\par 
In other words, the octave is most commonly approached by contracting
from minor and major tenths rather than expanding from a major sixth interval.


\par 
This same basic process can be used to address a variety of similar problems.
For example, suppose we wanted to determine the most common word following
"gloria" in Gregorian chant texts.
We first extract the
\texttt{**text}
spine, use
\textbf{context}
to create pairs of words, and search in the normal way:

\par 

\texttt{extract -i '**text' chants* | context -n 2 $\backslash$
\\

| grep -i ' gloria\$' | sort | uniq -c | sort -nr}



\par 
A slight change to the regular expression for
\textbf{grep}
will allow us to determine what word typically
\textit{follows}
after the word "gloria."
In this case, we need to anchor the word "gloria"
to the beginning of the line by using the caret (\^{}).

\par 

\texttt{extract -i '**text' chants* | context -n 2 $\backslash$
\\

| grep -i '\^{}gloria ' | sort | uniq -c | sort -nr}



\par 
Suppose we wanted to determine what scale degree most commonly
precedes the dominant pitch in a sample of Czech folksongs.
First we translate the folksongs to the
\texttt{**deg}
representation using
the
\textbf{deg}
command, and then process as above:

\par 

\texttt{deg Czech* | context -n 2 -o = | grep '5 ' | sort $\backslash$
\\

| uniq -c | sort -nr}




\subsection*{Harmonic Progressions}

\par 
The V-I progression is the most common chord progression in Western
tonal music.
After the V-I progression, what is the most common chord progression
in Bach's chorale harmonizations?
We will assume that a Roman numeral
\texttt{**harm}
spine already exists.
First we extract the appropriate spine.
Then we create context records holding pairs of harmony data
(omitting barlines).
Then we eliminate global and local comments, interpretations,
and null data.
We then sort the data records, eliminate duplicates while counting,
and then sort by numerical count in reverse order.

\par 

\texttt{extract -i '**harm' chorales* | context -n 2 -o = $\backslash$
\\

| rid -GLId | sort | uniq -c | sort -nr}



\par 
Of course, there is no need to restrict ourselves to pairs of
successive data tokens (i.e. \textbf{-n 2}) as we have done in the above example.
Given a database of melodies, we can determine the most common
sequence of five melodic intervals as follows:

\par 

\texttt{mint melodies* | context -n 5 -o = | rid -GLId | sort $\backslash$
\\

| uniq -c | sort -nr}




\subsection*{Using \textit{context} with the \textit{-b} and \textit{-e} Options}

\par 
Example 19.1 shows an excerpt from a flute study by Anderson.
Although the work is monophonic,
the work's structure is based on an underlying chord progression
that is realized as a series of arpeggiation figures.
\\\\
\textbf{Example 19.1   Joachim Anderson, Opus 30, No. 24.}



The harmonic structure can be made more explicit by amalgamating
all of the notes in each arpeggio.
There are several possible ways of doing this, but the
slurs are particularly useful delineators.
The
\textbf{-b}
option for
\textbf{context}
allows the user to specify a regular expression that marks the
\textit{beginning}
of each collection of data tokens.
Consider the following command:

\par 

\texttt{context -b '(' Anderson}


\par 
Whenever a data record contains an open parenthesis a new
amalgamation begins.
The appropriate output for measure 1 of Example 19.1 would be:
\\\\

\texttt{**kern
*clefG2
*k[b-]
*d:
*M4/4
=1-
(16dd 16ff 16dd 16a)
.
.
.
(16dd 16gg 16dd 16b-)
.
.
.
(16dd 16ff 16dd 16a)
.
.
.
(16f 16a 16f 16e) =2}

etc.

Notice how the barline for measure 2 has been included in the
fourth group.
(Groups continue until the next open parenthesis is encountered.)
Once again we might eliminate barlines by using the
\textbf{-o}
option.
However, sometimes the barlines prove useful in further processing.

\par 
In the above passage by Anderson, the close of each slur provides a
convenient marker for ending each chord.
We can be more explicit in defining the grouping boundaries by
also including the
\textbf{-e}
option for
\textbf{context}.
This option allows the user to specify a regular expression that marks the
\textit{end}
of each collection of data tokens.
A suitably revised command would be:

\par 

\texttt{context -b '(' -e ')' Anderson}


\par 
The resulting output would begin as follows:
\\\\

\texttt{**kern
*clefG2
*k[b-]
*d:
*M4/4
=1-
(16dd 16ff 16dd 16a)
.
.
.
(16dd 16gg 16dd 16b-)
.
.
.
(16dd 16ff 16dd 16a)
.
.
.
(16f 16a 16f 16e)
.
.
.
=2
(16d 16ff 16dd 16a)}

etc.

We could pipe this output to the
\textbf{ms}
command in order to display the re-arranged passage.
We place the output in a postscript file and use a display tool such as
\textbf{ghostview}
to display the output:

\par 

\texttt{context -b '(' -e ')' Anderson | ms > output.ps}


\textbf{Example 19.2   Arpeggio Amalgamation.}



\par 
Notice that the resulting notation is "ungrammatical"
because the meter signature disagrees with the total duration
for each measure.


\par 
Having reformatted our input data using \textbf{context},
we can continue by translating the data to another representation.
For example, we might use the
\textbf{deg}
command to reformulate each pitch group as scale degrees.
This might allow us to search for particular harmonic patterns
such as (say) an augmented sixth chord:

\par 

\texttt{context -b '(' -e ')' Anderson | deg | grep '6-' | grep '4+' $\backslash$

| grep '1'}



\par 
Any regular expression can be used to identify the beginning and/or
ending of an amalgamated group.
For example, tokens might be grouped by barlines.
Suppose the
\textbf{census}
command tells us that a monophonic work contains sixty-fourth notes.
We might want to know whether the sixty-fourth notes all tend to
happen in one or two measures, or whether they occur throughout
the work.
Just how many measures contain sixty-fourth notes?

\par 

\texttt{context -b = inputfile | rid -GLId | grep -c '64'}



\par 
Similarly, for
\texttt{**kern}
inputs, the following command counts the
number of measures that contain at least one trill:

\par 

\texttt{context -b = inputfile | grep -c '\^{}=.*[Tt]'}



\par 
In \texttt{**kern} representations, the beginnings and endings of
beams are indicated by the letters `\texttt{L}' and `\texttt{J}' respectively.
We might group notes according to the beaming:

\par 

\texttt{context -b L -e J inputfile}



\par 
For example, the following command determines the location of any beams
that cross over phrase boundaries:

\par 

\texttt{context -b L -e J inputfile | grep -n '\}.*\{'}


\par 
As in the case of the
\textbf{-b}
option, the
\textbf{-e}
option can be used by itself.
This option might prove useful, for example, when
collecting all chord functions preceding a cadence.
In Bach's chorale harmonizations, for example,
cadences are conveniently marked by a pause.
In the \texttt{**harm} representation, pauses are indicated by
the semicolon (\texttt{;}).
We can create phrase related harmonic sequences as follows:

\par 

\texttt{context -o = -e ';' input}



\par 
For example, we might count the number of harmonic functions in
each phrase as follows:

\par 

\texttt{context -o = -e ';' input | rid -GLId | awk '\{print \$NF\}'}


\par 
In
Chapter 22
we will learn how to classify data into discrete categories.
Using the
\textbf{recode}
command described in that chapter, we might group notes together
according to changes of melodic direction.
That is, each group of would consist of notes that are all ascending
or all descending in pitch.


\subsection*{Using \textit{context} with \textit{sed} and \textit{humsed}}

\par 
The stream-editors (\textbf{sed} and \textbf{humsed}) are especially
handy companions for \textbf{context}.
Suppose we wanted to identify by measure number those measures
that contain a
\textit{iii-V}
progression.
Given a \texttt{**harm} input, we would first amalgamate all harmony
tokens for each measure.

\par 

\texttt{context -b \^{}= inputfile | grep 'iii V' | sed 's/ .*//; s/=//'}


\par 
Here we have used
\textbf{grep}
to isolate all those records that contain the character sequence \texttt{iii V}.
We have then used
\textbf{sed}
to eliminate all data following the first occurrence of a space.
This will leave only the barline token -$\,$- including the measure number.

\par 
When using
\textbf{grep}
it is common for the output to no longer conform to the Humdrum syntax.
This is the reason why we used \textbf{sed} rather than
\textbf{humsed}
in the above example.)
Remember that we can always use the
\textbf{yank} -m
command to create "grep-like" output that still conforms to
the Humdrum syntax.
If we wanted to maintain the Humdrum syntax, an equivalent to the above
command would be:

\par 

\texttt{context -b \^{}= inputfile | yank -m 'iii V' -r 0 $\backslash$

| humsed 's/ .*//; s/=//'}



\par 
The range option (\textbf{-r}) specifies that we grab the current record (0)
that matches the marker (\texttt{iii V}).
However, we are free to specify any other range.
Consider the following command variation:

\par 

\texttt{context -b \^{}= inputfile | rid -d | yank -m 'iii V' -r 1 $\backslash$

| grep 'ii IV' | humsed 's/ .*//; s/=//'}



\par 
This command identifies all those measures containing a
\textit{ii IV}
progression that have been preceded by a
\textit{iii V}
progression in the previous measure.

\par 
Consider another example.
Suppose we wanted to determine whether the first pitch
in a phrase tends to be lower than the last pitch in a phrase.
As before, we might first amalgamate all notes in each phrase
onto individual data records.
We can use
\textbf{humsed}
to eliminate all notes other than the first and last.
The regular expression \texttt{/ .* /} specifies any sequence
of characters preceded by a space and followed by a space.
Replacing matching strings with a single space will leave
output data records consisting of double-stops.
The first note of the double-stop will be the first note of
the phrase, and the second note of the double-stop will be
the last note of the same phrase:

\par 

\texttt{context -b \{ -e \} file | humsed 's/ .* / /'}


\par 
We can continue processing by piping the output to the
\textbf{semits}
command.
This will leave pairs of numbers representing
the semitone distances from middle C.
We might then isolate the data records by using
\textbf{rid}.

\par 

\texttt{ . . . | semits | rid -GLId | awk '\{print \$2-\$1\}'}


\par 
Finally, we have used the UNIX
\textbf{awk}
utility to carry out some simple numerical processing:
in this case, substracting the first semitone value from the second one.
Phrases that end on a pitch higher than the beginning pitch will
have positive semitone outputs.
Phrases that end on a pitch lower than the beginning pitch will
have negative semitone outputs.

\par 
If we wanted to determine the semitone pitch distance
\textit{between}
phrases, we need only to reverse the begin (\textbf{-b}) and end (\textbf{-e})
criteria.
That is, we will amalgamate the last note of one phrase with
the first note in the subsequent phrase.
The full pipeline would be as follows:

\par 

\texttt{context -b \{ -e \} file | humsed 's/ .* / /' | semits $\backslash$

| rid -GLId | awk '\{print \$2-\$1\}'}




\subsection*{Linking \textit{context} Outputs with Inputs}

\par 
Frequently, we would like to answer context-related questions
that mix different types of data together.
For example, how many ascending major sixth intervals occur in phrases
that end on the dominant?
For this question, we need concurrent access to both melodic interval
data as well as scale degree information.
The solution to such questions typically involves linking different
types of data together using the
\textbf{assemble}
command.
Suppose the first phrase in our input begins as follows:
\\\\

\texttt{**kern
*F:
*M3/4
\{8Bn
8c
=1
4.a
8g
4f
=2
4g
4d
4e
=3
2c\}
*-}

We need to pursue two independent lines of processing.
First we creat a temporary file of scale degree information:

\par 

\texttt{mint inputfile > temp.mnt}


\par 
Then we amalgamate the pitch data according the phrasing information,
and translate the resulting data to the
\texttt{**deg}
representation:

\par 

\texttt{context -b \{ -e \} -o \^{}= inputfile | deg > temp.deg}


\par 
Next we assemble the two temporary files together to form a single
document.

\par 

\texttt{assemble temp.mnt temp.deg}


\par 
The first phrase output will appear as follows:
\\\\

\texttt{**mint**deg
*F:*F:
*M3/4*M3/4
[B]4+ \^{}5 \^{}3 v2 v1 \^{}2 v6 \^{}7 v5
+m2.
=1.
+M6.
-M2.
-M2.
=2.
+M2.
-P4.
+M2.
=3.
-M3.}

etc.

We need to search for the interval of an ascending major sixth (\texttt{+M6})
associated with a phrase ending on the dominant (\texttt{5\$}).
Before using the approprate
\textbf{grep}
command, we need to use
\textbf{ditto}
to propagate the scale degree data over the null data tokens in
the \texttt{**deg} spine;
\textbf{ditto}
will generate the following output:
\\\\

\texttt{**mint**deg
*F:*F:
*M3/4*M3/4
[B]4+ \^{}5 \^{}3 v2 v1 \^{}2 v6 \^{}7 v5
+m24+ \^{}5 \^{}3 v2 v1 \^{}2 v6 \^{}7 v5
=14+ \^{}5 \^{}3 v2 v1 \^{}2 v6 \^{}7 v5
+M64+ \^{}5 \^{}3 v2 v1 \^{}2 v6 \^{}7 v5
-M24+ \^{}5 \^{}3 v2 v1 \^{}2 v6 \^{}7 v5
-M24+ \^{}5 \^{}3 v2 v1 \^{}2 v6 \^{}7 v5
=24+ \^{}5 \^{}3 v2 v1 \^{}2 v6 \^{}7 v5
+M24+ \^{}5 \^{}3 v2 v1 \^{}2 v6 \^{}7 v5
-P44+ \^{}5 \^{}3 v2 v1 \^{}2 v6 \^{}7 v5
+M24+ \^{}5 \^{}3 v2 v1 \^{}2 v6 \^{}7 v5
=34+ \^{}5 \^{}3 v2 v1 \^{}2 v6 \^{}7 v5
-M34+ \^{}5 \^{}3 v2 v1 \^{}2 v6 \^{}7 v5}

etc.

Finally, we use
\textbf{grep}
to search for the composite data:

\par 

\texttt{assemble temp.mnt temp.deg | ditto | grep '\^{}+M6.*5\$'}


\par 
In addition to linking together different types of data,
sometimes we may also need to use a stream editor to
modify the data in some way.
Suppose we wanted to test a theory that the tonic pitch
tends to be followed by a greater variety of melodic intervals than
precedes it.
That is, we might suspect that the tonic tends to be approached
in stereotypic ways -$\,$- such as from the leading-tone (+m2), from
the supertonic (-M2) or from the dominant (+P4);
but what follows the tonic may be less restricted.

\par 
In effect, we need to generate two inventories:
one for intervals that approach the tonic, and one for intervals
that follow the tonic.
We already know how to create an inventory of intervals approaching
a particular scale-degree:

\par 

\texttt{deg -a inputfile > temp1}
\\
\texttt{mint inputfile > temp2}
\\
\texttt{assemble temp1 temp2 | grep '\^{}[v\^{}]*1	' | sort | uniq -c $\backslash$

| sort -rn > inventory.pre}



\par 
For the intervals following the tonic, we need to use
\textbf{context} -n 2.
This will create pairs of intervals:
the first interval will indicate the approach, and the second interval
in each pair will indicate the continuation.

\par 

\texttt{deg -a inputfile > temp1}
\\
\texttt{mint inputfile | context -n 2 -o \^{}= > temp2}
\\
\texttt{humsed 's/ .*//' temp2 > intervals.pre}
\\
\texttt{humsed 's/.* //' temp2 > intervals.post}
\\
\texttt{assemble temp1 intervals.pre | grep '\^{}1	' | sort | uniq -c $\backslash$

| sort -rn > inventory.pre}

\texttt{assemble temp1 intervals.post | grep '\^{}1	' | sort | uniq -c $\backslash$

| sort -rn > inventory.post}



\par 
In some tasks, it may be necessary to generate more than one
\textbf{context}
output.
For example, suppose we wanted to identify possible
"cross relations" between two voices.
A cross relation occurs when an accidental occurs in one
voice but not in another voice within a brief period of time.
One approach is to extract each voice, translate to scale-degree
and create brief contexts of (say) 2 or 3 notes.
E.g.

\par 

\texttt{extract -f 1 inputfile | deg | context -n 3 -o \^{}= > lower.tmp}
\\
\texttt{extract -f 2 inputfile | deg | context -n 3 -o \^{}= > upper.tmp}


\par 
We can then assemble the two contexts together:

\par 

\texttt{assemble lower.tmp upper.tmp}


\par 
Suppose our inputs consisted of an ascending C major scale played in the
lower voice concurrent with an E major scale in the upper voice.
Our output would look as follows:
\\\\

\texttt{**deg**deg
*C:*C:
1 \^{}2 \^{}33 \^{}4+ \^{}5+
\^{}2 \^{}3 \^{}4\^{}4+ \^{}5+ \^{}6
\^{}3 \^{}4 \^{}5\^{}5+ \^{}6 \^{}7
\^{}4 \^{}5 \^{}6\^{}6 \^{}7 \^{}1+
\^{}5 \^{}6 \^{}7\^{}7 \^{}1+ \^{}2+
\^{}6 \^{}7 \^{}1\^{}1+ \^{}2+ \^{}3
..
..
*-*-}


In effect, each data record contains an agglomeration of three
successive notes from both voices.
Seaching for cross-relations would entail looking for scale
degrees that are both modified and unmodified concurrently.
For example, in the case of the subdominant pitch,
we could search for such instances as follows:

\par 

\texttt{assemble lower.tmp upper.tmp | rid -GLId $\backslash$

| egrep '4[+-].*	.*4([\^{}+-])|\$'}



\par 
The regular expression given to
\textbf{egrep}
searches for a subdominant pitch in the lower voice that is either
raised or lowered -$\,$- concurrent with
a subdominant pitch in the upper voice that has not been modified.
Notice the use of the tab character in the regular expressions to specify
the precise voice being searched.
We would also need to test for the reverse situation, where the modified
pitch is in the upper voice:

\par 

\texttt{assemble lower.tmp upper.tmp | rid -GLId $\backslash$

| egrep '4[\^{}+-].*	.*4[+-]'}



\par 
In a similar fashion, the user can mix together spines representing
highly diverse types of contextual information to carry out
searches for complex patterns or conditions.
For example, a user might search for a specific piano fingering
that coincides with particular interval-transitions and harmonic contexts.


\subsection*{Using \textit{context} with the \textit{-p} Option}

\par 
The
\textbf{-p}
option for
\textbf{context}
allows the output data records to be "pushed" forward
by a specified number of lines.
Consider the normal operation of
\textbf{context}
as illustrated below.
The left-hand spine represents the input
and the right-hand spine represents the output where the option
\textbf{-n 2}
has been specified.
\\\\

\texttt{**kern**kern
*C:*C:
cc d
dd e
ee f
ff g
gg a
aa b
bb cc
cc.
*-*-}

Now consider the effect of adding the
\textbf{-p}
option.
In this case, the complete command is:

\par 

\texttt{context -n 2 -p 1}


\par 
The corresponding result is:
\\\\

\texttt{**kern**kern
*C:*C:
c.
dc d
ed e
fe f
gf g
ag a
ba b
ccb cc
*-*-}

The data records have been pushed forward by one line:
a null token now appears at the beginning of the output spine
rather than at the end.
Similarly, consider the effect of the following command:

\par 

\texttt{context -n 4 -p 2}


\par 
The corresponding result is:
\\\\

\texttt{**kern**kern
*C:*C:
c.
d.
ec d e f
fd e f g
ge f g a
af g a b
bg a b cc
cc.
*-*-}

The output is now padded with two preceding null tokens
with a trailing null token at the end of the spine.
In summary, the
\textbf{-p}
option pushes the context records by a specified number of lines.
This allows us to move the contextual information around,
and so provides more possibilities for searching.
In the above case, the pitch `e' is aligned with contextual
information that indicates the two pitches that precede `e' and
the one pitch that follows it.


\par 
By way of example, suppose we are looking for a submediant pitch
that is approached by two melodic intervals of an ascending major
third followed by a descending major second.
First, we generate independent
\texttt{**mint}
and
\texttt{**deg}
outputs.
Next we process the \texttt{**mint} data using
\textbf{context}
to create pairs of successive intervals.
Without the
\textbf{-p}
option, the assembled output might look as follows:
\\\\

\texttt{**deg**mint
*C:*C:
3[e] +m2
\^{}4+m2 +M2
\^{}5+M2 +M3
\^{}7+M3 -M2
v6-M2 +m3
\^{}1+m3 -P4
v5.
*-*-}

With
\textbf{-p 1}
the output becomes:
\\\\

\texttt{**deg**mint
*C:*C:
3.
\^{}4[e] +m2
\^{}5+m2 +M2
\^{}7+M2 +M3
v6+M3 -M2
\^{}1-M2 +m3
v5+m3 -P4
*-*-}

Now we can search directly for the situation of interest:

\par 

\texttt{grep '6	+M3 -M2\$'}




\subsection*{Reprise}

\par 
The
\textbf{context}
command essentially transforms sequences of events into collections
of pseudo-concurrent events.
This pseudo-concurrent arrangement enables processing using
line-oriented or record-oriented tools -$\,$- most notably
\textbf{grep}, \textbf{sed},
\textbf{humsed}
and \textbf{awk}.
For example, it facilitates pattern searching using
\textbf{grep}
and also allows useful manipulations via tools such as \textbf{humsed}.
The manner by which data tokens are collected together
can be defined by a starting marker or an ending marker or both.
Particular types of data can be excluded or omitted from the collections
using the
\textbf{-o}
option, and the collections can be transported or pushed forward
through the spine using the
\textbf{-p}
option.

\par 
We've seen a number of ways by which
\textbf{context}
can be used to establish a particular context for data.
In
Chapter 21
we will see how the
\textbf{patt}
command can be used to establish other kinds of contexts
and how both of these commands can be used together.

\\
\begin{itemize}
\item 

\textbf{Next Chapter}
\item 

\textbf{Previous Chapter}
\item 

\textbf{Table of Contents}
\item 

\textbf{Detailed Contents}
\\\\

� Copyright 1999 David Huron
\end{document}
