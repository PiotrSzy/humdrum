% This file was converted from HTML to LaTeX with
% Tomasz Wegrzanowski's <maniek@beer.com> gnuhtml2latex program
% Version : 0.1
\documentclass{article}
\begin{document}



  
  
    
      
      
      
    
  



\\
\\

\section*{Chapter12}


[\textit{Previous Chapter}]
[\textit{Contents}]
[\textit{Next Chapter}]


\section*{Selecting Musical Parts and Passages}



A Humdrum file may contain an encoding of a full score,
or even of a large collection of scores.
Often, we would like to isolate or extract particular parts
or segments from a file or input stream.
For example, we might want to extract a particular instrumental part,
or select a group of related instruments;
we might want to isolate a particular passage,
remove certain measures, or extract a specific phrase;
we might want to pull out a labelled section of a score
(such as a \textit{Coda} or \textit{Da Capo}),
or we might want to select a particular verse in a strophic song.
In addition, we might want to isolate specific types of information,
such as the figured bass, melodic interval information,
or vocal text.

\par 
In this chapter we will explore two Humdrum tools for extracting
material:
\textbf{extract}
and
\textbf{yank}.

\par 
We know that Humdrum representations are structured like a grid with
horizontal data ("records") representing concurrent information,
and vertical data ("spines") representing sequentially
occuring information.
The Humdrum
\textbf{extract}
command can be used to isolate columns or spines of information.
The
\textbf{yank}
command can be used to isolate rows or records.
The
\textbf{extract}
command can be used to extract musical parts or other types of
information that might be represented in individual spines.
The
\textbf{yank}
command can be used to isolate passages or segments from an input,
such as specified measures, phrases, or sections.


\subsection*{Extracting Spines: The \textit{extract} Command}

\par 
The
\textbf{extract}
command allows the user to select one or more spines from a Humdrum input.
The command is typically used to extract parts (such as a tuba part)
from some multi-part score.
However,
\textbf{extract}
can also be used to isolate dynamic markings, musical lyrics,
or any other stream of information
that has been encoded as a separate Humdrum spine.

\par 
The
\textbf{extract" }
command has several modes of operation.
With the \textbf{-f} option, the user may specify a given
data column (spine) or "field" to extract.
Consider the opening of Bach's second Brandenburg
Concerto shown in Example 12.1.
\\\\
\textbf{Example 12.1.}  J.S. Bach \textit{Brandenburg Concerto No. 2}, mov. 1.

\par 

\texttt{!!!COM: Bach, Johann Sebastian
\texttt{!!!OPR: Six Concerts Avec plusieurs ... le prince regnant d'Anhalt-Coethen
\texttt{!!!OTL: Brandenburgische Konzerte F
\texttt{!!!XEN: Brandenburg Concerto No. 2 in F major.
\texttt{!!!OMV: Movement 1.
\texttt{!!!SCT: BWV 1047
\texttt{!! [Allegro]


\texttt{**kern**kern**kern**kern**kern**kern**kern**kern**kern**kern
\texttt{*ICklav*ICstr*ICstr*ICstr*ICstr*ICstr*ICstr*ICww*ICww*ICbras
\texttt{*Icemba*Icello*Icbass*Iviola*Ivioln*Ivioln*Ivioln*Ioboe*Ifltds*Itromp
\texttt{*IGcont*IGcont********
\texttt{*IGripn*IGripn*IGripn*IGripn*IGripn*IGripn*IGconc*IGconc*IGconc*IGconc
\texttt{!cembal!'cello!Bd'rip!Vd'rip!v'lin2!v'lin1!v'lino!oboe!flauto!tromba
\texttt{*k[b-]*k[b-]*k[b-]*k[b-]*k[b-]*k[b-]*k[b-]*k[b-]*k[b-]*k[]
\texttt{*F:*F:*F:*F:*F:*F:*F:*F:*F:*F:
\texttt{*M2/2*M2/2*M2/2*M2/2*M2/2*M2/2*M2/2*M2/2*M2/2*M2/2
\texttt{*MM54*MM54*MM54*MM54*MM54*MM54*MM54*MM54*MM54*MM54
\texttt{*clefF4*clefF4*clefF4*clefC3*clefG2*clefG2*clefG2*clefG2*clefG2*clefG2
\texttt{8FF/8FF/8FFF/8a$\backslash$8cc$\backslash$8ff$\backslash$8ff$\backslash$8ff$\backslash$8ff$\backslash$8f/
\texttt{=1=1=1=1=1=1=1=1=1=1
\texttt{16F$\backslash$LL16F$\backslash$LL16FF$\backslash$LL8f$\backslash$L8a/L8cc$\backslash$L8cc$\backslash$L8cc$\backslash$L8cc$\backslash$L8a/L
\texttt{16G$\backslash$16G$\backslash$16GG$\backslash$.......
\texttt{16A$\backslash$16A$\backslash$16AA$\backslash$8c$\backslash$16f/LL16a$\backslash$LL16a$\backslash$LL16a$\backslash$LL16a$\backslash$LL8cc/
\texttt{16G$\backslash$JJ16G$\backslash$JJ16GG$\backslash$JJ.16g/JJ16b-$\backslash$JJ16b-$\backslash$JJ16b-$\backslash$JJ16b-$\backslash$JJ.
\texttt{16F$\backslash$LL16F$\backslash$LL16FF$\backslash$LL8f$\backslash$8a/L8cc$\backslash$L8cc$\backslash$L8cc$\backslash$L8cc$\backslash$L8a/
\texttt{16G$\backslash$16G$\backslash$16GG$\backslash$.......
\texttt{16A$\backslash$16A$\backslash$16AA$\backslash$8c$\backslash$J16f/LL16a$\backslash$LL16a$\backslash$LL16a$\backslash$LL16a$\backslash$LL8f/J
\texttt{16G$\backslash$JJ16G$\backslash$JJ16GG$\backslash$JJ.16g/JJ16b-$\backslash$JJ16b-$\backslash$JJ16b-$\backslash$JJ16b-$\backslash$JJ.
\texttt{16F$\backslash$LL16F$\backslash$LL16FF$\backslash$LL8f$\backslash$L8a/L8cc$\backslash$L8cc$\backslash$L8cc$\backslash$L8cc$\backslash$L8a/L
\texttt{16E$\backslash$16E$\backslash$16EE$\backslash$.......
\texttt{16F$\backslash$16F$\backslash$16FF$\backslash$8a$\backslash$8cc/8ff$\backslash$8ff$\backslash$8ff$\backslash$8ff$\backslash$8cc/
\texttt{16G$\backslash$JJ16G$\backslash$JJ16GG$\backslash$JJ.......
\texttt{16A$\backslash$LL16A$\backslash$LL16AA$\backslash$LL8cc$\backslash$8f/8cc$\backslash$8cc$\backslash$8cc$\backslash$8cc$\backslash$8ff/
\texttt{16B-$\backslash$16B-$\backslash$16BB-$\backslash$.......
\texttt{16A$\backslash$16A$\backslash$16AA$\backslash$8c$\backslash$J8cc/J8ff$\backslash$J8ff$\backslash$J8ff$\backslash$J8ff$\backslash$J8cc/J
\texttt{16G$\backslash$JJ16G$\backslash$JJ16GG$\backslash$JJ.......
\texttt{=2=2=2=2=2=2=2=2=2=2
\texttt{*-*-*-*-*-*-*-*-*-*-}

Suppose we wanted to extract the 'cello part.
In the above encoding, the 'cello occupies
the second spine (second field) from the left, hence:

\par 

\texttt{extract -f 2 brandenburg2.krn}


\par 
The resulting output would begin as follows:
\\\\

\texttt{!!!COM: Bach, Johann Sebastian
!!!OPR: Six Concerts Avec plusieurs ... le prince regnant d'Anhalt-Coethen
!!!OTL: Brandenburgische Konzerte F
!!!XEN: Brandenburg Concerto No. 2 in F major.
!!!OMV: Movement 1.
!!!SCT: BWV 1047
!! [Allegro]
**kern
*ICstr
*Icello
*IGcont
*IGripn
!'cello
*k[b-]
*F:
*M2/2
*MM54
*clefF4
8FF/
=1
16F$\backslash$LL
16G$\backslash$ }
etc.

Notice that the
\textbf{extract}
command outputs all global comments.
In the case of local comments,
\textbf{extract}
outputs only those local comments that belong to the output spine.


\par 
The oboe and flauto dolce parts are encoded in spines 8 and 9.
So we could extract the `cello, oboe and flauto dolce parts by
submitting a list of the corresponding fields.
Spine numbers are separated by commas:

\par 

\texttt{extract -f 2,8,9 brandenburg2.krn}


\par 
Numerical
\textbf{ranges}
can be specified using the dash.
For example, if we wanted to extract all of the string parts
(spines 2 through 7):

\par 

\texttt{extract -f 2-7 brandenburg2.krn}


\par 
With the
\textbf{-f}
option, field specifications may also be made with respect
to the right-most field.
The dollars-sign character (\texttt{\$})
refers to the right-most field in the input.
The trumpet part can be extracted as follows:

\par 

\texttt{extract -f '\$' brandenburg2.krn}


\par 
(Notice the use of the single quotes to ensure that the shell
doesn't misinterpret the dollar sign.)
Simple arithmetic expressions are also permitted;
for example `\texttt{\$-1}' refers to the right-most field minus one, etc.
By way of example, the command

\par 

\texttt{extract -f '\$-2' brandenburg2.krn}


\par 
will extract the oboe part.


\subsection*{Extraction by Interpretation}


\par 
Typically, it is inconvenient to have to determine the numerical
position of various spines in order to extract them.
With the
\textbf{-i}
option,
\textbf{extract}
outputs all spines containing a specified
\textit{interpretation.}
Suppose we had a file containing a Schubert song,
including vocal score, piano accompaniment and vocal text
(encoded using
\texttt{**text}).
The vocal text from the file \texttt{lieder} can be
extracted as follows:

\par 

\texttt{extract -i '**text' lieder}


\par 
(Notice again the need for single quotes in order to avoid
the asterisk being interpreted by the shell.)
Several different types of data can be extracted simultaneously.
For example:


\par 

\texttt{extract -i '**semits,**MIDI' hildegard}


\par 
will extract all spines in the file \texttt{hildegard}
containing
\texttt{**semits}
or
\texttt{**MIDI}
data.

\par 
An important use of the
\textbf{-i}
option for
\textbf{extract}
is to ensure that a particular input contains only a specified
type of information.
For example, the lower-case letter `\texttt{r}' represents a rest in the
\texttt{**kern}
representation.
If we wish to determine which sonorities contain rests,
we might want to use
\textbf{grep}
to search for this letter.
However, the input might contain other Humdrum interpretations
(such as \texttt{**text}) where the presence of the letter `\texttt{r}'
does not signify a rest.
We can ensure that our search is limited to \texttt{**kern} data
by using the
\textbf{extract}
command:

\par 

\texttt{extract -i '**kern' | grep  ...}



\par 
Both exclusive interpretations and tandem interpretations can be
specified with the
\textbf{-i}
option.
For example, the following command will extract any
\textit{transposing}
instruments in the score \texttt{albeniz}:

\par 

\texttt{extract -i '*ITr' albeniz}





\par 
Tandem interpretations are commonly used to designate instrument classes
and groups, so different configurations of instruments are easily extracted.
The Brandenburg Concerto shown in Example 12.1 illustrates a number
of tandem interpretations related to instrumentation classes and groups.
For example, the interpretation \texttt{*ICww} identifies woodwind
instruments;
\texttt{*ICbras} identifies brass instruments;
\texttt{*ICstr} identifies string instruments.
In addition, \texttt{*IGcont} identifies "continuo" instruments;
\texttt{*IGripn} identifies "ripieno" instruments;
and \texttt{*IGconc} identifies "concertino" instruments.
The following three commands extract (1) the woodwind instruments,
(2) the ripieno instruments, and (3) any vocal parts, respectively.

\par 

\texttt{extract -i '*ICww' concerto4}
\\
\texttt{extract -i '*IGrip' brandenburg2}
\\
\texttt{extract -i '*ICvox' symphony9}



\par 
Once again, more than one interpretation can be extracted simultaneously.
The following command will extract the instrument-class
"strings" and the instrument "oboe" from the file \texttt{milhaud}.

\par 

\texttt{extract -i '*ICstr,*Ioboe' milhaud}



\par 
Similarly, the following command will extract the shamisen
and shakuhachi parts from a score:

\par 

\texttt{extract -i '*Ishami,*Ishaku' hito.uta}


\par 
The behavior of
\textbf{extract}
is subtly different for tandem interpretations versus exclusive
interpretations.
Remember that exclusive interpretations are mutually exclusive,
whereas tandem interpretations are not.
Consider the following Humdrum representation:

\texttt{**foo
a
b
**bar
x
y
z
*-}

The command

\par 

\texttt{extract -i '**foo'}


\par 
will result in the output:

\texttt{**foo
a
b
*-}

Whereas the command

\par 

\texttt{extract -i '**bar'}


\par 
will result in the output:

\texttt{**bar
x
y
z
*-}

The \texttt{**foo} and \texttt{**bar} data are mutually exclusive.
Now consider an input file where \texttt{foo} and \texttt{bar}
are tandem interpretations:

\texttt{**foobar
*foo
a
b
*bar
x
y
z
*-}

The command

\par 

\texttt{extract -i '*foo'}


\par 
will result in the output:

\texttt{**foobar
*foo
a
b
*bar
x
y
*-}

Whereas the command

\par 

\texttt{extract -i '*bar'}


\par 
will result in the output:

\texttt{**foobar
*foo
*bar
x
y
z
*-}

When searching for a particular exclusive interpretation,
\textbf{extract}
resets each time a new exclusive interpretation is encountered.
By contrast, when
\textbf{extract}
finds a target tandem interpretation, it begins outputting
and doesn't stop until the spine is terminated.


\subsection*{Using \textit{extract} in Pipelines}

\par 
Of course the output from
\textbf{extract}
can be used to generate inputs for other Humdrum tools.
Here are a few examples.

\par 
Recall that the
\textbf{census}
command tells us basic information about a file.
With the
\textbf{-k}
option,
\textbf{census}
will tell us the number of barlines, the number of rests,
the number of notes, the highest and lowest notes,
and the longest and shortest notes for a **kern input.
The following commands can be used to determine this information
for (1) a bassoon part, (2) all woodwind parts:

\par 

\texttt{extract -i '*Ifagot' ives | census -k}
\\
\texttt{extract -i '*ICww' ives | census -k}



\par 
With the
\textbf{midi}
and
\textbf{perform}
commands,
\textbf{extract}
allows the user to hear particular parts.
For example, the following command extracts the bass and soprano
voices, translates them to
\texttt{**MIDI}
data, and plays the output:

\par 

\texttt{extract -i '*Ibass,*Isopran' lassus | midi | perform}



\par 
We might extract a particular part (such as the trumpet part) and use the
\textbf{trans}
command to transpose it to another key:

\par 

\texttt{extract -i '*Itromp' purcell | trans -d +1 -c +2}


\par 
In addition, we might extract a particular instrument or group of
instruments for notational display using the
\textbf{ms}
command.
The following command will extract the string parts and create a postscript
file for displaying or printing.

\par 

\texttt{extract -i '*ICstr' brahms | ms > brahms.ps}



\par 
The UNIX
\textbf{lpr}
command can be used to print a file or input stream.
Suppose we want to transpose the piano accompaniment
for a song by Hugo Wolf up an augmented second,
and then print the transposed part:

\par 

\texttt{extract -i '*IGacmp' wolf | trans -d +1 -c +3 | ms | lpr}



\subsection*{Extracting Spines that Meander}

\par 
As we saw in
Chapter 5,
spines can move around via various
spine-path interpretations.
Changes of spine position will cause havoc
when extracting by fields (the \textbf{-f} option);
\textbf{extract}
will generate an error message and terminate.
With the
\textbf{-i}
option,
\textbf{extract}
will follow the material throughout the input.

\par 
Consider the following input:

\texttt{**mip**dip**dip**blip
Aabx
Aabx
**\^{}**
Aa1a2bx
Aa1a2bx
Aa1a2bx
*-*-*-*-*-}

Suppose we want to extract the second spine (the first \texttt{**dip}) spine.
Using the field option (\textbf{-f})
will generate an error message since this spine splits.
Similarly, using the interpretation (\textbf{-i}) option
will fail because the output will contain
\textit{all}
of the \texttt{**dip} spines.

\par 
The
\textbf{extract}
command provides a third
\textbf{-p}
option that traces specific spine
\textit{paths.}
Like the
\textbf{-f}
option, the
\textbf{-p}
option requires one or more numbers indicating the
\textit{beginning}
field position for the spine.
The command

\par 

\texttt{extract -p 2 }...


\par 
will generate the following output:

\texttt{**dip
a
a
*\^{}
a1a2
a1a2
a1a2
*-*-}

In 
\textit{spine-path mode,}
the
\textbf{extract" }
command follows a given spine starting at the beginning of the file,
and traces the course of that spine throughout the input stream.
If spine-path changes are encountered in the input
(such as spine exchanges, spine merges, or spine splits)
the output adapts accordingly.
If the "nth" spine is selected,
the output consists of the nth spine and follows the path of that spine
throughout the input until it is terminated or the end-of-file is encountered.
What begins as the nth column, may end up as
some other column (or columns) in the input.

\par 
There are complex circumstances where the
\textbf{-p}
option will not guarantee an output that conforms to the Humdrum syntax.
When using the
\textbf{-p}
option it is prudent to check the output using the
\textbf{humdrum}
command in order to ensure that the output is valid.
A full discussion of the
\textbf{-p}
option is given in the
\textit{Humdrum Reference Manual.}


\subsection*{Field-Trace Extracting}

\par 
For circumstances where the input is very complex,
\textbf{extract}
provides a
\textit{field-trace mode}
(\textbf{-t} option)
that allows the user to select any combination of data tokens
from the input stream.
The field-trace option is rarely used when extracting spines.
Refer to the
\textit{Humdrum Reference Manual}
for further information.


\subsection*{Extracting Passages:  The \textit{yank} Command}

\par 
A useful companion to the
\textbf{extract}
command is the Humdrum
\textbf{yank}
command.
The
\textbf{yank}
command can be used to selectively extract segments or
passages from a Humdrum input.
The yanked material can be identified by absolute line numbers,
or relative to some marker.
In addition,
\textbf{yank}
is able to output logical segments,
such as measures, phrases, or labelled sections,
and is able to output material according to content.
The output always consists of complete records;
\textbf{yank}
never outputs partial contents of a given input record.

\par 
The
\textbf{yank}
command provides five different ways of extracting material.
The simplest way of yanking material is by specifying ranges
of line numbers.
In the following command, the
\textbf{-l}
option invokes the line-number operation.
The
\textbf{-r}
option is used to specify the range.
Ranges are defined by integers separated by commas,
or with a dash indicating a range of consecutive values.

For example, the following command selects lines, 5, 13, 23, 24, 25 and 26
from the file named \texttt{casella}:

\par 

\texttt{yank -l -r 5,13,23-26 casella}



\par 
The dollar sign (\$) can be used to refer to the last record in the input.
For example, the following command yanks the first and last
records from the file \texttt{mosolov}.

\par 

\texttt{yank -l -r '1,\$' mosolov}




\par 
Once again note that single quotes are needed here in order to prevent
the shell from misinterpreting characters such as the dollar
sign or the asterisk.
Records close to the end of the input can be specified by subtracting
some value from \$.
For example, the following command yanks the first 20 records
from the last 30 records contained in the file \texttt{ginastera}.
Notice that the dash/minus sign is used both to convey a range
and as an arithmetic operator.

\par 

\texttt{yank -l -r '\$-30-\$-10' ginastera}


\par 
If
\textbf{yank}
is given a Humdrum input, it always produces a syntactically
correct Humdrum output.
All interpretations prior to and within the yanked
material are echoed in the output.
The
\textbf{yank}
command also appends the appropriate spine-path terminators at
the end of the yanked segment.
By way of example, if we yanked line 10 (containing 4 spines)
and line 100 (containing 5 spines),
\textbf{yank}
will include in the output the appropriate spine-path interpretations
that specify how 4 spines became 5 spines.


\subsection*{Yanking by Marker}

\par 
Alternatively,
\textbf{yank}
can output lines relative to some user-defined
\textit{marker.}
This mode of operation can be invoked using the
\textbf{-m}
option.
Markers are specified using regular expressions.
The range option (\textbf{-r}) specifies which lines are to be output
whenever a marker is encountered.

For example, the following command outputs the first and third data records
following each occurrence of the string "XXX" in the file \texttt{wieck}.

\par 

\texttt{yank -m XXX -r 1,3 wieck}


\par 
If the value zero is specified in the range, the record containing
the marker is itself output.


\par 
Since markers are interpreted by
\textbf{yank}
as regular expressions, complex markers can be defined.
For example, the following command yanks the first data record
following any record in the file \texttt{franck}
beginning with a letter and ending with a number:

\par 

\texttt{yank -m '\^{}[a-zA-Z].*[0-9]\$' -r 1 franck}


\par 
Using
\textbf{yank -m}
with a range defined as zero is an especially useful construction:

\par 

\texttt{yank -m \textit{regexp} -r 0}


\par 
This command is analogous to the familiar
\textbf{grep}
command.
However, the output from
\textbf{yank}
will preserve all of the appropriate interpretations.
In short,
\textbf{yank}
guarantees that the output conforms to the Humdrum syntax,
whereas
\textbf{grep}
does not.



\par 
Suppose, for example, that we wanted to calculate the pitch intervals
between notes that either begin or end a phrase in a monophonic input.
If we use
\textbf{grep}
to search for
\texttt{**kern}
phrase indicators, we will be unable
to process the resulting (non-Humdrum) output,
since it will typically consist of just data records:

\par 

\texttt{grep [\{\}] sibelius}



\par 
By contrast, the comparable
\textbf{yank}
command preserves the Humdrum syntax and so allows us to pipe
the output to the melodic interval command:

\par 

\texttt{yank -m [\{\}] -r 0 sibelius | mint}



\subsection*{Yanking by Delimiters}

\par 
It is often convenient to yank
material according to logical segments such as measures or phrases.
In order to access such segments, the user must specify a segment
\textit{delimiter}
using the
\textbf{-o}
option
or the
\textbf{-o}
and
\textbf{-e}
options.
For example, common system barlines are represented by
the presence of an equals sign (=)
at the beginning of a data token.
Thus the user might yank specific measures from a
file by defining the appropriate barline delimiter and providing a range of
(measure) numbers.
Consider the following command:


\par 

\texttt{yank -o \^{}= -r 1,12-13,25 joplin}


\par 
This command will extract the first, twelfth, thirteenth
and twenty-fifth measures from the file \texttt{joplin}.
Unlike the
\textbf{-m}
option, the
\textbf{-o}
option interprets the range list as
\textit{ordinal}
occurrences of segments delineated by the delimiter.
Whole segments are output rather than specified records as
is the case with
\textbf{-m.}
As in the case of markers, delimiters are interpreted according to
regular expression syntax.
Each input record containing the delimiter is regarded as the
\textit{start}
of the next logical segment.
In the above command, the range (\textbf{-r}) specifies that the first,
twelfth, thirteenth, and twenty-fifth logical segments (measures)
are to be yanked.
All records starting with the delimiter record are output up to,
but not including, the next occurrence of a delimiter record.


\par 
Where the input stream contains data prior to the first delimiter record,
this data may be addressed as logical segment "zero."
For example,

\par 

\texttt{yank -o \^{}= -r 0 mahler}


\par 
can be used to yank all records prior to the first common system barline.
Notice that
\textit{actual}
measure numbers are irrelevant with the
\textbf{-o}
option:
\textbf{yank}
selects segments according to their
\textit{ordinal}
position in the input stream rather than according to their
\textit{cardinal}
label.



\par 
Not all segments are defined by a single marker.
For example, unlike barlines, \texttt{**kern} phrases are marked
by separate phrase-begin signifiers (`\{') and phrase-end signifiers (`\}').
The
\textbf{-e}
option for
\textbf{yank}
can be used to explicitly identify markers that
\textit{end}
a segment.
For example, the following command extracts the first four phrases
in the file \texttt{tailleferre}:

\par 

\texttt{yank -o \{ -e \} -r '1-4' tailleferre}


\par 
When the
\textbf{-n}
option is invoked, however,
\textbf{yank}
expects a numerical value to be present in the input immediately following the
user-specified delimiter.
In this case,
\textbf{yank}
selects segments based on their numbered label rather than their ordinal
position in the input.
For example,


\par 

\texttt{yank -n \^{}= -r 12 goldberg}


\par 
will yank all segments beginning with the label \texttt{=12} in the input
file \texttt{goldberg}.
If more than one segment carries the specified segment number(s),
all such segments are output.
That is, if there are five measures labelled "measure 12",
all five measures will be output.
Note that the dollar sign anchor cannot be used in the range expression
for the
\textbf{-n}
option.
Note also that input tokens containing non-numeric characters appended
to the number will have no effect on the pattern match.
For example, input tokens such as \texttt{=12a, =12b}, or \texttt{=12};
will be treated as equivalent to \texttt{=12}.

\par 
As in the case of the
\textbf{-o}
option, a range of zero (`0') addresses material prior to the first
delimiter record.
(N.B. This behavior is unlike the
\textbf{-m}
option where zero addresses the record itself.)
Like the
\textbf{-o}
option, the value zero may be reused for each specified input file.
Thus, if \texttt{file1}, \texttt{file2} and \texttt{file3} are Humdrum files:


\par 

\texttt{yank -n \^{}= -r 0 file1 file2 file3}


\par 
will yank any leading (anacrusis) material in each of the three files.


\subsection*{Yanking by Section}

\par 
When the
\textbf{-s}
option is invoked,
\textbf{yank}
extracts passages according to Humdrum section labels
encoded in the input.
Humdrum section labels will be described fully in
Chapter 20.
For now, we can simply note that section labels are tandem
interpretations that conform to the syntax:

\par 

\texttt{*>\textit{label\_name}}




\par 
Label names can include any character except the tab.
Labels are frequently used to indicate formal divisions, such as
coda, exposition, bridge, second ending, trio, minuet, etc.
The following command yanks the second instance of a section
labelled \texttt{First Theme} in the file \texttt{mendelssohn}:

\par 

\texttt{yank -s 'First Theme' -r 2 mendelssohn}


\par 
Note that with "through-composed" Humdrum files
it is possible to have more than one section containing
the same section-label.
Such situations are described in
Chapter 20.


\subsection*{Examples Using \textit{yank}}

\par 
As mentioned earlier,
\textbf{yank}
will always produce a syntactically correct Humdrum output if given a
proper Humdrum input.
All interpretations prior to, and within, the yanked material are
echoed in the output.

\par 
Any
\textit{comments}
prior to the yanked passage may be included in
the output by specifying the
\textbf{-c}
option.


\par 
The following examples illustrate how the
\textbf{yank}
command may be used.

\par 

\texttt{yank -l -r 1120 messiaen}


\par 
yanks line 1120 in the file \texttt{messiaen}.


\par 

\texttt{yank -n \^{}= -r 27 sinfonia}


\par 
yanks numbered measures 27 from the \texttt{**kern} file \texttt{sinfonia}.


\par 

\texttt{yank -n \^{}= -r 10-20 minuet waltz}


\par 
yanks numbered measures 10 to 20 from both
the \texttt{**kern} files \texttt{minuet} and \texttt{waltz}.


\par 

\texttt{yank -o \^{}= -r '0,\$' fugue ricercar}


\par 
yanks any initial anacrusis material plus the final measure
of both \texttt{fugue} and \texttt{ricercar}.

\par 

\texttt{cat fugue ricercar | yank -o \^{}= -r '0,\$'}


\par 
yanks any initial anacrusis material from the file \texttt{fugue} followed
by the final measure of \texttt{ricercar}.


\par 

\texttt{yank -n 'Rehearsal Marking ' -r 5-7 fugue ricercar}


\par 
yanks segments beginning with the labels
\texttt{"Rehearsal Marking 5", "Rehearsal Marking 6"},
and
\texttt{"Rehearsal Marking 7"}.
Segments are deemed to end when a record is encountered containing
the text
\texttt{"Rehearsal Marking "}.


\par 

\texttt{yank -o \{ -e \}  -r '1-\$' webern}


\par 
yanks all segments in the file \texttt{webern} beginning with a record
containing "\{" and ending with a record containing "\}."
The command:


\par 

\texttt{yank -o \{ -e \} -r '1-4,\$-3-\$' faure}


\par 
yanks the first four and last four segments in the file \texttt{faure},
where segments begin with an open brace (\{) and end with a
closed brace (\}).
In the \texttt{**kern} representation, this would extract the
first four and last four phrases in the file.


\par 

\texttt{yank -s Coda -r 1 stamitz}


\par 
will yank the first occurrence of a section labelled \texttt{Coda}
in the file \texttt{stamitz}.

\par 
Note that yanked segments are output in exactly the order
they appear in the input file.
For example, assuming that measure numbers in an input stream
increase sequentially,
\textbf{yank}
is unable to output measure number 6 prior to outputting measure number 5.
The order of output material can be rearranged by
invoked the
\textbf{yank}
command more than once
(e.g. \texttt{yank -l -r 100 ...; yank -l -r 99 ...;
yank -l -r 98 ...}).


\subsection*{Using \textit{yank} in Pipelines}


\par 
Like the other tools we have examined,
\textbf{yank}
can be profitably used in conjunction with other Humdrum tools.
It is often useful to employ more than one
\textbf{yank}
in a pipeline.
In the following command, the first
\textbf{yank}
isolates the `Trio' section from the input file, and the second
\textbf{yank}
isolates the first four measures of the extracted Trio:

\par 

\texttt{yank -s Trio dvorak | yank -o \^{}= 1-4}



\par 
Similarly, we can link two
\textbf{yank}
commands to extract particular phrases from specified sections.
For example, suppose we wanted to compare the first phrase
of the exposition with the first phrase of the recapitulation:


\par 

\texttt{yank -s Exposition haydn | yank -o \{ -e \} -r 1 > Ephrase}
\\
\texttt{yank -s Recapitulation haydn | yank -o \{ -e \} -r 1 > Rphrase}




\par 
Suppose we want to know how many notes there are
in measures 8-16 in a \texttt{**kern} file named \texttt{borodin}.

\par 

\texttt{yank -n = -r 8-16 borodin | census -k}



\par 
Are there any subdominant chords between measures 80 and 86?


\par 

\texttt{yank -n = -r 80-86 elgar | solfa | grep fa | grep la | grep do}




\par 
How frequent is the dominant pitch in Strauss' horn parts?

\par 

\texttt{extract -i '*Icor' strauss | solfa | grep -c so}



\par 
Combining
\textbf{yank}
and
\textbf{extract}
can be especially useful.
What is the highest note in the trumpet part in measure 29?


\par 

\texttt{extract -i '*Itromp' tallis | yank -n = -r 29 | census -k}



\par 
Also, we can combine
\textbf{yank}
with the
\textbf{midi}
and
\textbf{perform}
commands to hear particular sections.
Play the Trio section in a Waldteufel waltz:

\par 

\texttt{yank -s 'Trio' -r 1 waldteufel | midi | perform}



\par 
Listen to the soprano clarinet part in the fourth and eighth phrases.

\par 

\texttt{extract -i '*Iclars' quintet | yank -o \{ -e \} -r 4,8 $\backslash$

| midi | perform}



\par 
Note that when using
\textbf{yank}
to retrieve passages by markers (such as phrase marks),
care must be taken since markers may be miscoordinated between
several concurrent parts.
Example 12.2 shows a passage that has overlapping phrases.
When trying to extract a particular phrase for a particular part,
the outputs will differ significantly depending on whether
the
\textbf{yank}
command is invoked
\textit{before}
or
\textit{after}
the
\textbf{extract}
command.

\textbf{Example 12.2.}  A Passage Containing Unsynchronized Phrases.
\par 

\texttt{**kern**kern
=1-=1-
2r8r
\texttt{.\{8g
\texttt{.8a
\texttt{.8b
=2=2
8r4cc
\{8e.
8f4dd\}
8a.
=3=3
8g\{4ee
8e.
4d\}4ff
=4=4
*-*-}


The order of execution for some commands may cause some subtle differences.
Suppose we wanted to identify the melodic intervals
present in measures 8-32 for a work by Toru Takemitsu.
The following two commands are likely to produce different results:


\par 

\texttt{yank -n = -r 8-32 takemitsu | mint}
\\
\texttt{mint takemitsu | yank -n = -r 8-32}


\par 
In the second case, an interval will probably be calculated between
between the last note of measure 7 and the first note of measure 8.
This interval will be absent in the first case.



\subsection*{Reprise}

\par 
In this chapter we have learned how to extract musical parts using
\textbf{extract}
and how to grab musical passages using
\textbf{yank}.
We saw that the
\textbf{extract}
command is also useful for isolating specific types of information,
such as the lyrics, or ensuring that no other type of information
is present in a data stream.
In the case of
\textbf{yank}
we saw that passages can be extracted by defining arbitrary delimiters.
In addition to extracting by measures, by sonorities, or by
labelled sections, we can extract by rests, phrase marks -$\,$-
in fact, by any user-defined marker.
We also saw how the command
\textbf{yank -m \textit{regular-expression} -r 0}
can be used as a more sophisticated version of
\textbf{grep}
-$\,$-  a search tool that ensures the output will conform to the Humdrum syntax.

\par 
In the next chapter we will discuss how segments of music
can be put back together again.

\\
\begin{itemize}
\item 

\textbf{Next Chapter}
\item 

\textbf{Previous Chapter}
\item 

\textbf{Table of Contents}
\item 

\textbf{Detailed Contents}
\\\\

� Copyright 1999 David Huron
\end{document}
