% This file was converted from HTML to LaTeX with
% Tomasz Wegrzanowski's <maniek@beer.com> gnuhtml2latex program
% Version : 0.1
\documentclass{article}
\begin{document}



  
  
    
      
      
      
    
  



\\
\\

\section*{Chapter24}


[\textit{Previous Chapter}]
[\textit{Contents}]
[\textit{Next Chapter}]


\section*{The Shell (III)}



In
Chapter 16
we learned about the
\textbf{alias}
feature of the shell.
The
\textbf{alias}
command allowed us to create new commands by assigning a complex
pipeline to a single-word command.
In this chapter we will learn how to use the shell to write
more complex programs.
Shell programs allow users to reduce lengthy sequences of
Humdrum commands to a single user-defined command.


\subsection*{Shell Programs}

\par 
A shell program is simply a script consisting of one or more
shell commands.
Suppose we had a complex procedure consisting of a number of
commands and pipelines:

\par 

\texttt{extract -i '**Ursatz' inputfile | humsed '/X/d' $\backslash$
\\

| context -o Y -b Z > Ursatz}

\texttt{extract -i '**Urlinie' inputfile | humsed '/X/d' $\backslash$
\\

| context -o Y -b Z > Urlinie

assemble Ursatz Urlinie | rid -GLId | graph}


\par 
In the above hypothetical script, we have processed an input file
called \texttt{inputfile}.
It may be that this is a procedure we would like to apply to
several different files.
Rather than typing the above command sequence for each file,
an alternative is to place the above commands in a file.
Let's assume that we put the above commands in a file called
\texttt{Schenker}.
In order to execute this file as a shell script, we need to assign
\textit{execute permissions}
to the file.
We can do this by invoking the UNIX
\textbf{chmod}
command.

\par 

\texttt{chmod +x Schenker}


\par 
The
\textbf{+x}
option causes
\textbf{chmod}
to add execute permissions to the file \texttt{Schenker}.
Using
\textbf{chmod}
we can change modes related to \textit{executing} a file,
\textit{reading} a file, and \textit{writing} to a file.
Possible mode changes include the following:

+xadd execute permission
-xdeny execute permission
+radd read permission
-rdeny read permission
+wadd write permission
-wdeny write permission


\par 
Having added execute permissions to the file, we can now execute
the shell script or program.
This is done simply by typing the name of the file;
in effect, the filename becomes a new command:

\par 

\texttt{Schenker}


\par 
Each time we type this command, our script will be executed anew.
Notice that in our script, the final output has not been
sent to a file.
As a result, the output from our \textbf{Schenker} command will
be sent to the screen (standard output).
It is convenient not to specify an output file in the script
since this is often something the user would like to specify.
When typing our new command, we can use file-redirection to place
the output in a user-specified file:

\par 

\texttt{Schenker > outputfile}


\par 
As currently written, our program can be applied only to an input
file whose name is literally \texttt{inputfile}.
If we wanted to, we could edit our script and update the name of the
input filename every time we want to use the command.
However, it would be more convenient to specify the input filename
on the command line -$\,$- as we can do for other commands.
For example, it would be convenient to be able to type commands
such as the following:

\par 

\texttt{Schenker opus118 > opus118.out}


\par 
In order to allow such a possibility, we can use a predefined
feature of the shell.
Whenever the shell receives a command, each item of information
on the command line is assigned to a shell variable.
The first item on the command line is assigned to the variable \$0
(normally, this is the command name).
For example, in the above example, \$0 is assigned the string value
"\texttt{Schenker}."
The variables \$1, \$2, \$3, etc. are assigned to each successive
item of information on the command line.
So in the above example, \$1 is assigned the string value
"\texttt{opus118}."

\par 
These shell variables can be accessed within the shell script itself.
We need to revise the script so that each occurrence of the input file
is replaced by the variable \$1:

\par 

\texttt{extract -i '**Ursatz' \$1 | humsed '/X/d' $\backslash$
\\

| context -o Y -b Z > Ursatz}

\texttt{extract -i '**Urlinie' \$1 | humsed '/X/d' $\backslash$
\\

| context -o Y -b Z > Urlinie

assemble Ursatz Urlinie | rid -GLId | graph}


\par 
This change means that our \textbf{Schenker} command can be applied
to any user-specified input file -$\,$- simply by typing the filename
in the command.


\subsection*{Flow of Control: The \textit{if} Statement}

\par 
Suppose we wanted our \textbf{Schenker} command to apply only to tonal works -$\,$- more
specifically, to works with a known key.
Before processing a work, we might want to have \textbf{Schenker} test for the presence
of a tandem interpretation specifying the key.

\par 
Let's begin by using
\textbf{grep}
to search for a key tandem interpretation.
An appropriate
\textbf{grep}
command would be:

\par 

\texttt{grep '\^{}$\backslash$*[A-Ga-g][\#-]*:' \$1}


\par 
Recall that we can assign the output of any command to a shell variable
by placing the command within back-quotes or greves, i.e.   `...`.
Let's assign the key interpretation to the variable KEY:

\par 

\texttt{KEY=`grep '\^{}$\backslash$*[A-Ga-g][\#-]*:' \$1`}


\par 
If no key indicator is found by \textbf{grep}, then the variable KEY will be empty.
We can test for this condition using the shell
\textbf{if}
statement.

\par 

\texttt{KEY=`grep '\^{}$\backslash$*[A-Ga-g][\#-]*:' \$1`
\\
if [ "\$KEY" = "" ]
\\
then

echo "Sorry, this input file has no key."
\\
exit

fi}


\par 
Notice that we use the dollars sign prior to the variable to mean
\textit{the contents of variable KEY}.
The double quotation marks allow a string comparison.
Our test is whether the variable \$KEY is equivalent to the empty or null string "".
If the test is true, then the commands following the \texttt{then} statement are
executed.
By convention, these commands are indented for clarity.
In the above case, two commands are executed if the \$KEY variable is empty.
The \textbf{echo} command causes the quoted string to be output.
The \textbf{exit} command causes the script to terminate.
Notice the presence of the \textbf{fi} command (\textbf{if} backwards).
This command simply indicates that the if-block has ended.

\par 
Of course, if there is a key designation, then it is appropriate to
execute the rest of our \textbf{Schenker} script.
The complete script would be as follows:

\par 

\texttt{KEY=`grep '\^{}$\backslash$*[A-Ga-g][\#-]*:' \$1`
\\
if [ "\$KEY" = "" ]
\\
then

echo "Sorry, this input file has no key."
exit

else}

\texttt{extract -i '**Ursatz' \$1 | humsed '/X/d' $\backslash$
\\

| context -o Y -b Z > Ursatz}

\texttt{extract -i '**Urlinie' \$1 | humsed '/X/d' $\backslash$
\\

| context -o Y -b Z > Urlinie

assemble Ursatz Urlinie | rid -GLId | graph

fi}


\par 
Notice the addition of the \textbf{else} statement.
The \textbf{else} statement delineates the block of commands to be
executed whenever the \textbf{if} condition fails -$\,$- that is,
when the \$KEY variable does \textit{not} equal the null string.
Once again, to make the script more readable, we indent the
commands contained in the else-block.

\par 
The \textbf{if} command provides many other ways of testing
some condition.
For example, the shell provides ways to determine whether a
file exists, and other features.


\subsection*{Flow of Control: The \textit{for} Statement}

\par 
In music research, a common task is to apply a particular
process or script to a large number of score files.
By way of illustration,
suppose we wanted to know the maximum number of notes in
any single folk melody in a collection of Czech folksongs.
Suppose further that we are located in a directory
containing a large number of Czech folksongs named
\texttt{czech01.krn},
\texttt{czech02.krn},
\texttt{czech03.krn},
and so on.

\par 
We would like to run the \textbf{census -k} command on each
file separately, but we'd prefer not to type the
command for each score.
The
\textbf{for}
statement provides a convenient way to do this.
The following commands might be typed directly
at the shell:

\par 

\texttt{for J in czech*.krn
\\
> do
\\
> census -k \$J | grep 'Number of notes:'
\\
> done | sort -n}


\par 
The pattern \texttt{czech*.krn} will be expanded to all of the
files in the current directory that it matches.
The variable \textbf{J} will take on each name in turn.
The commands between \textbf{do} and \textbf{done} will be executed
for each value of the variable \textbf{\$J}.
That is, initially \textbf{\$J} will have the value \texttt{czech01.krn}.
Having completed the do-done block of commands,
the value of \textbf{\$J} will become \texttt{czech02.krn}, and the
do-done block will be repeated.
This will continue until the value of \textbf{\$J} has taken on
all of the possible matches for \texttt{czech*.krn}.

\par 
The output might appear as follows:

\texttt{Number of notes:           31
\\
Number of notes:           32
\\
Number of notes:           32
\\
Number of notes:           34
\\
Number of notes:           35
\\
Number of notes:           39
\\
Number of notes:           39
\\
Number of notes:           40
\\
Number of notes:           48
\\
Number of notes:           48
\\
Number of notes:           55
\\
Number of notes:           78}
\\
etc.


\par 
Incidentally, the output from a \textbf{for} construction such as above
can be piped to further commands, so we might identify the maximum
number of notes in a Czech melody by piping the output through \textbf{sort -n}.


\subsection*{A Script for Identifying Transgressions of Voice-Leading}

\par 
Shell programs can be of arbitrary complexity.
Below is a shell program (dubbed \textbf{leader}) whose purpose is to
identify all instances of betrayals of nine classic rules of voice-leading
for a two-part input.
A number of refinements have been added to the program -$\,$-
including input file checking, and formatting of the output.

\par 
The program is invoked as follows:

\par 

\texttt{leader }


\par 
The input is assumed to contain two voices, each in a separate
\texttt{**kern}
spine.
The nominally lower voice should be in the first spine.
For music containing more than two voices, the Humdrum
\textbf{extract}
command should be used to select successive pairs of voices
for processing by \textbf{leader}.

\par 
\texttt{\# LEADER
\\
\#
\\
\# A shell program to check for voice-leading infractions.
\\
\# This command is invoked as:
\\
\#
\\
\#   leader 
\\
\#
\\
\# where  is assumed to be a file containing two voices, each
\\
\# in a separate **kern spine, where the nominally lower voice is in the
\\
\# first spine.
\\

\\
\# Before processing, ensure that a proper input file has been specified.
\\
if [ ! -f \$1 ]
\\
then   echo "leader: file \$1 not found"
\\
       exit
\\
fi
\\
if [ \$\# -eq 0 ]
\\
then   echo "leader: input file not specified"
\\
       exit
\\
fi
\\


\\
\# 1. Record the ranges for the two voices.
\\
echo 'Range for Upper voice:'
\\
extract -f 2 \$1 | census -k | egrep 'Highest|Lowest' | sed 's/\^{}/      /'
\\
echo 'Range for Lower voice:'
\\
extract -f 1 \$1 | census -k | egrep 'Highest|Lowest' | sed 's/\^{}/      /'
\\


\\
\# 2. Check for augmented or diminished melodic intervals.
\\
extract -f 1 \$1 | mint -b r | sed '/$\backslash$[[Ad][Ad]*$\backslash$]/d' | egrep -n '\^{}[\^{}!*].*[Ad][\^{}1]' |$\backslash$
\\
    sed 's/:/ (/;s/\$/)/;s/\^{}/Augmented or diminished melodic interval at line: /'
\\
extract -f 2 \$1 | mint -b r | sed '/$\backslash$[[Ad][Ad]*$\backslash$]/d' | egrep -n '\^{}[\^{}!*].*[Ad][\^{}1]' |$\backslash$
\\
    sed 's/:/ (/;s/\$/)/;s/\^{}/Augmented or diminished melodic interval at line: /'
\\

\\
\# 3. Check for consecutive fifths and octaves.
\\
echo 'P5'  > \$TMPDIR/template;  echo 'P5'  >> \$TMPDIR/template
\\
hint -c \$1 | patt -f \$TMPDIR/template -s = | $\backslash$
\\
    sed 's/ of file.*/./;s/.*Pattern/Consecutive fifth/'
\\
echo 'P1'  > \$TMPDIR/template;  echo 'P1'  >> \$TMPDIR/template
\\
hint -c \$1 | patt -f \$TMPDIR/template -s = | $\backslash$
\\
    sed 's/ of file.*/./;s/.*Pattern/Consecutive octave/'
\\


\\
\# 4. Check for doubling of the leading-tone.
\\
deg \$1 | extract -i '**deg' | ditto -s = | sed 's/\^{}=.*/=/' | $\backslash$
\\
    egrep -n '\^{}7.*7|\^{}[\^{}!*].*7.*7' | egrep -v '7[-+]' | $\backslash$
\\
    sed 's/:.*/./;s/\^{}/Leading-tone doubled at line: /'
\\

\\
\# 5. Check for unisons.
\\
semits -x \$1 | ditto -s = | $\backslash$
\\
    awk '\{if(\$0~/[\^{}0-9$\backslash$t-]/)next\}\{if(\$1==\$2) print "Unison at line: " NR\}'
\\


\\
\# 6. Check for the crossing of parts.
\\
semits -x \$1 | ditto -s = | sed 's/\^{}=.*/=/' | $\backslash$
\\
    awk '\{if(\$0~/[\^{}0-9$\backslash$t-]/)next\}\{if(\$1>\$2) print "Crossed parts at line: " NR\}'
\\


\\
\# 7. Check for more than an octave between the two parts.
\\
semits -x \$1 | ditto -s = | awk '\{if(\$0~/[\^{}0-9$\backslash$t-]/)next\} $\backslash$
\\
    \{if(\$2-\$1>12) print "More than an octave between parts at line: " NR\}'
\\


\\
\# 8. Check for overlapping parts.
\\
extract -f 2 \$1 | sed 's/\^{}=.*/./' | context -n 2 -p 1 -d XXX | $\backslash$
\\
    rid -GL | humsed 's/XXX.*//' > \$TMPDIR/upper
\\
extract -f 1 \$1 | sed 's/\^{}=.*/./' > \$TMPDIR/lower
\\
assemble \$TMPDIR/lower \$TMPDIR/upper | semits -x | ditto | $\backslash$
\\
    awk '\{if(\$0~/[\^{}0-9$\backslash$t-]/)next\}\{if(\$1>\$2) print "Parts overlap at line: " NR\}'
\\
extract -f 1 \$1 | sed 's/\^{}=.*/./' | context -n 2 -p 1 -d XXX | $\backslash$
\\
    rid -GL | humsed 's/XXX.*//' > \$TMPDIR/lower
\\
extract -f 2 \$1 | sed 's/\^{}=.*/./' > \$TMPDIR/upper
\\
assemble \$TMPDIR/lower \$TMPDIR/upper | semits -x | ditto | $\backslash$
\\
    awk '\{if(\$0~/[\^{}0-9$\backslash$t-]/)next\}\{if(\$1>\$2) print "Parts overlap at line: " NR\}'
\\

\\

\# 9. Check for exposed octaves.
\\
hint -c \$1 > \$TMPDIR/s1
\\
extract -f 1 \$1 | deg > \$TMPDIR/s2
\\
extract -f 2 \$1 | deg > \$TMPDIR/s3
\\
extract -f 1 \$1 | mint | humsed 's/.*[3-9].*/leap/' > \$TMPDIR/s4
\\
extract -f 2 \$1 | mint | humsed 's/.*[3-9].*/leap/' > \$TMPDIR/s5
\\
assemble \$TMPDIR/s1 \$TMPDIR/s2 \$TMPDIR/s3 \$TMPDIR/s4 \$TMPDIR/s5 > \$TMPDIR/temp
\\
egrep -n 'P1.*$\backslash$\^{}.*$\backslash$\^{}.*leap.*leap|P1.*v.*v.*leap.*leap' \$TMPDIR/temp | $\backslash$
\\
    sed 's/:.*/./;s/\^{}/Exposed octave at line: /'
\\

\\
\# Clean-up some temporary files.
\\
rm \$TMPDIR/template \$TMPDIR/upper \$TMPDIR/lower \$TMPDIR/s[1-5] \$TMPDIR/temp}



\subsection*{Reprise}

\par 
In this chapter we have illustrated how to package complex Humdrum command
scripts into shell programs.
This allows us to create special-purpose commands.
We learned that files can be transformed into executable scripts through the
\textbf{chmod}
command.
We also learned how to pass parameters from the command line to
the script, and how to assign and modify the contents of variables.
In addition, we learned how to influence the flow of control using the
\textbf{if}
and
\textbf{for}
statements.
Finally, we learned that multi-line scripts can be typed directly
at the command line without creating a script file.

\par 
Shell scripts can be very brief or very long.
It is possible to create scripts that carry out highly sophisticated
processing such as searching for voice-leading transgressions.
There are innumerable features to shell programming that have
not been touched-on in this chapter.
Several books are available that provide comprehensive tutorials
for shell programming.



\subsection*{Locating Violations of the Rules of Voice-Leading}

\par 
The traditional rules of voice-leading have formed a standard
component of conservatory training for art musicians.

\par 
For illustration purposes, we'll apply some of the Humdrum tools
to the problem of identifying betrayals of the classic rules of
voice-leading in a \texttt{**kern}-encoded score.
Note that our purpose here is not to legislate how to
compose or arrange!
We're simply using the traditional voice-leading rules as
a way to introduce various pattern-searching techniques.


\subsection*{(1) Parts Out Of Range}

\par 
\paragraph*{RULE:  "Avoid parts that are out of range."}

\par 
The Humdrum
\textbf{census}
command provides a summary of various
elementary features of any Humdrum input.
With the
\textbf{-k}
option,
\textbf{census}
provides a summary of a further ten features
pertaining to \texttt{**kern} format inputs.
This includes the highest
and lowest notes present.

\par 

\texttt{census -k }


\par 
Since we are interested in the highest and lowest notes for
each individual part (rather than for the whole piece),
we should
\textbf{extract}
each part before processing it with
\textbf{census.}

\par 

\texttt{extract -i '*soprano'   > soprano.part}
\\
\texttt{census -k soprano.part}


\par 
On UNIX, a set of commands that sequentially process a given
input can be joined together as a "pipeline".  A pipeline feeds
the output of one process to the input of another process.  This
means that we can simplify the above sequence of commands into a
single pipeline:

\par 

\texttt{extract -i '*soprano'  | census -k}


\par 
We could then repeat the pipeline for each voice present:

\par 

\texttt{extract -i '*alto'     | census -k
\\
extract -i '*tenor'    | census -k
\\
extract -i '*bass'     | census -k}


\par 
If we wanted to get a little fancier, we could filter the
output so that only the highest and lowest pitch information
is output.
The UNIX
\textbf{grep}
command will let us define a string for searching;
\textbf{egrep}
permits compound strings, such as the use of the OR bar (|):

\par 

\texttt{extract -i '*soprano'   | census -k | egrep 'Highest|Lowest'}



\subsection*{(2) Augmented/Diminished Melodic Intervals}

\par 
\paragraph*{RULE:  "Avoid parts that move by augmented or diminished intervals.}

\par 
Implementing this is simple.  We first translate our pitch-
related data to the melodic interval format -$\,$-
\texttt{**mint}.
This can be done using the Humdrum
\textbf{mint}
command.
For example,
consider the following melodic fragment from the 2nd movement
of Bach's Brandenburg Concerto No. 5:
\\\\

\texttt{**kern
8r
8f\#
8b
16.cc\#
32dd
8a\#
16.b
32cc\#
8dd
*-}

Given this input, the
\textbf{mint}
command will produce the following
output.  Plus signs indicate ascending intervals, while minus
signs indicate descending intervals; 'P' means perfect, 'M'
means major, 'm' means minor, 'A' means augmented, and 'd'
means diminished:
\\\\

\texttt{**mint
[f\#]
+P4
+M2
+m2
-d4
+m2
+M2
+m2
*-}

Searching for diminished or augmented intervals is as simple
as using the Unix
\textbf{grep}
command, with the appropriate regular
expression:

\par 

\texttt{grep -n '[Ad]' }


\par 
The
\textbf{-n}
option will cause
\textbf{grep}
to prepend the line number of
any matching patterns, so we can refer back to the original
input file.

\par 
In order to avoid the letters `A' or `d' found in comments or
interpretations, we might consider using the Humdrum
\textbf{rid}
command:

\par 

\texttt{rid -GLI  | grep -n '[Ad]'}


\par 
However, this will cause the line numbers output by
\textbf{grep}
to be wrong.
The line numbers will correspond to the input file
with the comments and interpretations removed.

\par 
A better approach is to send the complete file to \textbf{grep},
and use a more circumspect regular expression to eliminate comments
and interpretations within \textbf{grep}.
\textbf{egrep}
allows us to define
more complex regular expressions:

\par 

\texttt{egrep '\^{}[\^{}!*].*[Ad]'}


\par 
The expression `\^{}[\^{}!*]' means "not an exclamation mark or asterisk
at the beginning of a line."  The expression `.*[Ad]' means
"zero or more instances of any character followed by either an
upper-case letter `A' or a lower-case letter `d'.

\par 
In other words, the complete regular expression matches any
line that contains either an upper-case `A' or lower-case `d'
as long as the beginning of the line does not start with an
exclamation mark (i.e. Humdrum comment) or an asterisk (i.e.
Humdrum interpretation).

\par 
If we want to look for augmented or diminished intervals in a
particular part or voice, we would begin by using the
\textbf{extract}
command to isolate the voice of interest.

\par 
Finally, putting all of the elements together in a Unix pipeline,
we get the following command for identifying augmented or
diminished melodic intervals:

\par 

\texttt{extract -i 'alto'  | mint | egrep -n '\^{}[\^{}!*].*[Ad]'}


\par 
If there is no output, then there are no augmented or diminished
intervals present.


\subsection*{(3) Consecutive Fifths or Octaves}

\par 
\paragraph*{RULE:  "Avoid consecutive fifths and octaves between any two parts.}

\par 
Let's focus on identifying consecutive fifths -$\,$- since the
process is the same for octaves.

\par 
Either the Humdrum
\textbf{patt}
or
\textbf{pattern}
commands can be used to find patterns that span more than
one line or record.
For this example, we'll use \textbf{patt}.

\par 
First, we need to reformat our input so the data represent
harmonic intervals rather than pitches.  The Humdrum
\textbf{hint}
command will change most pitch representations to the harmonic
interval representation -$\,$-
\texttt{**hint}.
Consider, for example, the following input:
\\\\

\texttt{**kern   **kern
=1       =1
4c       4e
4G       4d
=2       =2
2F       2c
*-       *-}

Given the following command:

\par 

\texttt{hint }


\par 
The following output will be produced:
\\\\

\texttt{**hint
=1
M3
P5
=2
P5
*-}

(Notice that, in this case, the consecutive fifths are
separated by a barline.)

\par 
Second, we need to define a pattern template for the
\textbf{patt}
command.
The template is a series of one or more regular
expressions that are stored in a separate file.  In this case
the pattern is trivial: just two consecutive perfect fifth
tokens.  We might store the following pattern in the file
"template":
\\\\

\texttt{P5
P5}

(Note that if we were looking for consecutive `fifths' that need
not necessarily be `perfect,' we could simply eliminate the
letter "P" in each interval given in the template.)

\par 
Given the above output from the
\textbf{hint}
command, we could search
for occurrences of the defined pattern using the following
command:

\par 



\par 
The
\textbf{-f}
option is used to identify the file (\texttt{template}) in
which the pattern-template has been stored.
The
\textbf{-s}
option tells
\textbf{patt}
of any input records that should be skipped during
the search process.
The
\textbf{-s}
option is followed by a regular expression -$\,$- in this
case the equals-sign -$\,$- so that any input records containing
the equals-sign (i.e. \texttt{**hint} barlines) are ignored.

\par 
The default output from
\textbf{patt}
identifies the location of any
instances of the pattern it finds in the source document.

\par 
The appropriate pipeline is:

\par 

\texttt{hint  | patt -f template -s =}


\par 
There are a few refinements we ought to add to this process.
Currently, we are searching for parallel perfect fifths only.
The consecutive fifths rule pertains to all intervals that
are compound-equivalents to perfect fifths (such as perfect
twelfths, etc.).

\par 
This additional criteria is easily handled.  The
\textbf{hint}
command provides a
\textbf{-c}
option that causes all compound intervals
to be represented by their non-compound equivalents.
For example, major tenths and major seventheenths, etc. will all
be represented as "M3", and so on.  Hence we would modify our
pipeline:

\par 

\texttt{hint -c  | patt -f template -s =}


\par 
(Note that an alternative way of handling the compound-intervals
question would be to define slightly more complex regular
expressions in our template file, e.g.
\\\\

\texttt{P5|P12|P19
P5|P12|P19}

In regular expressions the vertical bar (|) denotes the logical `OR'
operation.
So the above pattern says "a perfect fifth OR
a perfect twelfth OR a perfect nineteenth followed by a ...")

\par 
Another refinement relates to the selection of voices.
So far, we have presumed that the input consists of just two Humdrum
spines containing separate parts.  In a multi-part score, we
must examine each pair of voices in turn, in order to determine
whether any pair exhibit consecutive fifths or octaves.

\par 
The simplest (but more tedious) approach, is simply to execute
our command pipeline for each pairing of voices.  For example,
in a traditional four-part harmonization:

\par 

\texttt{extract -i '*soprano,*alto'   | hint -c | patt -f template -s =
\\
extract -i '*soprano,*tenor'  | hint -c | patt -f template -s =
\\
extract -i '*soprano,*bass'   | hint -c | patt -f template -s =
\\
extract -i '*alto,*tenor'     | hint -c | patt -f template -s =
\\
extract -i '*alto,*bass'      | hint -c | patt -f template -s =
\\
extract -i '*tenor,*bass'     | hint -c | patt -f template -s =}


\par 
(There are shorter ways of doing these permutations that involves
a little shell programming, but we'll leave that for another time.)


\subsection*{(4) Doubled Leading Tone}

\par 
\paragraph*{RULE:  "Avoid doubling the leading-tone."}

\par 
Pitches can be identified as "leading-tones" only when we have
some idea of their key-related scale-degree.  The Humdrum
\textbf{deg}
command translates pitch representations to a scale-degree 
representation where the numbers 1 to 7 represent tonic to
leading-tone designations.

\par 
Notice that the score input must contain an explicit key
indication (a special type of Humdrum tandem interpretation).
For example, the key of G major is indicated through the presence
of the following interpretation:
\\\\

\texttt{*G:}

Minor keys are indicated using lower-case characters.  For
example, the following passage is in B minor:
\\\\

\texttt{**kern
*b:
8r
8f\#
8b
16.cc\#
32dd
8a\#
16.b
32cc\#
8dd
*-}

The
\textbf{deg}
command can be used to transform this representation to
scale degree.  The passage begins on the dominant (degree `5'),
ascends (\^{}) to the tonic (`1'), ascends to the supertonic (\^{}2),
ascends to the median (\^{}3) and then descends to the leading-
tone (v7), etc.:
\\\\

\texttt{**deg
*b:
5
\^{}1
\^{}2
\^{}3
v7
\^{}1
\^{}2
\^{}3
*-}

The following command will eliminate durations or other possible
number representations that might conflict with scale-degree
designations:

\par 

\texttt{deg }


\par 
Having translated the representation in this way, we need to search
for any lines which contain two instances of the number `7' -$\,$-
that is, two concurrent instances of the leading-tone.

\par 
Searching for the number `7' is easily done using the standard
Unix
\textbf{grep}
(get regular expression) command:

\par 

\texttt{deg  | grep -n '7'}


\par 
This will find and output all records that contain the number 7;
the
\textbf{-n}
option means that the corresponding line number will
also be output.

\par 
However, we want to find instances where two or more 7s occur
on a single line.  For this, we can use a slightly more
complex regular expression

\par 

\texttt{deg  | grep -n '7.*7'}


\par 
In the construction ".*" the period (.) means any character, and
the asterisk means "zero or more instances of ..."  Hence, the
regular expression means "the number 7 followed by zero or more
instances of any character, followed by the number 7".  In short,
this expression will match any record in which the number 7
occurs at least twice.

\par 
As in the case of our earlier search for augmented and diminished
intervals,
\textbf{grep}
is insensitive to whether the matching character
string is found in a data record, or whether it occurs in a Humdrum
comment or interpretation.  In order to avoid matching comments
or interpretations, a further refinement to our regular expression
is appropriate.

\par 

\texttt{deg  | egrep -n '\^{}7.*7|\^{}[\^{}!*].*7.*7'}


\par 
In this case, the regular expression says the following: "find
any occurrence of the number 7 at the beginning of the line
followed by zero or more characters followed by the number 7;
or match any character at the beginning of the line -$\,$- other than
an exclamation mark or an asterisk -$\,$- followed by zero or more
characters, followed by the number 7, followed by zero or more
characters, followed by another number 7.

\par 
Depending on the input, it is possible that Humdrum spines will
be present that do not represent scale degree information.
It is possible that these other kinds of data may also make
use of the number 7 -$\,$- but
\textit{not}
to represent the leading-tone.
In other words, it is possible that a matching `7' has nothing
to do with scale degrees.  We can ensure that this doesn't
happen by first ensuring that
\textit{only}
scale-degree spines are present in the input to be searched.

\par 
In order to do this, we can use the Humdrum
\textbf{extract}
command as
a filter, and identify the types of interpretations we want to
pass.  In the following modification to our pipe, the
\textbf{extract}
command has been used to ensure that only
\texttt{**deg}
spines are present:

\par 

\texttt{deg  | extract -i '**deg' | egrep -n '\^{}7.*7|\^{}[\^{}!*].*7.*7'}


\par 
There are still some refinements that we could add to this
command sequence, but as it stands it is guaranteed to find all
doubled leading-tones -$\,$- provided the notes begin at the same
time.  Consider the following hypothetical passage:
\\\\

\texttt{**kern    **kern
*C:       *C:
8c        8g
=1        =1
4B        8g
.         16a
.         16b
4A        4cc
*-        *-}

Given the above command sequence, no doubled leading-tones
would be identified in this passage.  However, we might wish
to implement a more stringent criterion that would seek out any
instances where the leading-tone sounds at the same time in more
than one voice.  This occurs in the above example with the
sixteenth-note B concurrent with the held quarter-note B in
the other part.

\par 
This criterion can be accommodated by a further refinement to
our command pipeline.  The Humdrum
\textbf{ditto}
command is used to replace null data tokens by the immediately
preceding data token in the same spine.
Consider first, the output from the
\textbf{deg}
command for the above example:
\\\\

\texttt{**deg     **deg
*C:       *C:
1         5
=1        =1
v7        5
.         \^{}6
.         \^{}7
v6        \^{}1
*-        *-}

If we now invoke the
\textbf{ditto}
command, the modified output is:
\\\\

\texttt{**deg     **deg
*C:       *C:
1         5
=1        =1
v7        5
v7        \^{}6
v7        \^{}7
v6        \^{}1
*-        *-}

Notice that the two null tokens in the left-hand spine have
been replaced by copies of the most recent data token.
Now our
\textbf{grep}
command will find the two leading tones in the
second last data record.

\par 
In summary, the complete command pipeline would be:

\par 

\texttt{deg  | extract -i '**deg' | ditto -s = | egrep -n '\^{}7.*7|\^{}[\^{}!*].*7.*7'}


\par 
This may seem somewhat complicated, but the basic structure of
this pipeline is suitable for a very wide variety of pattern
searches.


\subsection*{(5) Avoid Unisons}

\par 
\paragraph*{RULE:  "Avoid the sharing of pitches by two parts."}

\par 
For this rule, let's assume that we also want to identify unisons
that are spelled enharmonically (such as F-sharp and G-flat).

\par 
First, we need to translate the two parts into some absolute
pitch representation -$\,$- such as frequency or semitones.  This
will ensure that enharmonically equivalent pitches have the
same representation -$\,$- and so will facilitate comparison.

\par 
The Humdrum
\textbf{semits}
command translates pitches to semitone
distances where middle C is denoted as zero.  For example,
where two voices both play B3 at the same time, both the parts
will have a
\texttt{**semits}
value of minus one (-1).

\par 
Like the
\textbf{deg}
command, the
\textbf{semits}
command provides a
\textbf{"-x}
option
that eliminates from the output stream any characters that don't
pertain to semitones.
Hence the following command will eliminate
durations or other possible numerical representations that might
conflict with semitone designations:

\par 

\texttt{extract -i '*alto,*tenor'}\textit{ }\texttt{ | semits -x}


\par 
Next we need to compare the two parts at each moment in order to
determine whether they have the same numerical value.
The Unix
\textbf{awk}
command will allow us to do some arithmetic.  Awk auto-
matically parses an input, so the value of the first spine is
referred to as `\$1', the value of the second spine is `\$2' and
so on.  The
\textbf{awk}
expression \texttt{`\$1==\$2'} is a test of whether the
first and second spines are equivalent.
The
\textbf{awk}
action \texttt{`print NR'} means to print the current
line number (record number is \texttt{`NR'}).

\par 
So the following command will print the line number for any
input in which the semitone value is the same for both the
alto and tenor voices:

\par 

\texttt{extract -i '*alto,*tenor'  | semits -x | awk '\{if(\$1==\$2) print NR\}'}


\par 
There is a problem with this pipeline however.  The
\textbf{awk}
command will match all sorts of non-numeric inputs -$\,$- such as where
null tokens (.) occur in both parts at the same time.  Consequently,
we need to be careful to avoid non-numeric inputs and comments.

\par 
The regular expression `[\^{}0-9]' will match any line that doesn't
consist solely of numbers.  The expression `[\^{}0-9-]' will match
any line that doesn't consist solely of numbers or the minus sign.
Since the tab character will also be present in our data records,
we should also include the tab in our regular expression.  The
tab may be denoted in regular expressions by the lower-case
letter `t' preceded by a backslash.  Hence the expression
`[\^{}0-9$\backslash$t-]' will match only those lines consisting solely of
numbers, the minus sign, and tabs.

\par 
The following
\textbf{awk}
script will output the line numbers for all
inputs where the first and second spines contain identical
numbers:

\par 

\texttt{awk '\{if(\$0~/[\^{}0-9$\backslash$t-]/)next\}\{if(\$1==\$2) print NR\}'}


\par 
Adding this construction to our pipeline produces the following
command for identifying unisons:

\par 

\texttt{extract -f 1,2  | semits -x | ditto -s = | awk '\{if(\$0~/[\^{}0-9$\backslash$t-]/)next\}\{if(\$1==\$2) print NR\}'}


\par 


\subsection*{(6) Crossed Parts}

\par 
\paragraph*{RULE:  "Avoid the crossing of parts."}

\par 
Part-crossing occurs when a nominally higher voice uses a pitch
that is lower than a nominally lower voice.

\par 
The relations "higher" and "lower" suggest the use of an arithmetic
operator such as greater-than (>) or less-than (<).  In brief, we
will approach this question by translating the pitches to a
numerical scale, and then use the general-purpose Unix
\textbf{awk}
command to test whether the nominally lower voice is truly lower.

\par 
First we need to translate the pitch representation to some sort
of numerical form.  We have several options.  We could translate
the pitches to frequency
(\texttt{**freq}),
or we could translate them to
semitones (\texttt{**semits}),
or we could translate them to cents (**cents).
Let's use \texttt{**semits}.
Once again, in this representation, middle-C
is represented by the number zero, and all other pitches are
represented by their semitone distance (positive or negative) with
respect to this reference.

\par 
We extract the two parts of interest, and then translate them to
the semitone numerical representation:

\par 

\texttt{extract -i '*soprano,*alto'  | semits -x}


\par 
Since part-crossing may occur when one voice is holding a note,
we should use the Humdrum
\textbf{ditto}
command, as we did for the doubled
leading-tone problem.  Hence:

\par 

\texttt{extract -i '*alto,*tenor'  | semits -x | ditto -s =}


\par 
Finally, we can use the Unix
\textbf{awk}
command to do a little arithmetic.
Once again, in \textbf{awk}, \texttt{`\$1'} and \texttt{`\$2'}
refer to the first and second input fields,
and the built-in variable \texttt{`NR'} refers to the current
record (line) number.
The expression \texttt{`\{if(\$1>\$2) print NR\}'} is
a miniature program that says: "if the first input field is
numerically greater than the second field for the current line,
then print the line number":

\par 

\texttt{extract -i '*alto,*tenor'  | semits -x | ditto -s = | awk '\{if(\$1>\$2) print NR\}'}


\par 
In short, if the left-most spine has a lower numerical value than the
second spine, then tell us where that occurs.

\par 
Since the \texttt{**semits} representation uses the
lower-case letter `r' to represent a rest,
we should avoid the possibility of comparing a
number (note) with a rest.
We can use a variation on the same
\textbf{awk}
script as we used when checking for unisons:

\par 

\texttt{awk '\{if(\$0~/[\^{}0-9$\backslash$t-]/)next\}\{if(\$1>\$2) print NR\}'}


\par 
Finally, the complete pipeline for identifying crossed parts:

\par 

\texttt{extract -i '*soprano,*alto'  | semits -x | ditto -s = | awk '\{if(\$0~/[\^{}0-9$\backslash$t-]/)next\}\{if(\$1>\$2) print NR\}'}
\\
extract -i '*alto,*tenor'    | semits -x | ditto -s = | awk '\{if(\$0~/[\^{}0-9$\backslash$t-]/)next\}\{if(\$1>\$2) print NR\}'
\\
extract -i '*tenor,*bass'    | semits -x | ditto -s = | awk '\{if(\$0~/[\^{}0-9$\backslash$t-]/)next\}\{if(\$1>\$2) print NR\}'


\par 


\subsection*{(7) Parts Separated by Greater than an Octave}

\par 
\paragraph*{RULE:  "Avoid intervals greater than an octave between the
soprano and alto voices.  Also avoid intervals greater than an
octave between the alto and tenor voices."}

\par 
Finding infringements of this voice-leading rule requires just
a slight modification to our method for identifying the crossing
of parts.

\par 
Having transformed the pitch input to a numerical form, we simply
need to check whether the difference between the two semitone
values is greater than 12 semitones.

\par 
The
\textbf{awk}
portion of our command is modified so that we are informed
if the nominally higher voice is more than 12 semitones away
from the other voice:

\par 

\texttt{extract -i '*soprano,*alto'  | semits -x | ditto -s = | awk '\{if(\$0~/[\^{}0-9$\backslash$t-]/)next\}\{if(\$2-\$1>12) print NR\}'
\\
extract -i '*alto,*tenor'    | semits -x | ditto -s = | awk '\{if(\$0~/[\^{}0-9$\backslash$t-]/)next\}\{if(\$2-\$1>12) print NR\}'}



\subsection*{(8) Overlapped Parts}

\par 
\paragraph*{RULE:  "Avoid the overlapping of parts, where the pitch in
an ostensibly lower voice moves to a pitch higher than
the previous pitch in an ostensibly higher voice; or
where the pitch in an ostensibly higher voice moves
to a pitch lower than the previous pitch in an
ostensibly lower voice."}

\par 
The following passage illustrates a violation of the part
overlapping rule:
\\\\

\texttt{**pitch   **pitch
C4        E4
F4        A4
E4        G4
*-        *-}

(In the second sonority, the lower voice (F4) moves to a
pitch higher than the previous pitch in the higher voice (E4).)

\par 
This rule is similar to the part-crossing rule, only we have
to compare the current pitch in one part with the previous
pitch in another part.

\par 
Rather than making a direct comparison, for the purpose of this
tutorial, we will make a modification to our earlier part-crossing
detector.  In brief, we will extract one of the parts, shift the
data tokens within that part, paste the two parts back together,
and then check to determine whether the shifted part shows any
"crossed parts" -$\,$- using our earlier command pipeline.

\par 
The following command pipe will shift the data tokens in a spine
down one record.  (The last data record will disappear.)

\par 

\texttt{context -n 2 -p 1 -d XXX  | humsed 's/XXX.*//'}


\par 
In this tutorial, we won't discuss how this works, since the
\textbf{context}
and
\textbf{humsed}
commands will be covered in a future
tutorial.  For now, we can note that shifting (say) the alto
part can be done by extracting the appropriate voice, and then
using the shift command sequence shown above:

\par 

\texttt{extract -i '**alto'  | context -n 2 -p 1 -d XXX | humsed 's/XXX.*//' > alto.shf}


\par 
If we want to compare, say, the soprano and alto voices, we need to
extract both parts, and shift one of them:

\par 

\texttt{extract -i '*soprano'  > soprano
\\
extract -i '*alto'  | context -n 2 -p 1 -d XXX | humsed 's/XXX.*//' > alto.shf}


\par 
Next, we need to assemble the shifted and unshifted parts back into
a single score:

\par 

\texttt{assemble alto.shf soprano > tempfile}


\par 
Then we test this intermediate file for instances of "part-crossing"
-$\,$-  using our earlier command pipeline:

\par 

\texttt{semits -x tempfile | ditto -s = | awk '\{if(\$0~/[\^{}0-9$\backslash$t-]/)next\}\{if(\$1>\$2) print NR\}'}


\par 
Avoiding the temporary file altogether:

\par 

\texttt{assemble alto.shf soprano | semits -x | ditto -s = | awk '\{if(\$0~/[\^{}0-9$\backslash$t-]/)next\}\{if(\$1>\$2) print NR\}'}


\par 
Note that this procedure has determined whether any of the notes
in the soprano voice are lower than the previous note in the
alto voice.  We also need to check whether any of the notes in
the alto voice are higher than the previous note in the soprano
voice.  To do this, we simply repeat the process, shifting the other
voice:

\par 

\texttt{extract -i '*soprano'  | context -n 2 -p 1 -d XXX | humsed 's/XXX.*//' > soprano.shf
\\
extract -i '*alto'  > alto
\\
assemble alto soprano.shf | semits -x | ditto -s = | awk '\{if(\$0~/[\^{}0-9$\backslash$t-]/)next\}\{if(\$1>\$2) print NR\}'}


\par 
This processing needs to be applied for each pair of successive
voices -$\,$- soprano/alto, alto/tenor, tenor/bass.


\subsection*{(9) Exposed Octaves}

\par 
\paragraph*{RULE:  "When approaching an octave by similar motion, ensure
that at least one of the parts moves by step."}

\par 
Violations of the exposed octaves rules must meet three
conditions.  First, the two voices must be separated by an
octave (or two octaves, or a unison, etc.).  (This suggests
that we use the
\textbf{hint}
command with the
\textbf{-c}
option in order to reduce compound intervals to their
non-compound equivalents.)
Second, the voices must be moving in the same direction.
(The \texttt{**deg} representation may be suitable here,
since it distinguishes
notes according to whether they are approached from below ("\^{}")
or above ("v").)  Third, both voices must be moving by leap
(e.g. more than two semitones).

\par 
To address this problem, let's plan to create five different
spines.  The first spine will encode harmonic interval size so
that all compound equivalents to a unison are represented by
the string "P1".

\par 
The second spine will indicate whether the lower voice is
ascending or descending ("\^{}" or "v").  Similarly, the third
spine will indicate whether the upper voices is ascending or
descending.

\par 
The fourth spine will indicate whether the melodic motion for
the lower voice is by leap ("leap"), and the fifth spine will
indicate whether the melodic motion for the upper voice is by
leap.

\par 
Examples of violations of the exposed octaves rule will appear
as one of the following two situations:
\\\\

\texttt{(**hint     **updown    **updown    **size   **size)
P1         \^{}           \^{}           leap     leap
P1         v           v           leap     leap

Any other situation means that the exposed octaves rule has not
been violated.

\par 
In other words, our final test can be expressed using the
following
\textbf{egrep}
command:

\par 

\texttt{egrep -n 'P1.*\^{}.*\^{}.*leap.*leap|P1.*v.*v.*leap.*leap'}


\par 
Now all we need to do is generate our five spines and assemble
them in the proper order.

\par 
The first spine is easily generated using the
\textbf{hint}
command.
Remember that the
\textbf{-c}
option means that all intervals an octave
or greater will be represented by the within-octave equivalent.

\par 

\texttt{extract -i '*alto,*tenor'  | hint -c > spine1}


\par 
The second and third spines can be generated using the Humdrum
\textbf{deg}
command:

\par 

\texttt{extract -i '*alto'   | deg > spine2
\\
extract -i '*tenor'  | deg > spine3}


\par 
The fourth and fifth spines require a little more work.  First,
we calculate the melodic intervals for each voice using the
Humdrum
\textbf{mint}
command.

\par 

\texttt{extract -i '*alto'  | mint  ...}


\par 
Secondly, we need to change all data tokens indicating intervals
greater than a diatonic second (3 or more semitones) into the
data token consisting of the (arbitrary) character string "leap".
This can be done using the
\textbf{humsed}
stream editor.

\par 

\texttt{ ...  humsed 's/.*[3-9].*/leap/' > spine4   [spine5 for the other voice]}


\par 
Putting it all together, the following command sequence will let
us identify any instances of exposed octaves between two arbitrary
voices:

\par 

\texttt{extract -i '*alto,*tenor'  | hint -c > spine1
\\
extract -i '*alto'   | deg > spine2
\\
extract -i '*tenor'  | deg > spine3
\\
extract -i '*alto'   | mint | humsed 's/.*[3-9].*/leap/' > spine4
\\
extract -i '*tenor'  | mint | humsed 's/.*[3-9].*/leap/' > spine5
\\
assemble spine1 spine2 spine3 spine4 spine5 > tempfile
\\
egrep -n 'P1.*\^{}.*\^{}.*leap.*leap|P1.*v.*v.*leap.*leap' tempfile}


\par 

\\
\begin{itemize}
\item 

\textbf{Next Chapter}
\item 

\textbf{Previous Chapter}
\item 

\textbf{Table of Contents}
\item 

\textbf{Detailed Contents}
\\\\

� Copyright 1999 David Huron
\end{document}
