% This file was converted from HTML to LaTeX with
% Tomasz Wegrzanowski's <maniek@beer.com> gnuhtml2latex program
% Version : 0.1
\documentclass{article}
\begin{document}



  
  
    
      
      
      
    
  



\\
\\

\section*{Chapter32}


[\textit{Previous Chapter}]
[\textit{Contents}]
[\textit{Next Chapter}]


\section*{The Shell (IV)}



In research applications, it is impossible to anticipate all
the types of manipulations we might want to carry out.
For some tasks, we will need to write our own software
to carry out specific operations of interest.
Fortunately, many specialized tasks require only a brief program
to achieve the goal.
The Humdrum tools can be used in conjunction with user-developed
software to carry out specific tasks.

\par 
Many users will already have some programming ability and
will be able to apply this knowledge using their preferred
programming language.
For those users who have less programming background,
it may be useful to learn some basic programming skills.
While the shell provides a useful programming environment,
for more complex tasks, it is better to use one of the many
good programming languages.

\par 
For data manipulation tasks comparable to those described in this book,
the most appropriate programming language include
\textbf{perl}
and
\textbf{awk}.
The
\textbf{awk}
programming language is especially useful for text processing,
retrieving, transforming, reducing, and validating text data.
The
\textbf{perl}
programming language provides even more extensive capabilities,
but requires a somewhat greater effort to learn.
For research-oriented programming,
\textbf{perl}
is the programming language of choice.
However, for this brief introduction we will describe features
of the
\textbf{awk}
programming language.
Awk is a so-called "scripted" language.
It is easy to learn but nevertheless quite powerful.


\subsection*{The \textit{awk} Programming Language}

\par 
Awk programs can be executed from the shell command line.
A simple program is the following:

\par 

\texttt{awk '\{print "hello"\}'}


\par 
The
\textbf{awk}
command invokes the awk program interpreter.
The material within the single quotes is the actual program.
Once the program is started, it is is executed once each time
you type the carriage return or ENTER key.
To stop the program, simply type control-D (on UNIX
systems) or control-Z (on DOS systems).

\par 
In the default configuration, an awk program will be executed
once for each line of input.
If no input file is specified, then "standard input"
is assumed.
That is, input will come from either data arriving through a pipeline,
or data typed at the keyboard.


\subsection*{Automatic Parsing of Input Data}

\par 
Each line of input data is automatically assigned to the awk variable \texttt{\$0}.
This means that the command

\par 

\texttt{awk '\{print \$0\}'}


\par 
will simply echoe each line of input as the output.
Similarly, the following command will print each line of input preceded by
a colon and a space:

\par 

\texttt{awk '\{print ": " \$0\}'}


\par 
For any input line, awk also automatically parses the data into individual
tokens or fields.
A token is deemed to be any sequence of characters that is
separated from other tokens by any blank space such as spaces or tabs.
The first data token is automatically assigned to an awk variable \texttt{\$1}.
The second data token is assigned to the variable \texttt{\$2}, and so on.
For example, suppose a program encountered the following input line:

\par 

\texttt{243xyz 3    29   \#\%\$	**     Ullyses 234-034}


\par 
The variables would be automatically assigned as follows:

\par 

\texttt{\$1 = 234xyz
\\
\$2 = 3
\\
\$2 = 29
\\
\$4 = \#\%\$
\\
\$5 = **
\\
\$6 = Ullyses
\\
\$7 = 234-034}


\par 
Given this input, the command

\par 

\texttt{awk '\{print \$2 + \$3\}'}


\par 
will print the sum of \texttt{\$2} and \texttt{\$3}, namely 32.


\subsection*{Arithmetic Operations}

\par 
Suppose that we have two
\texttt{**semits}
spines as input and we would like
to print the semitone difference between the two parts for each sonority.
Typically, the higher part is placed in the right-most spine,
so it makes most sense to subtract \$1 from \$2.
Negative numbers mean that the nominally lower part has crossed above
the nominally higher part:

\par 

\texttt{awk '\{print \$1 - \$2\}'}


\par 
In addition to addition and subtraction,
other possible arithemtic operators include 
the slash (/) for division, the asterisk (*) for multiplication,
the caret (\^{}) for exponentiation, and the percent sign (\%) for modulo arithmetic.
Parentheses can be used to clarify the order of operations.
For example, the following command prints the product of the first and
second tokens (\$1 * \$2) divided by the third token raised to the fourth token power:

\par 

\texttt{awk '\{print (\$1 * \$2) / (\$3\^{}\$4)\}'}


\par 
As we have already seen, character strings can also be included in print statements.
For example, we might want to print the first and third input tokens separated
by a tab:

\par 

\texttt{awk '\{print \$1 "$\backslash$t" \$3\}'}



\subsection*{Conditional Statements}

\par 
Often we'd like to avoid processing certain records.
For example, we might wish to avoid processing barlines.
The awk
\textbf{if}
statement can be used to restrict the operation to particular circumstances.
Consider the following awk program:

\par 

\texttt{awk '\{if (\$0 !~/\^{}=/) print \$1 - \$2\}'}


\par 
The
\textbf{if}
condition is given in parentheses.
The string given between the slashes (\texttt{/\^{}=/}) is a regular expression:
in this case, it identifies any equals sign that occurs at the beginning
of an input line.
The tilde means "match" and the exclamation mark means "not".
Hence the program means: if the entire line (\texttt{\$0}) does not match (\texttt{!~})
an equals sign occuring at the beginning of the line (\texttt{/\^{}=/}),
then print the value of the first token minus the value of the second token
(\texttt{print \$1 - \$2}).

\par 
Awk also provide an
\textbf{else}
condition.
The syntax is:

\par 

if (condition)

[then] \{do something\}
\\
else \{do something else instead\}



\par 
For Humdrum inputs, we may want to avoid processing comments and interpretations.
Whenever we encounter a comment or interpretation, we might
simply echo the input record in the output:

\par 

\texttt{awk '\{if(\$0 ~/\^{}[*!]/) \{print \$0\} else \{print \$1 - \$2\}\}'}


\par 
Sometimes we might simply want to do nothing at all when we encounter
a comment or interpretation:

\par 

\texttt{awk '\{if(\$0 ~/\^{}[*!]/) \{\} else \{print \$1 - \$2\}\}'}


\par 
Recall that input tokens in awk are separated by any blank space
such as spaces or tabs.
This means that a Humdrum multiple-stop will be treated as containing
two or more tokens.
We can avoid this situation by explicitly telling awk to
assign the "field separate" (FS) to the tab character.
For example, the following program prints the value in the third spine
of a Humdrum input.
Without reassigning the field separator, the third token might be
the third element of a multiple-stop in the first spine,
or the second element of a multiple-stop appearing in the second spine.

\par 

\texttt{awk '\{FS="$\backslash$t"; print \$3\}'}


\par 
Notice the use of the semicolon to separate individual instructions.


\subsection*{Assigning Variables}

\par 
Within an awk program, the user can assign and manipulate variables
that store particular values.
Variables may hold numerical values or they may hold character strings.
In the following examples, the value 178 is assigned to the variable `\texttt{A}';
the value 2.2 is assigned to the variable `\texttt{number}';
and the character string "\texttt{Dear Gail}" is assigned to the
variable `\texttt{salutation}':

\par 

\texttt{A=178
\\
number = 2.2
\\
salutation = "Dear Gail"}


\par 
Named variables can be used for various arithmetic operations.
For example:

\par 

\texttt{A=178+18
\\
number = 2.2 + A
\\
number\_squared = number \^{} 2}



\subsection*{Manipulating Character Strings}

\par 
Variables holding character strings can be concatenated together.
In the following example, after the first three assignments,
the variable \texttt{saluation}
will contain the character string "\texttt{Dear Craig}":

\par 

\texttt{opening = "Dear"
\\
space = " "
\\
name = "Craig"
\\
salutation = opening space name}


\par 
Awk provides a number of built-in functions for manipulating text.
One function (\textbf{gsub}) carries out global substitutions.
The syntax is:

\par 

\texttt{gsub("target-string","replacement-string",variable)}


\par 
For example, the following instruction changes all occurrences
of \texttt{X} to \texttt{Y} in a variable named \texttt{string}:

\par 

\texttt{gsub("X","Y",string)}


\par 
Suppose that we wanted to increment all measure numbers by 1.
Let's presume our input contains only a single spine.
First we test for the presence of the equal sign at the
beginning of the input record.
If the input is not a barline, then we simple print the line
in the output.
Otherwise we: (1) assign the input to the variable \texttt{barline},
(2) eliminate all non-numeric characters using \texttt{gsub},
(3) add one to the remaining numeric value, and
(4) output the new number preceded by the equal sign:

\par 

\texttt{awk '\{

if (\$0 !~/\^{}=/) \{print \$0\}
\\
else \{

barline = \$1
\\
gsub("[\^{}0-9]","",barline)
\\
barline = barline + 1
\\
print "=" barline
\\
\}

\}'}



\par 
Notice that we are at liberty to add spaces, tabs,
and newlines in order to improve the readability of our program.


\subsection*{The \textit{for} Loop}

\par 
Often we would like to repeat a process for several concurrent
spines.
For example, suppose we had four spines of
\texttt{**solfa}
data and we want to output the total number of leading-tones
for each sonority.
Awk provides a
\textbf{for}
instruction that allows us to cycle through a series of values.
The \textbf{for}-loop construction has the following syntax:

\par 

for (initial-value; condition-for-continuing; increment-action)

\{do something repeatedly\}



\par 
In the case of counting the number of leading-tones for each of four spines,
our program would be as follows:

\par 

\texttt{awk '\{

count = 0
\\
for (i=1; i<=4; i++)

\{if (\$i ~/ti/) count++\}

print count

\}'}


\par 
The initial value for the for-loop is 1 (\texttt{i=1});
each time the loop is executed the value of \texttt{i} is incremented
by 1 (\texttt{i++});
and the loop continues executing as long as \texttt{i} is less-than
or equal to 4 (\texttt{i<=4}).
The value \texttt{\$i} will take successive values so that the loop
will test whether each of \$1, \$2, \$3 and \$4 match the regular
expression \texttt{/ti/}.
For each match, the variable \texttt{count} is incremented by 1.
Finally, the value of count is printed.
The count is set to zero each time the program is run
(that is, for each line of input).

\par 
It would be nice if our program could adapt to inputs containing
any number of spines.
For each line of input, awk automatically identifies the number
of input tokens or fields and stores the value in the varible \texttt{NF}.
Simply replacing the number \texttt{4} by \texttt{NF} will achieve
our goal.
In our revised program we have also added some comments to clarify our code.
Like the shell, awk comments consist of material following the
octothorpe character (\#):
\textbf{}

\par 

\texttt{awk '\{

\# A program to count occurrences of the leading-tone.
\\
count = 0
\\
for (i=1; i<=NF; i++)

\{if (\$i ~/ti/) count++\}

print count

\}'}


\par 
A problem with the above script is that it will attempt to
count occurrences of \texttt{ti} in Humdrum comments, interpretations,
and barlines.
We can improve our program by echoing these in the output
without processing them.
Another refinement makes use of the awk
\textbf{next}
instruction.
Whenever a
\textbf{next}
statement is encountered, the program immediately moves on
to the next input line and begins processing again from the
start of the program.

\par 

\texttt{awk '\{

\# A program to count occurrences of the leading-tone.
\\
count = 0
\\
if (\$0 ~/\^{}[!*=]/) \{print \$0; next\}
\\
for (i=1; i<=NF; i++)

\{if (\$i ~/ti/) count++\}

print count

\}'}


\par 
Although our output data will consist of a single column (spine)
of numbers, it is possible that an input will contain more than
one interpretation -$\,$- and so cause the output to fail to
conform to the Humdrum syntax.
Rather than simply echoing any interpretation records,
we might ensure that only a single interpretation is generated
for the output.
First, we might look for exclusive interpretations (beginning \texttt{**})
and output a suitable interpretation of our own (e.g., \texttt{**leading-tones}).
In the case of tandem interpretations (beginning with only a single asterisk),
we could output a single null interpretation.
Similarly, when we encounter a barline, we might ensure that only
one barline token is output.
Finally, we should remain vigilant for spine-path terminators (\texttt{*-})
and ensure that our output is similarly properly terminated.
The revised program is as follows:

\par 

\texttt{awk '\{

\# A program to count occurrences of the leading-tone.
\\
count = 0
\\
if (\$0 ~/\^{}**/) \{print "**leading-tones"; next\}
\\
if (\$0 ~/\^{}*-/) \{print "*-"; next\}
\\
if (\$0 ~/\^{}*[\^{}*]/) \{print "*"; next\}
\\
if (\$0 ~/\^{}!!/) \{print \$0; next\}
\\
if (\$0 ~/\^{}=/) \{print \$1; next\}
\\
\{
\\
for (i=1; i<=NF; i++)

\{if (\$i ~/ti/) count++\}

print count

\}'}


\par 
Of course there are many other features of the awk programming
language that we have not described here.
These features include associative arrays,
built-in variables,
string-processing functions,
user-defined functions,
system calls,
begin and end blocks,
other control-flow statements,
and pipes and file manipulations.



\subsection*{Reprise}

\par 
In this chapter we have introduced some features of the
\textbf{awk}
pattern/action language.
A programming language, like
\textbf{awk}
or
\textbf{perl}
can be used to transform data in highly specific and specialized ways.
The power of Humdrum is significantly enhanced when users are
able to create their own specialized filters.

\\
\begin{itemize}
\item 

\textbf{Next Chapter}
\item 

\textbf{Previous Chapter}
\item 

\textbf{Table of Contents}
\item 

\textbf{Detailed Contents}
\\\\

� Copyright 1999 David Huron
\end{document}
