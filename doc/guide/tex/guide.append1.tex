% This file was converted from HTML to LaTeX with
% Tomasz Wegrzanowski's <maniek@beer.com> gnuhtml2latex program
% Version : 0.1
\documentclass{article}
\begin{document}



  
  
    
      
      
      
    
  



\section*{Appendix I}



\section*{Reference Records}



Reference records are formal ways of encoding "library-type"
information pertaining to a Humdrum document.
Reference records provide standardized ways of
encoding bibliographic information
-$\,$-  suitable for computer-based access.

\par 
A perpetual problem with reference information pertains to the
language in which information is represented.
Humdrum provides comprehensive methods for dealing with
multiple languages.
These methods are described below, and Humdrum
users are encouraged to become familiar with these
conventions.

\par 
Humdrum reference records are designated by three exclamation marks at
the beginning of a line, followed by a multi-letter code,
followed by an optional number, followed by a colon, followed by some text.

\par 
Over 80 reference codes are pre-defined in Humdrum.
Each of these reference records is described below under nine
categories:
(1) authorship information,
(2) performance information,
(3) recording information,
(4) work identification information,
(5) imprint information,
(6) copyright information,
(7) analytic information,
(8) historical information, and
(9) representation information.
A final section discusses how to cite electronic documents.


\subsection*{Accommodating Different Languages}
\par 
Humdrum attempts to accommodate all languages.
Reference information can be encoded using the original language
of the source material or document.
In addition, reference information can be encoded in
various translations that help end users.
Humdrum uses the

\textbf{International Standards Organization ISO 639-2}
standard for representing languages.
This standard provides 3-letter codes for more than 4,000
existing and historical languages.
For example, the ISO standard codes for English, French and Swahili are
\texttt{
eng
}
\texttt{
fre
}
and
\texttt{
swa
}
respectively.
(A table of 50 common languages is given at the end of this appendix.)

\par 

\par 
Humdrum allows any type of reference information to be encoded
in any or all of these languages.
In general, the \textit{primary} reference information is encoded using
the original language of the source material.
Consider the case of a Friuli folksong
(Friuli is a dialect of Italian).
The title of the song should be encoded in the original
language, but it may also be useful to provide
translations in Italian and English.

\par 
The principal reference code indicating title is
\texttt{
!!!OTL:
}
(see below).
Following the OTL reference code, the language of encoding
can be indicated by appending an "at" sign (@),
followed by the 3-letter ISO language code.
Here the title is rendered in three languages:

\texttt{
!!!OTL@@FUR: Ai preit la biele stele
\\
!!!OTL@ITA: Ho pregato la buona stella
\\
!!!OTL@ENG: The Good Star
}

Notice the use of the double "at" sign (i.e., "@@").
This convention is used to indicate that Friuli is
the primary reference language.
The title can also be encoded without any language
designation:

\texttt{
!!!OTL: Ai preit la biele stele
}

This implies that the title is already rendered in the \textit{primary}
reference language.

\par 




\subsection*{Authorship Information}

\par 
\textbf{!!!COM:}

\textbf{Composer's name}.
In some cases, opinions differ regarding the best spelling of a composer's name.
If so, all common spellings should be given -$\,$- each alternative
separated from the previous by a semicolon.
E.g.

\par 

\texttt{
!!!COM: Chopin, Fryderyk; Chopin, Frederick
}


\par 
With respect to accents, refer to the discussion concerning
the \textbf{!!!RLN:} reference record (see below).
If a work was composed by more than one composer,
then each composer's name should appear on a separate \textbf{!!!COM:} record
with a number designation prior to the colon.
For example,

\par 

\texttt{
!!!COM1: Composer, A.
\\
!!!COM2: Composer, B.
}




\par 
\textbf{!!!COA:}
\textit{Attributed composer}.
This may include attributions known to be false.
Several attributions may be combined on a single record by separating
each name by a semicolon.
Note that if a document contains both \textbf{!!!COA:} and \textbf{!!!COM:}
records, then the attributed composer is explicitly assumed to be false.



\par 
\textbf{!!!COS:}
\textit{Suspected composer}.
This reference code indicates the belief of the editor or producer
of the document as to the true identity of the composer(s).
If more than one composer is suspected, each name should appear
on a separate \textbf{!!!COS:} record with a number
designation prior to the colon.




\par 
\textbf{!!!COL:}
\textit{Composer's abbreviated, alias, or stage name}.
e.g. Madonna.




\par 
\textbf{!!!COC:}
\textit{Composer(s) corporate name}.
Corporate names may include the names of popular groups
(especially when the actual composer is not known).
Corporate names may also include business names,
e.g. Muzak.





\par 
\textbf{!!!CDT:}
\textit{Composer's dates}.
The birth and death dates should be encoded using the

\texttt{**Zeit}
format described in the \textit{Humdrum Reference Manual}.
The \texttt{**Zeit} format provides a highly refined representation,
including methods for representing uncertainty, approximation,
and boundary dates (e.g. prior to ..., after ...).



\par 
\textbf{!!!CNT:}
\textit{Nationality of the composer}.
This reference information is encoded using the language of the nationality.
Thus a German composer is encoded as \texttt{Deutscher} rather
than "German", and a French composer is encoded as
\texttt{Francais} rather than "French."
Of course the specific language can be explicitly encoded using
the 3-letter language codes described above.
(e.g., \texttt{!!!CNT@@FRE: Francais}).
Where the composer changed nationality, successive nationalities
should be listed (in chronological order) separated by semicolons.



\par 
\textbf{!!!LYR:}
\textit{Lyricist}.
The name of the lyricist.
If more than one lyricist was involved in the work,
then each lyricist's name should appear on a separate \textbf{!!!LYR:} record
with a number designation prior to the colon.
If the composer was also the lyricist, this should be explicitly
encoding using the independent \textbf{!!!LYR:} record -$\,$-
rather than implicitly assumed.



\par 
\textbf{!!!LIB:}
\textit{Librettist}.
The name of the librettist.
If more than one librettist was involved in the work,
then each librettist's name should appear on a separate \textbf{!!!LIB:} record
with a number designation prior to the colon.
If the composer was also the librettist, this should be explicitly
encoding using the independent \textbf{!!!LIB:} record -$\,$-
rather than implicitly assumed.



\par 
\textbf{!!!LAR:}
\textit{Arranger}.
The name of the arranger.
If more than one arranger was involved in the work,
then each arranger's name should appear on a separate \textbf{!!!LAR:} record
with a number designation prior to the colon.



\par 
\textbf{!!!LOR:}
\textit{Orchestrator}.
The name of the orchestrator.
If more than one orchestrator was involved in the work,
then each orchestrator's name should appear on a separate \textbf{!!!LOR:} record
with a number designation prior to the colon.



\par 
\textbf{!!!TXO:}
\textit{Original language of vocal/choral text}.
The name of the language should be encoded in that language.
For example, \texttt{russki} rather than \texttt{Russian}.



\par 
\textbf{!!!TXL:}
\textit{Language of the
\textbf{encoded}
vocal/choral text}.
The name of the language should be encoded in the language
used for encoding.
For example, \texttt{Italiano} rather than \texttt{Italian}.



\par 
\textbf{!!!TRN:}
\textit{Translator of text}.
The name of the translator of any vocal, choral, or dramatic text.
If more than one translator was involved in the work,
then each translator's name should appear on a separate \textbf{!!!TRN:} record
with a number designation prior to the colon.


\subsection*{Recording Information}

\par 
Humdrum representations may encode information pertaining
to sound recordings (such as sound-based analyses).
For information derived from sound recordings
the following reference records may be pertinent.



\par 
\textbf{!!!RTL:}
\textit{Title of album}.



\par 
\textbf{!!!RMM:}
\textit{Manufacturer or sponsoring company}.
The company or organization responsible for the release,
distribution, and/or manufacturing of the recording.



\par 
\textbf{!!!RC\#:}
\textit{Recording company's catalogue number}.
The album's numerical designation.



\par 
\textbf{!!!RRD:}
\textit{Date of release}.
The release date should be encoded using the

\texttt{**date}
format described in the
\textit{Humdrum Reference Manual}.



\par 
\textbf{!!!RLC:}
\textit{Place of recording}.
(Local language should be used.)



\par 
\textbf{!!!RNP:}
\textit{Name of the producer}.



\par 
\textbf{!!!RDT:}
\textit{Date of recording}.
The date of recording should be encoded using the

\texttt{**date}
format described in the
described in the
\textit{Humdrum Reference Manual}.



\par 
\textbf{!!!RT\#:}
\textit{Track number}.


\subsection*{Performance Information}

\par 
Humdrum representations may encode performance-activity information
rather than (or in addition to) score-related information.
If the representation encodes a given performance
(such as a MIDI performance),
then the following reference records may be pertinent.



\par 
\textbf{!!!MPN:}
\textit{Performer's name}.
If more than one performer was involved in the work,
then each performer's name should appear on a separate \textbf{!!!MPN:} record
with a number designation prior to the colon.



\par 
\textbf{!!!MPS:}
\textit{Suspected performer}.
If more than one performer is suspected, each name should appear
on a separate \textbf{!!!MPS:} record with a number designation
prior to the colon.



\par 
\textbf{!!!MRD:}
\textit{Date of performance}.
The performance date should be encoded using the

\texttt{**date}
format described in the
\textit{Humdrum Reference Manual}.



\par 
\textbf{!!!MLC:}
\textit{Place of performance}.
(Local language should be used.)



\par 
\textbf{!!!MCN:}
\textit{Name of the conductor of the performance}.



\par 
\textbf{!!!MPD:}
\textit{Date of first performance}.
The date of first performance should be encoded using the

\texttt{**date}
format described in the
described in the
\textit{Humdrum Reference Manual}.


\subsection*{Work Identification Information}



\par 
\textbf{!!!OTL:}
\textbf{Title.}
The title of the specific section or segment encoded in the current file.
Titles must be rendered in the original language,
e.g.
\textit{Le sacre du printemps.}
(Title translations are encoded using other reference records.)



\par 
\textbf{!!!XEN:}
\textit{Translated title (in English)}.
(Note that reference codes are also available for translations to
languages other than English, French, German, or Japanese.)



\par 
\textbf{!!!XFR:}
\textit{Translated title (in French)}.
(Note that reference codes are also available for translations to
languages other than English, French, German, or Japanese.)



\par 
\textbf{!!!XDE:}
\textit{Translated title (in German)}.
(Note that reference codes are also available for translations to
languages other than English, French, German, or Japanese.)



\par 
\textbf{!!!XNI:}
\textit{Translated title (in Japanese)}.
(Note that reference codes are also available for translations to
languages other than English, French, German, or Japanese.)



\par 
\textbf{!!!OTP:}
\textit{Popular Title}.
This reference record encodes well-known or alias titles
such as "Pathetique Sonata".



\par 
\textbf{!!!OTA:}
\textit{Alternative title}.
This reference record encodes earlier or alternate titles.



\par 
\textbf{!!!OPR:}
\textit{Title of larger (or parent) work} from which the encoded piece is a part.
For example, "Gute Nacht" (OTL) from
\textit{Winterreise}
(OPR).



\par 
\textbf{!!!OAC:}
\textit{Act number}.
For operas and musicals, this reference record encodes the
act number as an Arabic (rather than Roman) numeral.
The number may be preceded by the word "Act" as in \texttt{Act 3}.



\par 
\textbf{!!!OSC:}
\textit{Scene number}.
For operas and musicals, this reference record encodes the
scene number as an Arabic (rather than Roman) numeral.
The number may be preceded by the word "Scene" as in \texttt{Scene 3}.



\par 
\textbf{!!!OMV:}
\textit{Movement number}.
For multi-movement works such as sonatas and symphonies,
this reference record encodes the movement number as
an Arabic (rather than Roman) numeral.
The number may be preceded by the word "Movement" or "mov."
etc., as in \texttt{mov. 3}.



\par 
\textbf{!!!OMD:}
\textit{Movement designation or movement name}.
Typically movements may be named according to the tempo
(e.g. "Allegro ma no troppo") or according to a style, genre
or form (e.g. "Fugue"), or according to a programmatic title
(e.g. "In Full Flower").



\par 
\textbf{!!!OPS:}
\textit{Opus number}.
The number may be preceded by the word "Opus" as in \texttt{Opus 23}.
Once again, Arabic numerals are used.



\par 
\textbf{!!!ONM:}
\textit{Number}.
The number may be preceded by the abbreviations "No." or "Nr."
as in \texttt{No. 4}.



\par 
\textbf{!!!OVM:}
\textit{Volume}.
The volume number may be preceded by the abbreviation "Vol."
as in \texttt{Vol. 2}.
Arabic numbers are used.



\par 
\textbf{!!!ODE:}
\textit{Dedication}.
Name of person or organization to whom the work is dedicated.
If the work was dedicated to more than one person,
then each dedicatee's name should appear on a separate \textbf{!!!ODE:} record
with a number designation prior to the colon.



\par 
\textbf{!!!OCO:}
\textit{Commission}.
Name of person or organization that commissioned the work.
If the work was commissioned by more than one person,
then each commissioner's name should appear on a separate \textbf{!!!OCO:} record
with a number designation prior to the colon.



\par 
\textbf{!!!OCL:}
\textit{Collector}.
Name of person who collected or transcribed the work.
If the work was collected by more than one person,
then each collector's name should appear on a separate \textbf{!!!OCL:} record
with a number designation prior to the colon.



\par 
\textbf{!!!ONB:}
\textit{Free format note} related to the title or identity of the encoded work.
Nota bene.
If more than one such note is encoded,
each should appear on a separate \textbf{!!!ONB:} record
with a number designation prior to the colon.



\par 
\textbf{!!!ODT:}
\textit{Date of composition}.
The date (or period) of composition should be encoded using the

\texttt{**date}
or

\texttt{**Zeit}
formats described in the
\textit{Humdrum Reference Manual}.
The \texttt{**date} and \texttt{**Zeit} formats provides a highly refined
representation, including methods for representing uncertainty, approximation,
and boundary dates (e.g. prior to ..., after ...).



\par 
\textbf{!!!OCY:}
\textit{Country of composition}.
Local names should be used, such as `Espana'.



\par 
\textbf{!!!OPC:}
\textit{City, town or village of composition}.
Local names should be used, such as `Den Haag.'


\subsection*{Group Information}


\par 
\textbf{!!!GTL:}
\textit{Group Title}.
A logical collection of works
such as the "London Symphonies" by Haydn,
or the four concertos by Vivaldi
forming "The Seasons".



\par 
\textbf{!!!GAW:}
\textit{Associated Work}.
Some works are associated with other works, such as plays,
novels, paintings, films, or other musical works.
E.g. Mendelssohn's Overture to Shakespeare's \textit{Midsummer Night's Dream}.
This reference allows associated works to be explicitly identified
by author and title.
E.g. 

\par 

\texttt{
!!!GAW: St�phane Mallarm�, L'Apr�s-midi d'un faune; [The Afternoon of a Faun].
}



\par 
\textbf{!!!GCO:}
\textit{Collection designation}.
This is a free-form text record that can be used to identify
a collection of pieces,
such as works appearing in a compendium or anthology.
E.g. Norton Scores, Smithsonian Collection, Burkhart Anthology.


\subsection*{Imprint Information}



\par 
\textbf{!!!PUB:}
\textit{Publication status}.
This reference record identifies whether the document has ever been "published".
One of the following English terms may appear: \texttt{published}
or \texttt{unpublished}.



\par 
\textbf{!!!PPR:}
\textit{First publisher}.
Name of the first publisher of the work.



\par 
\textbf{!!!PDT:}
\textit{Date first published}.
The date of publication should be encoded using the

\texttt{**date}
format described in the
\textit{Humdrum Reference Manual}.



\par 
\textbf{!!!PPP:}
\textit{Place first published}.
(Local language should be used.)



\par 
\textbf{!!!PC\#:}
\textit{Publisher's catalogue number}.
This should not be confused with better known scholarly catalogues,
such as those of K�chel, Hoboken, etc.



\par 
\textbf{!!!SCT:}
\textit{Scholarly catalogue abbreviation and number}.
E.g. BWV 551



\par 
\textbf{!!!SCA:}
\textit{Scholarly catalogue (unabbreviated) name}.
E.g.\texttt{Koechel 117}.



\par 
\textbf{!!!SMS:}
\textit{Manuscript source name}.
For unpublished sources, the manuscript source name.



\par 
\textbf{!!!SML:}
\textit{Manuscript location}.
For unpublished sources, the location of the manuscript source.



\par 
\textbf{!!!SMA:}
\textit{Acknowledgement of manuscript access}.
This reference information may be used to encode a free format
acknowledgement or note of thanks to a given manuscript owner for
scholarly or other access.


\subsection*{Copyright Information}



\par 
\textbf{!!!YEP:}
\textit{Publisher of electronic edition}.
This reference identifies the publisher of the electronic document.



\par 
\textbf{!!!YEC:}
\textit{Date and owner of electronic copyright}.
This reference identifies the year and owner of the copyright for the
electronic document.



\par 
\textbf{!!!YER:}
Date electronic edition released.



\par 
\textbf{!!!YEM:}
\textit{Copyright message}.
This record conveys any special text related to copyright.
It might convey a simple warning (e.g. "All rights reserved."),
convey registration or licensing information,
or indicate that the document is shareware.



\par 
\textbf{!!!YEN:}
\textit{Country of copyright}.
This reference identifies the country in which the electronic document
was created, or where the copyright was established.
In effect, it identifies the country under whose laws the
copyright declaration is to be interpreted.



\par 
\textbf{!!!YOR:}
\textit{Original document}.
This reference identifies any original source or sources from which
encoded document was prepared.
Note that original documents may themselves be copyrighted,
and that permission may be required in order to create an electronic
derivative document.
Original documents may also have lapsed copyrights.



\par 
\textbf{!!!YOO:}
\textit{Original document owner}.
If the electronic document was prepared from a copyrighted original
document, this reference identifies the copyright owner of the
original document.
Note that unless the electronic and original documents have the same owner,
some licensing agreement or other legal arrangement is necessary
in order to create an electronic derivative document.



\par 
\textbf{!!!YOY:}
\textit{Original copyright year}.
If the electronic document was prepared from a copyrighted original
document, this reference identifies the year of copyright for the
original document.
Note that some licensing agreement or other legal arrangement
is necessary in order to create an electronic derivative document.



\par 
\textbf{!!!YOE:}
\textit{Original editor}.
The editor of the original
document from which the electronic edition was prepared.
Note that some licensing agreement or other legal arrangement
may be necessary in order to create an electronic derivative document.



\par 
\textbf{!!!EED:}
\textit{Electronic Editor}.
Name of the editor of the electronic document.
If more than one editor was involved in the work,
then each editor's name should appear on a separate \textbf{!!!EED:} record
with a number designation prior to the colon.



\par 
\textbf{!!!ENC:}
\textit{Encoder of the electronic document}.
This reference identifies the name of the person or persons who encoded
the electronic document.
(Not to be confused with the electronic editor.)
If more than one encoder was involved in the work,
then each encoder's name should appear on a separate \textbf{!!!ENC:} record
with a number designation prior to the colon.



\par 
\textbf{!!!EMD:}
\textit{Document modification description}.
This record type is used to chronicle all modifications made
to the original electronic document.
EMD records should indicate the date of modification,
the name of the person making the modification, and a brief
description of the type of modification made.
For each successive modification, a separate \textbf{!!!EMD:} record should appear
with a number designation prior to the colon.



\par 
\textbf{!!!EEV:}
\textit{Electronic edition version}.
This reference identifies the specific editorial version of the work.
e.g. Version 1.3g
Only a single \textbf{!!!EEV:} record can appear in a given electronic document.



\par 
\textbf{!!!EFL:}
\textit{File number}.
Some files are part of a series or group of related files.
This record indicates that the current document is file \textit{x}
in a group of \textit{y} files.
The two numbers are separated by a slash as in:

\par 

\texttt{
!!!EFL: 1/4
}




\par 
\textbf{!!!EST:}
\textit{Encoding status}.
This record indicates the current status of the document as it is
being produced.
Free-format text may indicate that the encoding is in-progress,
list tasks remaining, or indicate that the encoding is complete.
\textbf{!!!EST:} records are normally eliminated prior to
distribution of the document.



\par 
\textbf{!!!VTS:}
\textit{Checksum validation number}.
This reference encodes the checksum number for the file -$\,$- excluding
the \textbf{!!!VTS:} record itself.
When this record is eliminated from the file,
any POSIX.2 standard
\textbf{cksum}
command can be used to determine whether the file originates
with the publisher, or whether it has been modified in some way.
(See the Humdrum
\textbf{veritas}
command described in Section 4.)
Note that this validation process is easily circumvented by
malicious individuals.
For true security, the checksum value should be compared
with a printed list of checksums provided by the electronic publisher.


\subsection*{Analytic Information}


\par 
\textbf{!!!ACO:}
\textit{Collection designation}.
This is a free-form text record that can be used to identify
a collection, set, or group of related works,
such as works appearing in a compendium or anthology.
E.g. Norton Scores, Smithsonian Collection, Jones Anthology.



\par 
\textbf{!!!AFR:}
\textit{Form designation}.
This is a free-form text record that can be used to identify
the form (if appropriate) of the work.
E.g. fuga, sonata-allegro, passacaglia, rounded binary, rondo.



\par 
\textbf{!!!AGN:}
\textit{Genre designation}.
This is a free-form text record that can be used to identify
the genre of the work.
E.g. opera, string quartet, barbershop quartet.



\par 
\textbf{!!!AST:}
\textit{Style, period, or type of work designation}.
This is a free-form text record that can be used to characterize
the style, period, or type of work.
This reference can include any term or terms
deemed appropriate by the producer of the document.
Designations might include keywords or keyphrases such as:
Baroque, bebop, Ecole Notre Dame, minimalist, serial, reggae,
slendro, heterophony, etc.



\par 
\textbf{!!!AMD:}
\textit{Mode Classification}.
A combined numerical/name system for mode identification
-$\,$- used especially for medieval monophonic and later polyphonic works.
Modes are indicated by numbers from 1 to 12,
followed by a semicolon,
followed by the corresponding written name
(with an initial upper-case character).
Permissible mode numbers and names are:


1;Dorian
2;Hypodorian
3;Phyrgian
4;Hypophyrgian
5;Lydian
6;Hypolydian
7;Mixolydian
8;Hypomixolydian
9;Ionian
10;Hypoionian
11;Aeolian
12;Hypoaeolian


Other non-standard mode names can be used at the discretion
of the electronic editor.



\par 
\textbf{!!!AMT:}
\textit{Metric Classification}.
Meters for a file may be classified as one of the following eight
categories: \texttt{simple duple}, \texttt{simple triple},
\texttt{simple quadruple}, \texttt{compound duple}, \texttt{compound triple},
\texttt{compound quadruple}, \texttt{irregular}, or \texttt{various}.



\par 
\textbf{!!!AIN:}
\textbf{Instrumentation.}
This reference is used to list all of the instruments (including voice)
used in the work.
Instruments should be encoded using the abbreviations specified
by the \texttt{*I} tandem interpretation described in
Appendix II.
Instrument codes must appear in alphabetical order separated by spaces.
(Note that alphabetical ordering is essential in order to facilitate
searches for specific combinations or subsets of instruments
using the
\textbf{grep}
command.)
E.g. 

\par 

\texttt{
!!!AIN: clars corno fagot flt oboe
}




\par 
\textbf{!!!ARE:}
\textit{Geographical region of origin}.
This reference identifies the geographical location
from which the work originates.
Location designations are encoded using the local language.
The location begins with the continent designation,
and becomes successively more refined.
If the information is available, refinement can continue
to suburban district or even street address.
\par 

\texttt{
!!!ARE: Europa, Mitteleuropa, Deutschland, Wuerttemberg, Sindelfingen
\\
!!!ARE: America, North America, United States of America, Ohio, Columbus
}




\par 
\textbf{!!!ARL:}
\textit{Geographical location of origin}.
Like the ARE record, this reference record
identifies the geographical location
from which a work originates.
Location designations are encoded using latitude
and longitude values -$\,$- suitable for creating maps.
The first numerical value indicates latitude (positive
values indicating North, negative values indicating South).
The second numerical value indicates longitude (positive
values only, indicating distance East from the central
meridian).
A slash separates the latitude and longitude values.
A series of trailing characters is used to indicate
the degree of accuracy of the location information:
\% (continent), @ (country), \# (province or state),
: (town or village).
For large regions such as countries or provinces,
the geographical center of the region is used.
\par 

\texttt{
!!!ARL: 51.5/10.5@
}





\subsection*{Historical and Background Information}


\par 
\textbf{!!!HAO:}
\textit{Aural History}.
This is a free-form text record used to relay
any story or stories about the origin, purpose
or background of the work.
This reference record is especially useful in
ethnomusicological materials, where a particular
story accompanies a song.
The story may be encoded using several successive HAO records.



\par 
\textbf{!!!HTX:}
\textit{Free-form Translation of Vocal Text}.
This is a free-form text record used to relay
a non-literal translation of a vocal text.
This reference record is again especially useful in
ethnomusicological materials.


\subsection*{Representation Information}



\par 
\textbf{!!!RLN:}
\textit{ASCII language setting}.
This reference identifies the "language" code in which
the file was encoded.
This is applicable only to computer platforms which provide
"extended ASCII" text capabilities (e.g. Danish or Spanish characters).



\par 
\textbf{!!!RDF:}
\textit{User-defined signifiers}.
All Humdrum representations provide some signifiers (ASCII characters) that
remain undefined.
Users are free to use these undefined signifiers as they choose.
When undefined signifiers appear in a given document,
the \textbf{!!!RDF**\textit{interp}:} code should
be used to specify what the signifiers denote.
Notice that the code RDF is followed by the name of the
interpretation to which the signifier definition applies.
In the following example, the letters "X" and "x" symbols that
are defined within a hypothetical **piano representation.
E.g.

\par 

\texttt{
!!!RDF**piano: X=hands cross, left over right
\\
!!!RDF**piano: x=hands cross, right over left
}




\par 
\textbf{!!!RDT:}
\textit{Date encoded}.
This reference uses the Humdrum

\texttt{**date}
format to identify the date(s) when the document was encoded.



\par 
\textbf{!!!RNB:}
\textit{Representation note}.
This reference provides a free-format text that conveys
some document-specific note related to matters of representation.



\par 
\textbf{!!!RWG:}
\textit{Representation warning}.
This reference may be used to encode explicit warnings
concerning the encoded material.


\subsection*{Electronic Citation}

\par 

\par 
Electronic editions of music might be cited in printed or other documents
by including the following information.
The "author" (e.g. \textbf{!!!COM:}),
the "title" -$\,$- either original title
(\textbf{!!!OTL:})
or translated title (e.g. \textbf{!!!XEN:}).
The editor (\textbf{!!!EED:}),
publisher (\textbf{!!!YEP:}),
date of publication and copyright owner
(\textbf{!!!YEC:}),
and electronic version
(\textbf{!!!EEV:}),
In addition, a full citation ought to include the validation
checksum (\textbf{!!!VTS:}).
This number will allow others to verify that a particular electronic
document is precisely the one cited.
A sample electronic citation might be:

\par 

Franz Liszt, Hungarian Rhapsody No. 8 in F-sharp minor (solo piano).
\\
Amsterdam: Rijkaard Software Publishers, 1994;
H. Vo$\backslash$o'r$\backslash$(hc'i$\backslash$o's$\backslash$(hc'ek (Ed.),
\\
Electronic edition version 2.1, checksum 891678772.


\par 
In Humdrum files it does not matter where reference records appear.
Since it is common for users to inspect the beginning of a file in order
to check whether the file is being properly processed, the number of
reference records at the beginning of the file should be kept to a
minimum.
A good habit is to place the composer, title of the work, and
copyright records at the beginning of the file, and to relegate
all other reference records to the end of the file.


\subsection*{ISO Language Codes}


\par 
The following table provides further language designation codes
not identified in the preceding discussion.
A

\textbf{complete list}
of ISO 639-2 language codes is available.

CodeLanguage
\textbf{@ALB}in Albanian
\textbf{@ARA}in Arabic
\textbf{@ARM}in Armenian
\textbf{@AZE}in Azeri
\textbf{@BEN}in Bengali
\textbf{@BUL}in Bulgarian
\textbf{@CHI}in Chinese
\textbf{@SCR}in Croatian
\textbf{@CZE}in Czech
\textbf{@DAN}in Danish
\textbf{@DUT}in Dutch
\textbf{@ENG}in English
\textbf{@EST}in Estonian
\textbf{@FIN}in Finnish
\textbf{@FRE}in French
\textbf{@GLA}in Gaelic
\textbf{@GER}in German
\textbf{@GRE}in Greek (modern)
\textbf{@HEB}in Hebrew
\textbf{@HIN}in Hindi
\textbf{@HUN}in Hungarian
\textbf{@ICE}in Icelandic
\textbf{@ITA}in Italian
\textbf{@JPN}in Japanese
\textbf{@JAV}in Javanese
\textbf{@KHM}in Khmer
\textbf{@KOR}in Korean
\textbf{@LIT}in Lithuanian
\textbf{@LAT}in Latin
\textbf{@LAV}in Latvian
\textbf{@MAL}in Malayalam
\textbf{@MON}in Mongolian
\textbf{@NOR}in Norwegian
\textbf{@POL}in Polish
\textbf{@POR}in Portugese
\textbf{@RUM}in Romanian
\textbf{@RUS}in Russian
\textbf{@SCC}in Serbian
\textbf{@SLO}in Slovak
\textbf{@SLV}in Slovenian
\textbf{@SPA}in Spanish, Castilian
\textbf{@SWA}in Swahili
\textbf{@SWE}in Swedish
\textbf{@TAM}in Tamil
\textbf{@THA}in Thai
\textbf{@TIB}in Tibetan
\textbf{@TUR}in Turkish
\textbf{@UKR}in Ukranian
\textbf{@URD}in Urdu
\textbf{@VIE}in Vietnamese
\textbf{@WEL}in Welsh
\textbf{@XHO}in Xhosa
\textbf{@ZUL}in Zulu


\\
\begin{itemize}
\item 

\textbf{Next Appendix}
\item 

\textbf{Previous Appendix}
\item 

\textbf{Table of Contents}
\item 

\textbf{Detailed Contents}
\item 

\textbf{Detailed Contents}
\\\\

� Copyright 1999, 2002 David Huron
\end{document}
